%%%%%%%%%%%%%%%%%%%%%%%%%%%%%%%%%%%%%%%%%%%%%%%%%%%%%%%%%%%%%
\begin{frame}[fragile]
\begin{algorithm}[H]
	\begin{algorithmic}[1]
		\STATE  Particionar $[0,T]$  %$\displaystyle dt =\frac{1}{N}$
		\FOR{$j=0$ to $N$} 
			\STATE $t_j=jdt$
		\ENDFOR
		\STATE Sea $B(t_j)=B_j$					
		\FOR {$j=1$ to $N$}
			\STATE $B_j=B_{j-1}+dB_j $ \qquad $dB_j\sim \sqrt{dt}N(01)$  		
		\ENDFOR
	\end{algorithmic}
\end{algorithm}
\end{frame}
%%%%%%%%%%%%%%%%%%%%%%%%%%%%%%%%%%%%%%%%%%%%%%%
\begin{frame}{C\'odigo}
 \only<+>{
  \begin{figure}
  \centering
      \tiny
   \lstset{language=python}
         \lstinputlisting[firstline=1,lastline=13]{BrownianPaths.py}
   \end{figure}
	}
\end{frame}
%%%%%%%%%%%%%%%%%%%%%%%%%%%%%%%%%%%%%%%%%%%%%%%
\begin{frame}{Trayectorias de MB}
\vspace{1.0cm}
\begin{overlayarea}{\textwidth}{\textheight}	
	\only<+>{	
		\begin{center}
			\includegraphics[width=\textwidth]{./IMAGENES/RW/MB(1).png}
		\end{center}
	}
	\only<+>{	
		\begin{center}
			\includegraphics[width=\textwidth]{./IMAGENES/RW/MB(2).png}
		\end{center}
	}
	\only<+>{	
		\begin{center}
			\includegraphics[width=\textwidth]{./IMAGENES/RW/MB(3).png}
		\end{center}
	}
	\end{overlayarea}
\end{frame}
%%%%%%%%%%%%%%%%%%%%%%%%%%%%%%%%%%%%%%%%%%%%%%%%
\begin{frame}
	\frametitle{Escalamiento}
		\vspace{-1.5cm}
	\uncover<+->{		
		\begin{empheq}[box={\Garybox[Recordemos]}]{equation*}
			B(t)-B(s) \sim \sqrt{t-s}N(0,1) \text{ son i.i.d }				
		\end{empheq}
	}				
	\uncover<+->{	
	\begin{block}{Podemos ver $B_N$ como suma de v.a.i.i.d}
		\begin{overlayarea}{\textwidth}{.45\textheight}
						
			\only<+>{
				Dados $t\in [0,T]$ y cualquier partición 
				$$
					0=t_0<t_1<\dots< t_{M-1}<t_M=t.
				$$
			}
			\only<+>{				
				Si usamos una suma telescópica tenemos
				$$
					B(t)	=\sum_{j=0}^m B(t_j)-B(t_{j-1}) \qquad (B(t_0=0)=0)
				$$	
			}
			
			\only<+->{			
				Con esto podemos  escalar los incrementos del MB a partir de 
				$$
					B_N =\sum_{j=0}^N dB_j	\qquad dB_j\sim \sqrt{dt} N(0,1) 
				$$
			}
			\only<+->{			
				Veamos una trayectoria y el código python.			
			}
		\end{overlayarea}
	\end{block}	
	}
\end{frame}
%%%%%%%%%%%%%%%%%%%%%%%%%%%%%%%%%%%%%%%
\begin{frame}{C\'odigo}
 \only<+>{
  \begin{figure}
  \centering
      \tiny
   \lstset{language=python}
         \lstinputlisting[firstline=14,lastline=29]{BrownianPaths.py}
   \end{figure}
	}
	\only<+>{
		 \begin{figure}
  \centering
      \tiny
   \lstset{language=python}
         \lstinputlisting[firstline=30,lastline=37]{BrownianPaths.py}
   \end{figure}
	
	}
\end{frame}
%%%%%%%%%%%%%%%%%%%%%%%%%%%%%%%%%%%%%%%%%%%%%%%%%%%%%%%%%%%%%%%%%%%%%%%%%%%%%%%
\begin{frame}
	\frametitle{MB con incrementos escalados}
	\vspace{1.5cm}	
	\framezoom<0><2>[border](0cm,4.5cm)(1.5cm,1.5cm)	
	\begin{overlayarea}{\textwidth}{\textheight}	
	%\only<+>{	
		\begin{center}
			\includegraphics[width=\textwidth]{./IMAGENES/RW/MBEscalado(1).png}
		\end{center}
	%}
	\end{overlayarea}
\end{frame}
%%%%%%%%%%%%%%%%%%%%%%%%%%%%%%%%%%%%%%%%%%%%%%%%%%%%%%%%%%%%%%%%%%%%%%%%%%%%%%%%%
\begin{frame}
	\frametitle{Distribuci\'on MB}
		\vspace{-1.7cm}		
		\begin{overlayarea}{\textwidth}{.68\textheight}	
			\only<+>{	
				\begin{center}
					\includegraphics[width=.95\textwidth]{./IMAGENES/RW/MBs.png}
				\end{center}
			}
			\only<+>{	
				\begin{center}
					\includegraphics[width=.95\textwidth]{./IMAGENES/RW/MBSigma.png}
				\end{center}
			}
	\end{overlayarea}
\end{frame}
