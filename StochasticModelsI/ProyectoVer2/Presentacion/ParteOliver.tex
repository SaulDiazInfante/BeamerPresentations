%%%%%%%%%%%%%%%%%%%%%%%%%%%%%%%%%%%%%%%%%%%%%%%%%%%%%%%%%%%%%%%%%%%%%%%%%%
\begin{frame}{Objetivos}
	 \uncover<2->{
	\begin{block}{General}
		Simular el Movimiento Browniano.
	\end{block}
	}
	\vspace{0.25cm}
	\uncover<3->{
	\begin{block}{Espec\'ificos}
		\begin{itemize}
			\uncover<3->{
    			\item Contruir el Movimiento Browniano a partir de una caminata Aleatoria.
            }
			\uncover<4->{
    			\item Identificar las propiedades de Movimiento Browniano.
            }
			\uncover<5->{
				\item Describir el Algoritmo a utilizar en la Simulaci\'on del Movimiento Browniano.
            }
   		\end{itemize}
    }
\end{block}
\end{frame}
%%%%%%%%%%%%%%%%%%%%%%%%%%%%%%%%%%%%%%%%%%%%%%%%%%%%%%%%%%%%%%%%%%%%%%%%%%%%%%%%%
\begin{frame}
	\frametitle{Plan de Charla}
    \tableofcontents[pausesections]
\end{frame}
%%%%%%%%%%%%%%%%%%%%%%%%%%%%%%%%%%%%%%%%%%%
	\begin{frame}{Historia}
		\begin{center}
			\movie[width=9.1cm,height=5.2cm,showcontrols=true,loop,poster]{}{BrownianMotion.flv}
			\end{center}							
		\end{frame}
%%%%%%%%%%%%%%%%%%%%%%%%%%%%%%%%%%%%%%%%%%%

%%%%%%%%%%%%%%%%%%%%%%%%%%%%%%%%%%%%%%%%%%
\begin{frame}{Caminata Aleatoria}
	\begin{overlayarea}{\textwidth}{.7\textheight}
	\only<1>{	
	\begin{center}
		\includegraphics[width=\textwidth]{./IMAGENES/RW/RandomWalk(1).png}	
	\end{center}
	}
	\only<2>{	
		\begin{center}
			\includegraphics[width=\textwidth]{./IMAGENES/RW/RandomWalk[2].png}	
		\end{center}
	}
	\only<3>{
		\begin{center}
			\includegraphics[width=\textwidth]{./IMAGENES/RW/RandomWalk[3].png}	
		\end{center}	
	}	
	\end{overlayarea}
\end{frame}

%%%%%%%%%%%%%%%%%%%%%%%%%%%%%%%%%%%%%%%%%%%
\begin{frame}{Construcci\'on}
	\uncover<2->{
		Consideremos
			$$\{ X_{n} \}_{n=1}^{\infty} \quad \quad \mbox{v.a.i.id},$$
		tales que
			$$P(X_{j} = h) = P(X_{j} = -h) = \frac{1}{2}.$$
	}
	\uncover<3->{
		Sea $Y_{\delta, h}(0) = 0$, y hagamos
		$$ Y_{ \delta,h}(n \delta ) = X_{1}+X_{2}+\cdots+X_{n}.$$
	}
	\uncover<4->{
		Para $t > 0$, definamos $Y_{\delta,h}(t)$ con $n\delta < t < (n + 1)\delta$ como
		$$Y_{\delta,h}(t) = \frac{(n + 1)\delta -t}{\delta} Y_{\delta,h}(n \delta ) +  \frac{t - n\delta}
		{\delta}Y_{\delta,h}((n+1) \delta ).$$
	}
	\uncover<5->{
	\begin{block}{Interrogante}
		?`Cu\'al es el limite de $Y_{\delta, h}$ cuando $\delta,h \longrightarrow 0 ?$
	\end{block}
}
\end{frame}
%%%%%%%%%%%%%%%%%%%%%%%%%%%%%%%%%%%%%%%%%%%%%
\begin{frame}{Construcci\'on}
	\begin{overlayarea}{\textwidth}{.6\textheight}
	\only<+->{
		Para $\lambda \in \mathbb{R}$ fijo, calculemos $\displaystyle\lim_{\delta,h \to 0}
		{\mathbb{E}\left[e^{ i\lambda Y_{\delta,h}(t)}\right]}$. \\
		\hypertarget{cns:Limite}{}
		\hyperlink{dfn:FuncionCaracteristica}{\beamergotobutton{caracteristica}}
		\uncover<+->{
		Tomando $t=n\delta$, se sigue que
		\begin{eqnarray*}
			\mathbb{E}\left[e^{ i\lambda 					
			Y_{\delta,h}(t)}\right]&=&
			\prod_{j=1}^{n}\mathbb{E}\left[e^{i\lambda X_{j}}\right] \\
				&=&\left( \mathbb{E}\left[e^{i\lambda X_{j}}\right]\right)^{n}\\
				&=&\left(\frac{1}{2}e^{i\lambda h}+\frac{1}{2}e^{-i\lambda h}\right)^{n}\\
				&=&\left(cos(\lambda h)\right)^{n}\\
				&=&\left(cos(\lambda h)\right)^{\frac{t}{\delta}}.
		\end{eqnarray*}
		}
	}
\end{overlayarea}
\end{frame}
%%%%%%%%%%%%%%%%%%%%%%%%%%%%%%%%%%%%%%%%%%%
\begin{frame}{Construcci\'on}
	Sea $$ u =\left(cos(\lambda h)\right)^{\frac{1}{\delta}},$$ as\'i
	$$ u \approx e^{ - \frac{1}{2\delta}\lambda^{2}h^{2}}.$$
	\uncover<2->{
	Luego
	$$
		\mathbb{E}\left[ e^{i\lambda Y_{\delta,h}(t)}\right]\approx e^{- \frac{1}
		{2\delta}t\lambda^{2}h^{2}}.
	$$

	}
	\uncover<3->{
	Si $h^{2} = \delta$, entonces
	$$
		\displaystyle\lim_{\delta,h \to 0}{\mathbb{E}\left[ e^{i\lambda Y_{\delta,h}(t)}\right]} 
		= e^{ - \frac{1}{2}t\lambda^{2}},
		\quad \quad \lambda \in \mathbb{R}.$$
}
\end{frame}
%%%%%%%%%%%%%%%%%%%%%%%%%%%%%%%%%%%%%%%%%%%%%%%%%%%%%%%%%%%%%%%%%%%%%%%%%%%%%%%%%%%
\begin{frame}{Construcci\'on}
	\begin{Teorema} 
		Sea $Y_{\delta,h}(t)$ una caminata aleatoria que inicia en $0$ 
		de saltos $h$ y $-h$  con igual probabilidad en los tiempos 
		$\delta, 2\delta,3\delta, \ldots $.  
		Asumamos que $h^{2} = \delta$. 
		Entonces para cada $t \geq 0$, el limite
		$$ B(t) = \displaystyle\lim_{\delta \to 0}{Y_{\delta,h}(t)},$$
		existe en distribuci\'on. Adem\'as, tenemos que
		$$
			\mathbb{E}\left[e^{i\lambda B(t)}\right] 
			=e^{- \frac{1}{2}t\lambda^{2}}, \quad \quad \lambda \in 	\mathbb{R}.
		$$
	\end{Teorema}
\end{frame}
%%%%%%%%%%%%%%%%%%%%%%%%%%%%%%%%%%%%%%%%%%%%%%%%%%%%%%%%%%%%%%%%%%%%%%%%%%%%%%%%%%%%