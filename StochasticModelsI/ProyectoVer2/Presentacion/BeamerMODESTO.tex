\documentclass[spanish,10pt,xcolor=dvipsnames,table]{beamer}
\usepackage[spanish,activeacute]{babel}
\usepackage{color}
\usepackage{amsmath}
\usepackage{amssymb}
\usepackage{graphicx}
\usepackage{latexsym}
\usepackage{ucs}
\usepackage[utf8]{inputenc}
\usepackage{bibunits}
\usepackage[amssymb]{SIunits}
\usepackage{sistyle}
\usepackage{times}
\usepackage{tikz}
\usepackage{verbatim}
\usepackage[utf8]{inputenc}
%\usepackage{beamerthemesplit}
\usepackage{multirow}
\usepackage{hyperref}
%\usepackage{graphics,epsfig, subfigure}
\usepackage{url}
\usetikzlibrary{arrows,shapes}
\theoremstyle{plain} % default
\newtheorem{Teorema}{Teorema}
\newtheorem{Ejemplo}{Ejemplo}
\theoremstyle{definition}
\newtheorem{definicion}{Definici\'on}
\newtheorem{Corolario}{Corolario}
\newtheorem{Proposicion}{Proposici\'on}
\newtheorem{Prueba}{Prueba}
\usepackage{esint}
\def\Q#1#2{\frac{\partial #1}{\partial #2}}
\usepackage{listings}
\usepackage{algorithm2e}
\usepackage{algorithmic}
%\usepackage{algpseudocode}
\usepackage{float}
\usepackage{listings}
\usepackage{empheq}
\usepackage{multimedia}
%\usepackage[x11names,rgb]{xcolor}
\lstset{%
    language=[AlLaTeX]TEX,%
    float=hbp,%
    basicstyle=\ttfamily\small, %
    identifierstyle=\color{colIdentifier}, %
    keywordstyle=\color{colKeys}, %
    stringstyle=\color{colString}, %
    commentstyle=\color{colComments}, %
    columns=flexible, %
    tabsize=3, %
    frame=single, %
    extendedchars=true, %
    showspaces=false, %
    showstringspaces=false, %
    numbers=left, %
    numberstyle=\tiny, %
    breaklines=true, %
    backgroundcolor=\color{hellgelb}, %
    breakautoindent=true, %
    captionpos=b,%
    xleftmargin=18pt,%
    xrightmargin=\fboxsep%
}
%---------------------Recuadros empheq
\definecolor{myblue}{rgb}{.8, .8, 1}
\definecolor{shadecolor}{cmyk}{0,0,0.41,0}
\newcommand*\mybluebox[1]{%
	\colorbox{myblue}{\hspace{1em}#1\hspace{1em}}
}
\newcommand*\myyellowbox[1]{%
	\colorbox{darkyellow}{\hspace{1em}#1\hspace{1em}}
}
%--------------------------------------------------------------------------
\definecolor{shadecolor}{RGB}{214,223,216}
\definecolor{light-blue}{rgb}{1,0.9,0}
\newsavebox{\mysaveboxM} % M for math
\newsavebox{\mysaveboxT} % T for text
\newcommand*\Garybox[2][Example]{%
	\sbox{\mysaveboxM}{#2}%
		\sbox{\mysaveboxT}{\fcolorbox{black}{light-blue}{#1}}%
			\sbox{\mysaveboxM}{%
	\parbox[b][\ht\mysaveboxM+.5\ht\mysaveboxT+.5\dp\mysaveboxT][b]{%
		\wd\mysaveboxM}{#2}%
	}%
	\sbox{\mysaveboxM}{%
		\fcolorbox{black}{shadecolor}{%
		\makebox[\linewidth-10em]{\usebox{\mysaveboxM}}%
		}%
	}%
	\usebox{\mysaveboxM}%
	\makebox[0pt][r]{%
		\makebox[\wd\mysaveboxM][c]{%
			\raisebox{\ht\mysaveboxM-0.5\ht\mysaveboxT
			+0.5\dp\mysaveboxT-0.5\fboxrule}{\usebox{\mysaveboxT}}%
		}%
	}%
}
%----------------------------------------------------------------------------------------------
\definecolor{kugreen}{RGB}{50,93,61}
\definecolor{kugreenlys}{RGB}{132,158,139}
\definecolor{kugreenlyslys}{RGB}{173,190,177}
\definecolor{kugreenlyslyslys}{RGB}{214,223,216}
\definecolor{greenArea}{RGB}{124,252,124}
\definecolor{hellmagenta}{rgb}{1,0.75,0.9}
\definecolor{hellcyan}{rgb}{0.75,1,0.9}
\definecolor{hellgelb}{rgb}{1,1,0.8}
\definecolor{colKeys}{rgb}{0,0,1}
\definecolor{colIdentifier}{rgb}{0,0,0}
\definecolor{colComments}{rgb}{1,0,0}
\definecolor{colString}{rgb}{0,0.5,0}
\definecolor{darkyellow}{rgb}{1,0.9,0}
\setbeamercovered{transparent}
\mode<presentation>
{  
	\usetheme{PaloAlto}
	\usecolortheme[named=kugreen]{structure}
	\useinnertheme{progressbar}
	%\usefonttheme{default}
	\usefonttheme{serif}
	\setbeamercovered{transparent}
	\setbeamertemplate{blocks}[rounded][shadow=true]
	\setbeamertemplate{navigation symbols}[only frame symbol]
}
\setbeamertemplate{background}{
\parbox[c][\paperheight]{\paperwidth}
    {
    \vfill \hfill
    \begin{tikzpicture}
      \node[opacity=.1]
      {
       \includegraphics[width=.5\textwidth]{./IMAGENES/Logo/CimatLogo.png}
      };
    \end{tikzpicture}
      \vspace{.5cm} %\hspace{.5cm}
      }
  }
\logo{\includegraphics[width=1.0cm]{./IMAGENES/Logo/Logo.png}}
\title{Simulaci\'on del Movimiento Browniano}
\author{
        Sa\'ul D\'iaz Infante
}
\institute{CIMAT A.C.}
\date{\today}
\AtBeginSection[]
{
  \begin{frame}<beamer>{Simulaci\'on de MB}
    \tableofcontents[currentsection,currentsubsection]
  \end{frame}
}
\begin{document}
  \frame{\titlepage \vspace{-0.5cm}} 
	\section{Introducci\'{o}n}
		%%%%%%%%%%%%%%%%%%%%%%%%%%%%%%%%%%%%%%%%%%%%%%%%%%%%%%%%%%%%%%%%%%%%%%%%%%
\begin{frame}{Objetivos}
	 \uncover<2->{
	\begin{block}{General}
		Simular el Movimiento Browniano.
	\end{block}
	}
	\vspace{0.25cm}
	\uncover<3->{
	\begin{block}{Espec\'ificos}
		\begin{itemize}
			\uncover<3->{
    			\item Contruir el Movimiento Browniano a partir de una caminata Aleatoria.
            }
			\uncover<4->{
    			\item Identificar las propiedades de Movimiento Browniano.
            }
			\uncover<5->{
				\item Describir el Algoritmo a utilizar en la Simulaci\'on del Movimiento Browniano.
            }
   		\end{itemize}
    }
\end{block}
\end{frame}
%%%%%%%%%%%%%%%%%%%%%%%%%%%%%%%%%%%%%%%%%%%%%%%%%%%%%%%%%%%%%%%%%%%%%%%%%%%%%%%%%
\begin{frame}
	\frametitle{Plan de Charla}
    \tableofcontents[pausesections]
\end{frame}
%%%%%%%%%%%%%%%%%%%%%%%%%%%%%%%%%%%%%%%%%%%
	\begin{frame}{Historia}
		\begin{center}
			\movie[width=9.1cm,height=5.2cm,showcontrols=true,loop,poster]{}{BrownianMotion.flv}
			\end{center}							
		\end{frame}
%%%%%%%%%%%%%%%%%%%%%%%%%%%%%%%%%%%%%%%%%%%

%%%%%%%%%%%%%%%%%%%%%%%%%%%%%%%%%%%%%%%%%%
\begin{frame}{Caminata Aleatoria}
	\begin{overlayarea}{\textwidth}{.7\textheight}
	\only<1>{	
	\begin{center}
		\includegraphics[width=\textwidth]{./IMAGENES/RW/RandomWalk(1).png}	
	\end{center}
	}
	\only<2>{	
		\begin{center}
			\includegraphics[width=\textwidth]{./IMAGENES/RW/RandomWalk[2].png}	
		\end{center}
	}
	\only<3>{
		\begin{center}
			\includegraphics[width=\textwidth]{./IMAGENES/RW/RandomWalk[3].png}	
		\end{center}	
	}	
	\end{overlayarea}
\end{frame}

%%%%%%%%%%%%%%%%%%%%%%%%%%%%%%%%%%%%%%%%%%%
\begin{frame}{Construcci\'on}
	\uncover<2->{
		Consideremos
			$$\{ X_{n} \}_{n=1}^{\infty} \quad \quad \mbox{v.a.i.id},$$
		tales que
			$$P(X_{j} = h) = P(X_{j} = -h) = \frac{1}{2}.$$
	}
	\uncover<3->{
		Sea $Y_{\delta, h}(0) = 0$, y hagamos
		$$ Y_{ \delta,h}(n \delta ) = X_{1}+X_{2}+\cdots+X_{n}.$$
	}
	\uncover<4->{
		Para $t > 0$, definamos $Y_{\delta,h}(t)$ con $n\delta < t < (n + 1)\delta$ como
		$$Y_{\delta,h}(t) = \frac{(n + 1)\delta -t}{\delta} Y_{\delta,h}(n \delta ) +  \frac{t - n\delta}
		{\delta}Y_{\delta,h}((n+1) \delta ).$$
	}
	\uncover<5->{
	\begin{block}{Interrogante}
		?`Cu\'al es el limite de $Y_{\delta, h}$ cuando $\delta,h \longrightarrow 0 ?$
	\end{block}
}
\end{frame}
%%%%%%%%%%%%%%%%%%%%%%%%%%%%%%%%%%%%%%%%%%%%%
\begin{frame}{Construcci\'on}
	\begin{overlayarea}{\textwidth}{.6\textheight}
	\only<+->{
		Para $\lambda \in \mathbb{R}$ fijo, calculemos $\displaystyle\lim_{\delta,h \to 0}
		{\mathbb{E}\left[e^{ i\lambda Y_{\delta,h}(t)}\right]}$. \\
		\hypertarget{cns:Limite}{}
		\hyperlink{dfn:FuncionCaracteristica}{\beamergotobutton{caracteristica}}
		\uncover<+->{
		Tomando $t=n\delta$, se sigue que
		\begin{eqnarray*}
			\mathbb{E}\left[e^{ i\lambda 					
			Y_{\delta,h}(t)}\right]&=&
			\prod_{j=1}^{n}\mathbb{E}\left[e^{i\lambda X_{j}}\right] \\
				&=&\left( \mathbb{E}\left[e^{i\lambda X_{j}}\right]\right)^{n}\\
				&=&\left(\frac{1}{2}e^{i\lambda h}+\frac{1}{2}e^{-i\lambda h}\right)^{n}\\
				&=&\left(cos(\lambda h)\right)^{n}\\
				&=&\left(cos(\lambda h)\right)^{\frac{t}{\delta}}.
		\end{eqnarray*}
		}
	}
\end{overlayarea}
\end{frame}
%%%%%%%%%%%%%%%%%%%%%%%%%%%%%%%%%%%%%%%%%%%
\begin{frame}{Construcci\'on}
	Sea $$ u =\left(cos(\lambda h)\right)^{\frac{1}{\delta}},$$ as\'i
	$$ u \approx e^{ - \frac{1}{2\delta}\lambda^{2}h^{2}}.$$
	\uncover<2->{
	Luego
	$$
		\mathbb{E}\left[ e^{i\lambda Y_{\delta,h}(t)}\right]\approx e^{- \frac{1}
		{2\delta}t\lambda^{2}h^{2}}.
	$$

	}
	\uncover<3->{
	Si $h^{2} = \delta$, entonces
	$$
		\displaystyle\lim_{\delta,h \to 0}{\mathbb{E}\left[ e^{i\lambda Y_{\delta,h}(t)}\right]} 
		= e^{ - \frac{1}{2}t\lambda^{2}},
		\quad \quad \lambda \in \mathbb{R}.$$
}
\end{frame}
%%%%%%%%%%%%%%%%%%%%%%%%%%%%%%%%%%%%%%%%%%%%%%%%%%%%%%%%%%%%%%%%%%%%%%%%%%%%%%%%%%%
\begin{frame}{Construcci\'on}
	\begin{Teorema} 
		Sea $Y_{\delta,h}(t)$ una caminata aleatoria que inicia en $0$ 
		de saltos $h$ y $-h$  con igual probabilidad en los tiempos 
		$\delta, 2\delta,3\delta, \ldots $.  
		Asumamos que $h^{2} = \delta$. 
		Entonces para cada $t \geq 0$, el limite
		$$ B(t) = \displaystyle\lim_{\delta \to 0}{Y_{\delta,h}(t)},$$
		existe en distribuci\'on. Adem\'as, tenemos que
		$$
			\mathbb{E}\left[e^{i\lambda B(t)}\right] 
			=e^{- \frac{1}{2}t\lambda^{2}}, \quad \quad \lambda \in 	\mathbb{R}.
		$$
	\end{Teorema}
\end{frame}
%%%%%%%%%%%%%%%%%%%%%%%%%%%%%%%%%%%%%%%%%%%%%%%%%%%%%%%%%%%%%%%%%%%%%%%%%%%%%%%%%%%%
	\section{Algoritmo RW} 
	  	%%%%%%%%%%%%%%%%%%%%%%%%%%%%%%%%%%%%%%%%%%%
\begin{frame}{Codigo}
	\only<1>{
		\begin{figure}
			\centering
			\tiny\lstset{language=python}
			\lstinputlisting[firstline=1,lastline=13]{RW01.py}
		\end{figure}
		}
	\only<2>{
		\begin{figure}
			\centering
			\tiny\lstset{language=python}
			\lstinputlisting[firstline=14,lastline=21]{RW01.py}
		\end{figure}
	}
\end{frame}
%%%%%%%%%%%%%%%%%%%%%%%%%%%%%%%%%%%%%%%%%%%
\begin{frame}{Caminata Aleatoria de $n$  transiciones}
	\begin{overlayarea}{\textwidth}{.7\textheight}
	\only<1>{	
		\begin{center}
			\includegraphics[width=\textwidth]{./IMAGENES/RW/RW01(1).png}	
		\end{center}
	}
	\only<2>{
		\begin{center}
			\includegraphics[width=\textwidth]{./IMAGENES/RW/RW01(2).png}	
		\end{center}	
	}
	\only<3>{
		\begin{center}
			\includegraphics[width=\textwidth]{./IMAGENES/RW/RW01(3).png}	
		\end{center}	
	}	
	\end{overlayarea}
\end{frame}
%%%%%%%%%%%%%%%%%%%%%%%%%%%%%%%%%%%%%%%%%%%%%%%%%%%%%%%%%%%%%%%%%%%%%%%%%
\begin{frame}{Construcci\'on}
	\begin{empheq}[box={\Garybox[Construcci\'on]}]{align*}
		h^{2}&=\delta\\	
		Y_{\delta,h}(t) 
		& \xrightarrow[\delta ,h \to 0]{\mathcal{D}} B(t) \qquad \forall t \geq 0\\
		\mathbb{E}\left[e^{i\lambda B(t)}\right] 
		&
		\xrightarrow{\delta ,h \to 0}
		e^{-\frac{1}{2}t\lambda^{2}}, \quad \lambda \in 	\mathbb{R}.	
	\end{empheq}	
\end{frame}
%%%%%%%%%%%%%%%%%%%%%%%%%%%%%%%%%%%%%%%%%%%%%%%%%%%%%%%%%%%%%%%%%%%%%%%%%%%%%
\begin{frame}{Distribuci\'on Gaussiana}
	\begin{figure}	
		\begin{tikzpicture}[scale=1.0]
    		\colorlet{col1}{blue!70}
		\colorlet{col2}{blue!60}
		\colorlet{col3}{blue!50}
		\colorlet{col4}{blue!40}
   		%\draw [help lines] (-4.25,-1.25) grid (4.25,1.5);
		%\draw [help lines,step=0.25cm] (-2.99,0) grid (2.99,0.99);
		\draw[->] (0,-1.25) -- (0,1.5) node [above]
		{$\displaystyle
			\frac{1}{\sigma\sqrt{2\pi}}\exp\biggl(\frac{-x^2}{2\sigma^2}\biggr)
		$};
		\begin{scope}[smooth,draw=gray!20,y=0.3989422804cm]
        \filldraw [fill=col3] plot[id=f1,domain=-3:-2] function {exp(-x*x/2)}
            -- (-2,0) -- (-3,0) -- cycle;
        \filldraw [fill=col2] plot[id=f2,domain=-2:-1] function {exp(-x*x/2)}
            -- (-1,0) -- (-2,0) -- cycle;
        \filldraw [fill=col1] plot[id=f3,domain=-1:0]  function {exp(-x*x/2)}
            -- (0,0)  -- (-1,0) -- cycle;
        \filldraw [fill=col1] plot[id=f4,domain=0:1] function {exp(-x*x/2)}
            -- (1,0)  --  (0,0) -- cycle;
        \filldraw [fill=col2] plot[id=f5,domain=1:2] function {exp(-x*x/2)}
            -- (2,0)  -- (1,0) -- cycle;
        \filldraw [fill=col3] plot[id=f6,domain=2:3] function {exp(-x*x/2)}
            -- (3,0)  -- (2,0) -- cycle;
        \draw[black] plot[id=f7,domain=-4.25:4.25,samples=100]
            function {exp(-x*x/2)};
		\end{scope}
       \draw[->] (-4.25,0) -- (4.25,0) node [right] {$x$};
    \foreach \pos/\label in {-3/$-3\sigma$,-2/$-2\sigma$,-1/$-\sigma$,
            1/$\sigma$,2/$2\sigma$,3/$3\sigma$}
        \draw (\pos,0) -- (\pos,-0.1) (\pos cm,-3ex) node
            [anchor=base,fill=white,inner sep=1pt]  {\label};

    \draw (-0.1,1) -- (.1,1) node [right,fill=white,inner sep=1pt] {$\sigma$};

    \foreach \pos/\percent/\height in {1/34/0.5,2/14/0.25,3/2/0.125,4/0.1/0.1}
    {
      \node[text=col\pos,anchor=base,yshift=2pt,xshift=-0.625cm,
        fill=white,inner sep=1pt] at (\pos,\height) {$\percent\%$};
      \node[text=col\pos,anchor=base,yshift=2pt,xshift=.625cm,
        fill=white,inner sep=1pt]  at (-\pos,\height) {$\percent\%$};
    }
	\end{tikzpicture}
	\end{figure}
\end{frame}
%%%%%%%%%%%%%%%%%%%%%%%%%%%%%%%%%%%%%%%%%%%%%%%%%%%%%%%%%%%%%%%%%%%%%%%%%%
\begin{frame}{Caminata Aleatoria en $[0,1]$}
	\vspace{-1.5cm}		
	\begin{overlayarea}{\textwidth}{.7\textheight}
	\only<1>{	
		\begin{center}
			\includegraphics[width=\textwidth]{./IMAGENES/RW/RWs01.png}	
		\end{center}
	}
	\only<2>{
		\begin{center}
			\includegraphics[width=\textwidth]{./IMAGENES/RW/RWs01Sigma.png}	
		\end{center}	
	}
	\end{overlayarea}
\end{frame}



   	
  	\section{Porpiedades de M.B}
		\begin{frame}
	\frametitle{Propiedades que se esperan de $B(t)$}
	Con base en la anterior discusi\'on, vemos que:
	\begin{alertblock}{}
		\begin{equation*}
			\begin{array}{rcl}
				|\frac{dY_{\delta,h}(t)}{dt}|&=&|\frac{1}{\delta}X_{n+1}|\\
         		&&\\
         		&=&|\frac{1}{\delta}|\cdot |X_{n+1}|\\
         		&&\\
         		&=&\frac{h}{\delta}%
		\end{array}
	\end{equation*}
	\uncover<2->{
		\begin{equation*}
			\lim_{\delta\to 0}\frac{h}{\delta}=\lim_{\delta\rightarrow 0}\frac{1}{\sqrt {\delta}}=\infty.
		\end{equation*}
	}
	\end{alertblock}
	\uncover<2->{
		\begin{exampleblock}{}
			Teniendo en cuenta lo anterior, es de esperarse que $B(t)$ no sea diferenciable.
		\end{exampleblock}
	}
\end{frame}
%--------------------------------------------------------------------------------------------------------
\begin{frame}
	\frametitle{Propiedades que se esperan de $B(t)$}
	\begin{exampleblock}{}
		\begin{itemize}
			\item Para todo $t$, $B(t)$  es una v.a $N(0,t)$.
		\end{itemize}
	\end{exampleblock}
	\uncover<2->{
		\begin{exampleblock}{}
			\begin{itemize}
				\item Tomando $\delta=|t-s|$
				\begin{equation*}
					|B(t)-B(s)|\approx \frac{1}{\sqrt {\delta}}|t-s|=|t-s|^{\frac{1}{2}}.
				\end{equation*}
			\end{itemize}
		\end{exampleblock}
	}
	\uncover<3->{
		\begin{exampleblock}{}
			\begin{itemize}
				\item $B(t)-B(s)\sim N(0,t-s)$
			\end{itemize}
		\end{exampleblock}
	}
	\uncover<4->{
		\begin{exampleblock}{}
			\begin{itemize}
				\item $B(t)$ tiene incrementos independientes.
			\end{itemize}
		\end{exampleblock}
	}
\end{frame}
%--------------------------------------------------------------------------------------------------------------
\begin{frame}
	\frametitle{Movimiento Browniano (M.B)}
	\begin{block}{Definici\'on}
		Un proceso estoc\'astico $B(t,\omega)$ es llamado un movimiento Browniano si satisface las 			siguientes condiciones:
		\begin{enumerate}
			\item $P\{\omega;B(0,\omega)=0\}=1$.
			\item Para todo $0\leq s < t$, la v.a $B(t) - B(s)\sim N(0,t -s)$.
			\item $B(t,\omega)$ tiene incrementos independientes, es decir, para 
				$0\leq t_{1} < t_{2}	<\ldots < t_{n}$,
				\begin{equation*}
					B(t_{1}), B(t_{2}) - B(t_{1}),\ldots,B(t_{n}) - B(t_{n-1}),
				\end{equation*}
				son independientes.
		\end{enumerate}
	\end{block}
\end{frame}
%%%%%%%%%%%%%%%%%%%%%%%%%%%%%%%%%%%%%%%%%%%%%%%%%%%%%%%%%%%%%%%%%%%%%%%%%%%%%%%%%%%%%%%%%
\begin{frame}
	\frametitle{Propiedades b\'asicas del M.B.}
	\begin{exampleblock}{}
		\begin{itemize}
			\item Para todo $s,t\geq 0$, $E[B(s)B(t)]=\min\{s,t\}$.
		\end{itemize}
	\end{exampleblock}
	\uncover<2->{
		\begin{exampleblock}{}
			\begin{itemize}
				\item Para 
					$t_{0}\geq 0$, el proceso estoc\'astico 
					$\tilde{B}(t)=B(t+t_{0}) - B(t_{0})$ es tambi\'en un M.B.
			\end{itemize}
		\end{exampleblock}
	}
	\uncover<3->{
		\begin{exampleblock}{}
			\begin{itemize}
				\item 
					Para cualquier n\'umero real $\lambda > 0$, el proceso estoc\'astico 
					$\tilde{B}(t)=\frac {B(\lambda t)}{\sqrt{\lambda}}$ es tambi\'en un M.B.
			\end{itemize}
		\end{exampleblock}
	}
\end{frame}
%%%%%%%%%%%%%%%%%%%%%%%%%%%%%%%%%%%%%%%%%%%%%%%%%%%%%%%%%%%%%%%%%%%%%%%%%%%%
\begin{frame}
	\frametitle{Propiedades de las trayectorias del M.B.}
	\begin{alertblock}{Continuidad}
		Las trayectorias del movimiento Browniano son continuas casi seguramente.\\
		(Ver Teorema de continuidad de Kolmogorov, Hui - Hsiung Kuo, p\'ag. 31)
	\end{alertblock}
	\uncover<2->{
		\begin{alertblock}{Variaci\'on}
			La variaci\'on de una trayectoria de un M.B sobre el intervalo $[a,b]$ es infinita casi
			seguramente, es decir,
			\begin{equation*}
				\limsup_{\Delta t\rightarrow 0} 
				\sum_{i=1}^{n-1} |B(t_{i+1}) - B(t_{i})|=\infty,\quad c.s.
			\end{equation*}
			(Ver Proposici\'on de variaci\'on no acotada de un M.B, Mikosch, p\'ag. 189)
		\end{alertblock}
	}
\end{frame}
%%%%%%%%%%%%%%%%%%%%%%%%%%%%%%%%%%%%%%%%%%%%%%%%%%%%%%%%%%%%%%%%%%%%%%%%%%%%%
\begin{frame}
	\frametitle{Propiedades de las trayectorias del M.B.}
	\only<+>{
		\begin{alertblock}{Variaci\'on cuadr\'atica}
			La variaci\'on cuadr\'atica de una trayectoria de un M.B sobre el intervalo $[a,b]$ es la 			longitud del intervalo, es decir,
			\begin{equation*}
				\limsup_{\Delta t\rightarrow 0} 
				\sum_{i=1}^{n-1} |B(t_{i+1}) - B(t_{i})|^{2}=b - a,
		\end{equation*}
		(Ver notas del curso Introducci\'on a los Procesos estoc\'asticos, Luis Rincon, p\'ag. 212)
		\end{alertblock}
	}
\end{frame}
%%%%%%%%%%%%%%%%%%%%%%%%%%%%%%%%%%%%%%%%%%%%%%%%%%%%%%%%%%%%%%%%%%%%%
	
	 \section{Algoritmo MB}
		%%%%%%%%%%%%%%%%%%%%%%%%%%%%%%%%%%%%%%%%%%%%%%%%%%%%%%%%%%%%%
\begin{frame}[fragile]
\begin{algorithm}[H]
	\begin{algorithmic}[1]
		\STATE  Particionar $[0,T]$  %$\displaystyle dt =\frac{1}{N}$
		\FOR{$j=0$ to $N$} 
			\STATE $t_j=jdt$
		\ENDFOR
		\STATE Sea $B(t_j)=B_j$					
		\FOR {$j=1$ to $N$}
			\STATE $B_j=B_{j-1}+dB_j $ \qquad $dB_j\sim \sqrt{dt}N(01)$  		
		\ENDFOR
	\end{algorithmic}
\end{algorithm}
\end{frame}
%%%%%%%%%%%%%%%%%%%%%%%%%%%%%%%%%%%%%%%%%%%%%%%
\begin{frame}{C\'odigo}
 \only<+>{
  \begin{figure}
  \centering
      \tiny
   \lstset{language=python}
         \lstinputlisting[firstline=1,lastline=13]{BrownianPaths.py}
   \end{figure}
	}
\end{frame}
%%%%%%%%%%%%%%%%%%%%%%%%%%%%%%%%%%%%%%%%%%%%%%%
\begin{frame}{Trayectorias de MB}
\vspace{1.0cm}
\begin{overlayarea}{\textwidth}{\textheight}	
	\only<+>{	
		\begin{center}
			\includegraphics[width=\textwidth]{./IMAGENES/RW/MB(1).png}
		\end{center}
	}
	\only<+>{	
		\begin{center}
			\includegraphics[width=\textwidth]{./IMAGENES/RW/MB(2).png}
		\end{center}
	}
	\only<+>{	
		\begin{center}
			\includegraphics[width=\textwidth]{./IMAGENES/RW/MB(3).png}
		\end{center}
	}
	\end{overlayarea}
\end{frame}
%%%%%%%%%%%%%%%%%%%%%%%%%%%%%%%%%%%%%%%%%%%%%%%%
\begin{frame}
	\frametitle{Escalamiento}
		\vspace{-1.5cm}
	\uncover<+->{		
		\begin{empheq}[box={\Garybox[Recordemos]}]{equation*}
			B(t)-B(s) \sim \sqrt{t-s}N(0,1) \text{ son i.i.d }				
		\end{empheq}
	}				
	\uncover<+->{	
	\begin{block}{Podemos ver $B_N$ como suma de v.a.i.i.d}
		\begin{overlayarea}{\textwidth}{.45\textheight}
						
			\only<+>{
				Dados $t\in [0,T]$ y cualquier partición 
				$$
					0=t_0<t_1<\dots< t_{M-1}<t_M=t.
				$$
			}
			\only<+>{				
				Si usamos una suma telescópica tenemos
				$$
					B(t)	=\sum_{j=0}^m B(t_j)-B(t_{j-1}) \qquad (B(t_0=0)=0)
				$$	
			}
			
			\only<+->{			
				Con esto podemos  escalar los incrementos del MB a partir de 
				$$
					B_N =\sum_{j=0}^N dB_j	\qquad dB_j\sim \sqrt{dt} N(0,1) 
				$$
			}
			\only<+->{			
				Veamos una trayectoria y el código python.			
			}
		\end{overlayarea}
	\end{block}	
	}
\end{frame}
%%%%%%%%%%%%%%%%%%%%%%%%%%%%%%%%%%%%%%%
\begin{frame}{C\'odigo}
 \only<+>{
  \begin{figure}
  \centering
      \tiny
   \lstset{language=python}
         \lstinputlisting[firstline=14,lastline=29]{BrownianPaths.py}
   \end{figure}
	}
	\only<+>{
		 \begin{figure}
  \centering
      \tiny
   \lstset{language=python}
         \lstinputlisting[firstline=30,lastline=37]{BrownianPaths.py}
   \end{figure}
	
	}
\end{frame}
%%%%%%%%%%%%%%%%%%%%%%%%%%%%%%%%%%%%%%%%%%%%%%%%%%%%%%%%%%%%%%%%%%%%%%%%%%%%%%%
\begin{frame}
	\frametitle{MB con incrementos escalados}
	\vspace{1.5cm}	
	\framezoom<0><2>[border](0cm,4.5cm)(1.5cm,1.5cm)	
	\begin{overlayarea}{\textwidth}{\textheight}	
	%\only<+>{	
		\begin{center}
			\includegraphics[width=\textwidth]{./IMAGENES/RW/MBEscalado(1).png}
		\end{center}
	%}
	\end{overlayarea}
\end{frame}
%%%%%%%%%%%%%%%%%%%%%%%%%%%%%%%%%%%%%%%%%%%%%%%%%%%%%%%%%%%%%%%%%%%%%%%%%%%%%%%%%
\begin{frame}
	\frametitle{Distribuci\'on MB}
		\vspace{-1.7cm}		
		\begin{overlayarea}{\textwidth}{.68\textheight}	
			\only<+>{	
				\begin{center}
					\includegraphics[width=.95\textwidth]{./IMAGENES/RW/MBs.png}
				\end{center}
			}
			\only<+>{	
				\begin{center}
					\includegraphics[width=.95\textwidth]{./IMAGENES/RW/MBSigma.png}
				\end{center}
			}
	\end{overlayarea}
\end{frame}
  
	 %\section{Integracion estoc\'astica}
	%\begin{frame}{Puente Browniano}
	%\end{frame}
	%\begin{frame}
%	  \frametitle{Movimiento Browniano Geométrico}
%	\end{frame}
	\begin{frame}[allowframebreaks]{Referencias}	
		\nocite{*}		
		\bibliographystyle{apalike}
		\bibliography{BibMODESTO}
  \end{frame}
  \appendix
	\part{Definiciones y resultados}
	%%%%%%%%%%%%%%%%%%%%%%%%%%%%%%%%%%%%%%
\begin{frame}{}
\hypertarget{dfn:FuncionCaracteristica}{}
\begin{block}{Definici\'on (Funci\'on caracter\'istica)}La funci\'on caracter\'istica de
	la variable aleatoria $X$ es la funci\'on
	$$\phi_{X}(t)= \mathbb{E}\left( e^{itX}\right),$$
	definida para cualquier n\'umero real $t$. El n\'umero $i$ es la unidad de los n\'umeros 	imaginarios.
\end{block}
\uncover<2>{
\begin{block}{Teorema de continuidad}
Sean $X,X_{1},X_{2}, \cdots $ variables aleatorias. Entonces
$$ X_{n} \xrightarrow{\mathcal{D}} X \Leftrightarrow \phi_{X_{n}}(t) \longrightarrow \phi_{X}(t) .$$
\end{block}
}
\hyperlink{cns:Limite}{\beamerreturnbutton{Const}}
\end{frame}
%%%%%%%%%%%%%%%%%%%%%%%%%%%%%%%%%%%%%%%%

\end{document}
