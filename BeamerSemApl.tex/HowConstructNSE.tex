\begin{bibunit}[aalpha]
   \begin{frame}%[label= thm:Alternativa1]
          \frametitle{Como construir MDFNE alternativa 1 }
    \begin{columns}
      \column{.5\textwidth}
      \only<1-3>{
      \begin{beamerboxesrounded}
       [upper=color titulo caja, lower=color cuerpo caja, shadow=true]
      {Dada}
       $\frac{dy}{dt}=f(y)$
      \begin{equation*}
            \{\bar{y}^{(i)}; i=1,2,\dots, I\},
          \end{equation*}
          Sean    $R_i $ y $ R^*$ tales que
          \begin{align*}
            R_i:= & \frac{df(\bar{y}^{(i)})}{dy}\\
            R^*:=&\max\{|R_i|, i=1,\dots, I \}
          \end{align*}
       \end{beamerboxesrounded}
      }
      \only<4>{
      \begin{Teorema}
        El esquema en diferencias  tiene puntos fijos con exactamente \emph{la misma estabilidad
        lineal} que $\textcolor{red}{\frac{dy}{dt}=f(y)}$ para todo $h>0$
      \end{Teorema}
      }
      \column{.5\textwidth}
      %Del an\'alisis de estabilidad lineal aplicado al $i$-\'esimo punto fijo se tienen lo
      %siguiente 
        \only<2>{
        \cite{mickens1994nonstandard}:
        \begin{enumerate}[(i)]
          \item
            Si $R_i>0$, el punto fijo $y(t)=\bar{y}^{(i)}$ es linealmente estable.
          \item
            Si $R_i<0$, el punto fijo $y(t)=\bar{y}^{(i)}$ es linealmente inestable.
        \end{enumerate}
       }
      \uncover<3-4>{
      \begin{exampleblock}{Consideremos}
        $$
           \frac{y_{k+1}-y_{k}}{
          \left(
            \frac{\phi(hR^*)}{R^*}
          \right)
          }=f(y_k),
        $$
        \begin{align*}
            \phi(z)=& z+\mathcal{O}(z^2), \quad z\to 0,\\%\label{eqn:ConditionPhia} \\
          0<& \phi(z)<1, \quad z>0% \label{eqn:ConditionPhib}.
        \end{align*}
      \end{exampleblock}
      }
    \end{columns}
    \biblio{BibliografiaTesis}
  \end{frame}
\end{bibunit}
%%%%%%%%%%%%%%%%%%%%%%%%%%%%%%%%%%%%%%%%%%%%%%%%%%%%%%%
\begin{bibunit}[aalpha]
  \begin{frame}
    \frametitle{Promedio de Steklov (alternativa 2)}
    \only<1>{
        \begin{block}{Usando el promedio especial de Steklov para $f$}
          \cite{matus2005exact}
          \begin{align*}
          f(y(t))\approx&
            \left(
            \frac{1}{y_{k+1}-y_{k}}
            \int_{y_k}^{y_{k+1}} \frac{du}{f(u)}
            \right)^{-1}\\
          t_k\leq & t \leq t_{k+1},\\
          y_k=&y(t_k), t_k=kh.
          \end{align*}
        \end{block}
      }
    \begin{columns}
      \column{.5\textwidth}
      \only<2-3>{
        \begin{block}{ El esquema toma la forma}
          \begin{align*}
          \frac{y_{k+1}-y_k}{h}=&\varphi(y_k,t_k),\\
          \varphi=&\varphi_1(y_k)\varphi(t_k),\\
          k=&0,1,\dots \notag\\
          y(0)=&y_0, \notag
          \end{align*}
        \end{block}
      }
      \column{.55\textwidth}
      \only<3>{
      \begin{alertblock}{Donde}
       \begin{align*}
        \varphi_1(y_k)=&\left(\frac{1}{y_{k+1}-y_{k}}
          \int_{y_k}^{y_{k+1}} \frac{du}{f_1(u)}
        \right)^{-1}, \\
        \varphi_2(t_k)=&\frac{1}{h}\int_{t_k}^{t_ {k+1}}f_2(t)dt .
      \end{align*}
      \end{alertblock}
    }
    \end{columns}
  \biblio{BibliografiaTesis}
  \end{frame}
\end{bibunit}