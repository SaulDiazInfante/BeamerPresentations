\section{Motivaci\'on y antecedentes}
 
 
  Muchos productos que usamos de manera cotidiana quedan englobados dentro de la categor\'ia de coloide. 
  Por mencionar algunos ejemplos están: pinturas, cosm\'eticos,
  medicamentos, productos alimenticios entre otros. 
  
  Por esto en la actualidad los coloides son un sistema de gran inter\'es industrial lo cual se traduce 
   en que las empresas enfocan parte de sus esfuerzos a la investigaci\'on de sus propiedades din\'amicas
  y estructurales con el fin de mejorar la calidad en sus productos. 

  Por otro lado, en los \'ultimos a\~nos los sistemas coloidales se han convertido en uno 
  de los sistemas modelo m\'as recurridos por acad\'emicos de ciencias tales como la f\'isica,
  qu\'imica, biolog\'ia y biof\'isica, solo por citar solo algunas.

  Quiz\'a esta combinaci\'on de inter\'es pr\'actico y cient\'ifico ha
  colocado a los coloides como uno de los sistemas m\'as ampliamente estudiados
  mediante experimento, teor\'ia y simulaci\'on computacional.
  
  El movimiento Browiniano caracteriza a las suspensiones coloidales. Fue el mismo 
  Robert Brown quien en 1827 al observar este movimiento azaroso en partículas suspendidas de 
  polen llego a creer que estaban vivas. M\'as tarde en 1896 L. Boltzmann encontr\'o las causas de 
  este movimiento cuando escribi\'o \textquotedblleft partículas muy pequeñas en un gas ejecutan 
  movimientos que se derivan del hecho de que la presión en la superficie de las partículas pueden 
  fluctuar\textquotedblright.

  Sin embargo fue A. Einstein quien en 1905 derivo una descripci\'on probabil\'istica 
  valida para un ensamble de part\'iculas  Brownianas. En su trabajo no se enfoc\'o 
  plenamente a las descripci\'on de las complicadas trayectorias de cada part\'icula; 
  en su lugar introduce dos conceptos: el de dinámica de grano grueso; a grandes rasgos
  considera una escala de tiempo $\tau$ para la cual dos eventos con diferencia de tiempo
  de este orden se consideran independientes. Y el de densidad de probabilidad a tiempos 
  cortos para el desplazamiento \cite{ToraMigu97:Stocheffecphysysyste}.
  
  En 1908 Langevin  inicia un estudio de estos sistemas en el cual de alguna manera se considera como 
  complementario al trabajo de Einstein. En su tratamiento el se preocupa m\'as por la trayectoria de una 
  part\'icula Browniana, con esto en mente escribe las ecuaciones de Newton de tipo 
  fuerza=masa $\times$  aceleraci\'on, considerando un termino azaroso, y las fuerzas de fricci\'on debidas 
  al solvente (\ref{eqn:Langevin}), Langevin formul\'o las siguientes ecuaciones:
  
  \begin{align}\label{eqn:Langevin}
  	\mathbf{\dot{r}}&=\mathbf{v} \\
	\mathbf{m\dot{v}}&=\mathbf{F}-\gamma \mathbf{v} + \mathbf{\xi}(t)
  \end{align}
  
  Donde $\mathbf{r}$ denota posici\'on, $\mathbf{v}$ velocidad, $\gamma$ es un coeficiente de fricci\'on,
  $\mathbf{F}$ representa la fuerza debida al movimiento de la part\'icula Browniana y $\xi$ la fuerza 
  debida a la interacci\'on con las dem\'as part\'iculas (estoc\'astica).
  
  El trabajo de laboratorio dice que este fen\'omeno es disipativo y puede en primera 
  aproximaci\'on expresarse mediante una fuerza proporcional a  
  $$
  -\gamma \mathbf{v}  \qquad 
	%\text{donde $\gamma$ es una constante de fricci\'on  y $\mathbf{v}$ es la velocidad del coloide}
  $$
  En un an\'alisis m\'as detallado aparecer\'ian la ley de Stokes para el coeficiente de 
  fricci\'on  $\gamma = 6\pi a\eta$ siendo $\eta$ la viscosidad del fluido y $a$ el radio 
  de la supuesta part\'icula coloide esf\'erica \cite{toral2002Ruimasruiporfav}.
  
  Las part\'iculas de sistemas coloidales al estar inmersas en un fluido siguen 
  una trayectoria altamente irregular. Una descripci\'on completa de esta din\'amica 
  requerir\'ia conocer con exactitud el efecto que sobre ella ejercen las innumerables
  mol\'eculas del fluido. Esto a su vez necesita el conocimiento de posici\'on y 
  velocidad para cada part\'icula.
  
  Si se considera el gran n\'umero de part\'iculas en un sistema real, es claro 
  que la tarea de una descripci\'on a nivel microsc\'opico seria humanamente imposible.
  Sin embargo en una descripción puramente macrosc\'opica de tipo fuerza $=$ masa$\times$
  aceleraci\'on, podr\'iamos considerar el efecto global de dichas mol\'eculas dando 
  un promedio de las fuerzas que ejerce el fluido sobre un coloide. 
  
  Esta descripci\'on macrosc\'opica resulta ser incompleta pues predice que una
  partícula tendrá al reposo lo que contradice a experimentos. Un paso intermedio
  entre la imposibilidad de considerar el movimiento detallado de las mol\'eculas de 
  fluido y la descripci\'on mediante una sola fuerza viscosa consiste en incluir en 
  las ecuaciones del movimiento el t\'ermino de fuerza adicional $\xi(t)$, de tipo aleatorio. 
 
  Otro modelo que surge de manera an\'aloga a la formulaci\'on de Einstein es el de Fokker-Plank
  \begin{equation}\label{eqn:Fokker-Planck}
	\frac{\partial f }{\partial t}=-\frac{\partial}{\partial x} 
	\left[
	  q(x)f
	\right]
	+\frac{1}{2}\frac{\partial}{\partial x}
	\left[
	  g(x) \frac{\partial}{\partial x}
	  \left[
		g(x)f
	  \right]
	\right]
  \end{equation}
  
  Esta formulaci\'on se centra m\'as en encontrar una funci\'on de densidad de probabilidad para
  los incrementos en el movimiento. 
  Resulta que bajo ciertas consideraci\'ones las formulaciones de Langevin y Fokker - Planck
  son equivalentes.



  Los modelos mencionados son de ecuaci\'on diferencial estoc\'astica, ordinaria y parcial respectivamente. 
  Sus soluciones,  a diferencia del caso determinista, son ahora procesos estoc\'asticos los cuales podemos
  entender como una familia de funciones de densidad de probabilidad  $f(x,t)$ 
  para cada tiempo $t \leq 0$ y variable aleatoria $x$.
  
  
  
  Para encontrar soluci\'on; un  algoritmo de uso com\'un es el de \emph{Ermack y McCammon}, 
  este algoritmo es lineal y simple de implementar pero, presenta limitaciones en tiempo de computo.
  Limitaciones que solo permiten la simulación de sistemas a escalas de tiempo relativamente
  cortas \cite{ErmakMcCammon}.
  
\section{Objetivo}
  En un intento por obtener mejores algoritmos de soluci\'on, existen trabajos que han implementado
  algoritmos del tipo predictor - corrector cono Runge Kutta  en su versi\'on estoc\'astica, pero no 
  obtienen una mejora significativa.
  Al igual que los algoritmos deterministas como el Runge; los resultado en otras \'areas  
  nos dice que los algoritmos en diferencias no est\'andar logran mejoras significativas \cite{Mickens}. 
  Por esto, en este trabajo pensamos que un algoritmo de este tipo  podr\'ia arrojar buenos resultados.

  Lo dicho en la secci\'on anterior nos da la pauta para proponernos como objetivo en esta tesis  
  el buscar una mejora en el tiempo de ejecuci\'on  para la simulaci\'on de sistemas en escala de tiempo mayor. 
  En concreto implementar un  algoritmo de diferencias finitas no estándar para simular la din\'amica de estos sistemas.
\pagebreak
\section{Metodolog\'ia}
  El presente trabajo esta pensado desarrollarse en dos etapas:
  \subsection*{Informaci\'on}
	\begin{itemize}
	  \item Estudio de conceptos de c\'alculo estoc\'astico.
	  \item Estudio de la expansi\'on en Serie de Taylor Estoc\'astica.
	  \item Estudio de la teor\'ia cualitativa de ecuaciones diferenciales estoc\'asticas (existencia y unicidad de soluciones).
	  \item Estudio de algoritmos en diferencias finitas no est\'andar.
	\end{itemize}
  \subsection*{Implementaci\'on del algoritmo}
	\begin{itemize}
	  \item Formulaci\'on de Algoritmos en diferencias finitas para ecuaciones diferenciales estoc\'asticas.
	  \item Estudio cualitativo de los algoritmos mencionados (convergencia, error y estabilidad). 
	  \item Implementaci\'on  de un algoritmo en diferencias finita no est\'andar para resolver la ecuaci\'on de Langevin.
	  \item Validaci\'on del algoritmo a través de un problema con soluci\'on analítica.
	\end{itemize}
\begin{frame}
  \frametitle{Implementaci\'on del algoritmo}
  \subsection{Implementaci\'on del algoritmo}
  \begin{block}

  \end{block}
\end{frame}
