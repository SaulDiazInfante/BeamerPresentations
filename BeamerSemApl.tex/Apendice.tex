\begin{frame}
    \frametitle{Movimiento Browniano}
  \begin{alertblock}{Propiedades básicas}
        Un movimiento Browniano unidimensional de parámetro $\sigma^2$ es
        un proceso estocástico ${B_t : t ≥ 0}$ con valores en $R$ que cumple las siguientes
        propiedades.
      \begin{itemize}
        \item $B_0=0$  c.s.
          Las trayectorias $ t \mapsto B_t$ son continuas.
        \item
          El proceso tiene incrementos independientes.
        \item
          La variable aleatoria $B_t-B_s$ tiene distribución normal $N(0,\sigma^2(t-s))$ para cualquier tiempo
          $0\leq s\leq t$
     \end{itemize}
    \end{alertblock}
\end{frame}
%%%%%%%%%%%%%%%%%%%%%%%%%%%%%%%%%
\begin{frame}
   \frametitle{Constante de Boltzmann}
    \begin{alertblock}{Relaciona temperatura absoluta y energía de un sistema.}
      $$k_b\approx \SI{1.3806504e-23}{\frac{\joule}{\kelvin}}$$
    \end{alertblock}
\end{frame}
%%%%%%%%%%%%%%%%%%%%%%%%%%%%%%%%%
\begin{frame}
   \frametitle{Ley de Stockes}
    \begin{exampleblock}{$a:$ radio, $\eta:$ viscosidad}
      La ley de Stokes se refiere a la fuerza de fricción experimentada por objetos esféricos moviéndose en el
      seno    de un fluido viscoso en un régimen laminar de bajos números de Reynolds.
      $F_c=6\pi a \eta v$
   \end{exampleblock}
\end{frame}
%%%%%%%%%%%%%%%%%%%%%%%%%%%%%%%%
\begin{frame}
   \frametitle{Respecto al potencial}
    \begin{block}{Interacción aditiva}
      $$
       \mathbf{F}_i= -\nabla_i U=-\nabla_i \sum_{i\neq j}^N V(\mathbf{r}_i-\mathbf{r}_j)
      $$
    \end{block}
    En la literatura revisada usan:
    \begin{exampleblock}{(Potencial de interacción a pares tipo Yukawa)}
        $$
          V(r)=\frac{V_0}{r} exp[-\lambda(r-1)]
        $$
    \end{exampleblock}
\end{frame}