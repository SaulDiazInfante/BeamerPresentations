%%%%%%%%%%%%%%%%%%%%%%%%%%%%%%%%%%%%%%%%%%%%%%%%%%%%%%%%%%%%%%%%%%%%%%%%%%%
%\begin{bibunit}[aalpha]
\begin{frame}{Ecuaciones diferenciales estoc\'asticas}
  %\begin{columns}
  %  \column{.5\textwidth}
    \only<+>{
    \begin{block}{EDE en el sentido de It\^o}
    \begin{equation*}
      dX_t=\underbrace{f(t,X_t)dt}_{\text{arrastre}}+\underbrace{g(t,X_t)dB_t}_{\text{difusi\'on}}, \quad
X_0=x_o.
    \end{equation*}

      \begin{itemize}
          \item Ruido Aditivo $g=g(t)$.
          \item Ruido Multiplicativo.
      \end{itemize}
    \end{block}
    }
\end{frame}
%%%%%%%%%%%%%%%%%%%%%%%%%%%%%%%%%%%%%%%%%%%%%%%%%%%%%%%%%%%%%%%%%%%%%%%%%%%%%%%%
\begin{bibunit}[aalpha]
\begin{frame}{Ecuaciones diferenciales estoc\'asticas}
  \begin{Teorema}[Teorema de existencia y unicidad \cite{Oeksendal}.]
  \begin{overlayarea}{\textwidth}{.4\textheight}
  \only<1-2>{
  $T>0$ y $f(\cdotp,\cdotp):[0,T]\times \mathbb{R}^n \to \mathbb{R}^n$,
  $g(\cdotp,\cdotp): [0,T]:\mathbb{R}^n\to \mathbb{R}^{n\times m}$ funciones medibles.
  }
  \only<+>{
  \begin{equation*}
    |f(t,x)|+|g(t,x)|\leq C(1+|x|), \qquad x\in \mathbb{R}^n , t \in [0,T]
  \end{equation*}
   %$C$ constante, ($|g|^2=\sum |g_{i,j}|$) y tales que
  \begin{equation*}
    |f(t,x)-b(t,y)|+|g(t,x)-g(t,y)|\leq D|x-y|, \qquad x,y\in \mathbb{R}^n, t\in[0,T],
  \end{equation*}
  %$D$ constante.
  }
  \only<+>{
  \par
  Sea $Z$ una \emph{variable aleatoria independiente} de la $\sigma $-\'algebra
  $A_{\infty}$ generada por el movimiento Browniano $B_s(\omega),s\geq 0$ y tal que
  \begin{equation*}
    \mathbb{E}(|Z|^2)<\infty.
  \end{equation*}
  }
  \only<+>{
  Entonces la ecuaci\'on diferencial estoc\'astica
  \begin{equation*}
   dX_t=f(t,X_t)+g(t,X_t)dB_t,  \qquad 0\leq t \leq T, X_0=Z
  \end{equation*}
  tiene soluci\'on \'unica $t$-continua, $X_t(\omega)$, adaptada a la filtraci\'on
  $\mathcal{A}_t^Z$ generada por $Z$ y $B_s(\cdot), s\leq t$.
  }
  \only<+>{
  Adem\'as
  \begin{equation*}
    \mathbb{E}
    \left(
      \int_0^T |X_t|^2 dt
    \right)<\infty.
  \end{equation*}
  }
  \end{overlayarea}
\end{Teorema}
\biblio{BibliografiaTesis}
\end{frame}
\end{bibunit}
%%%%%%%%%%%%%%%%%%%%%%%%%%%%%%%%%%%%%%%%%%%%%%%%%%%%%%%%%%%%%%%%%%%%%%%%%%%%%%%%
\begin{frame}{Ecuaciones diferenciales estoc\'asticas}
    \begin{block}{Consideramos una EDE autonoma }
  \begin{overlayarea}{\textwidth}{.5\textheight}
    \begin{equation*}
      dX_t=\underbrace{f(X_t)dt}_{\text{arrastre}}+\underbrace{g(X_t)dB_t}_{\text{difusi\'on}}, \quad X_0=x_o.
    \end{equation*}
    \only<+>{
    En forma integral
    $$
       X_t=x_0+\int_0^t f(X_s)ds+\int _0^t g(X_s)dB_s, \quad 0\leq t \leq T.
    $$
    }
    \only<+>{
    \begin{align*}
      X({t_j})=&X({t_{j-1}})+\int_{t_{j-1}}^{t_j}f(X(s))ds+\int_{t_{j-1}}^{t_j}g(X(s))dB_s,\\
      t_j=&j\Delta t\in[0,T], \quad j=0\dots N.\\
    \end{align*}
    }
  \end{overlayarea}
    \end{block}
%  \column{.5\textwidth}
%\end{columns}
 % \biblio{BibliografiaTesis}
\end{frame}
%\end{bibunit}
%%%%%%%%%%%%%%%%%%%%%%%%%%%%%%%%%%%%%%%%%%%%%%%%%%%%%%%%%%%%%%%%%%%%%%%%%%%%%%%%%%%%%%%%%%%
%\begin{bibunit}[aalpha]
\begin{frame}{Ecuaciones diferenciales estoc\'asticas}
  %\begin{columns}
  %  \column{.5\textwidth}
    \begin{block}{Construcción de un EMM}
  %\begin{overlayarea}{\textwidth}{.5\textheight}
    \begin{equation*}
      dX_t=\underbrace{f(X_t)dt}_{\text{arrastre}}+\underbrace{g(X_t)dB_t}_{\text{difusi\'on}}, \quad X_0=x_o.
    \end{equation*}
    \begin{itemize}
      \item
        \only<+>{
          Se aproxima  $dX_t=f(X_t)dt, \quad X_0=x_0$ con un MDFNE,
        }
        \only<+>{
          $X_{j+1}=\phi(X_j,t_j,t_{j+1}), \quad X_0=\phi(x_0,t_0,t_{0})$,
        }
        \only<+>{
          $X_{j+1}=\phi(X_j,t_j,t_{j+1}), \quad X_0=\phi(x_0,t_0,t_{0})$.
        }
    \end{itemize}
  %\end{overlayarea}
    \end{block}
%  \column{.5\textwidth}
%\end{columns}
 % \biblio{BibliografiaTesis}
\end{frame}
%\end{bibunit}
\begin{bibunit}[aalpha]
\begin{frame}{Propiedades Te\'oricas para EDEs Multiplicativas}
  \begin{overlayarea}{\textwidth}{.6\textheight}
  %\end{overlayarea}
  \only<+>{
  \begin{Definicion}[orden de convergencia fuerte]\label{dfn:StorngConvergence}
  Sea $X_n$ una aproximaci\'on a tiempo discreto de un proceso estocástico continuo en el
  tiempo $X$.  Se dice que  $X_n$ \emph{converge en sentido  fuerte}  a $X$,  con  orden
  $\gamma$,  si para cualesquiera $t_n =n\Delta t \in[0,T]$  fijo y $\Delta t$
  suficientemente peque\~no, existe una constante $C>0$ tal que
    \begin{equation*}
      \mathbb{E}(|X_n-X(t_n)|) \leq C \Delta t^{\gamma}.
  \end{equation*}
  \end{Definicion}
  }
  \only<+>{
    \begin{Definicion}[error en sentido fuerte]
      Sean $X$, $X_n$ como en la definición \autoref{dfn:StorngConvergence} al número
      \begin{equation}\label{eqn:StrongError}
        \epsilon_{\Delta t}^{fuerte}:= \mathbb{E}(|X_L-X(T)|),\qquad T=L\Delta t
      \end{equation}
      se le nombra \emph{error a tiempo final $L$}  en \emph{sentido fuerte}.
    \end{Definicion}
  }
  \only<+>{
  \begin{alertblock}{Ecuacion de prueba}
    \begin{align*}
      %dX_t=&\mu X_tdt+\lambda X_tdB_t, \qquad X_0=x_0\\
      dX_t=&\mu X_tdt+\lambda dB_t, \quad \mu, \lambda \in \mathbb{C}
    \end{align*}
  \end{alertblock}
  }
  \only<+>{
  \begin{Proposicion}
    La  soluci\'on de $dX_t=\mu X_tdt+\lambda X_tdB_t, \qquad X_0=x_0$.
    \begin{equation*}
    X_t=x_0\exp
    \left(
      \left[
      \mu-\frac{1}{2} \lambda^2
      \right]t +\lambda  B_t
    \right).
    \end{equation*}
   \begin{enumerate}
      \item
      $\mathbb{E}(X_t)=x_0\exp(\mu t)$.
      \item
      $\mathbf{var}(X_t)=
        x_0\exp(2\mu t)(\exp(\lambda^2t)-1)
      $.
      \item
       $\mathbf{cov}(X_t,X_s)=
        x_0^2\exp(\mu(t+s))(\exp(\lambda^2s)-1), \qquad 0\leq s <t.
       $
  \end{enumerate}
  \end{Proposicion}
  }
  \end{overlayarea}
%\biblio{BibliografiaTesis}
\end{frame}
\end{bibunit}
%%%%%%%%%%%%%%%%%%%%%%%%%%%%%%%%%%%%%%%%%%%%%%%%%%%%%%%%%
\begin{bibunit}[aalpha]
\begin{frame}{Propiedades Te\'oricas para EDEs Multiplicativas}
  %\begin{overlayarea}{\textwidth}{.6\textheight}
  \begin{columns}
    \column{.3\textwidth}
       \begin{exampleblock}{El método EMM para la ecuacion de prueba}
         tienen un orden de convergencia $\gamma=\frac{1}{2}$.
        \cite{Higham01}
       \end{exampleblock}
     \column{.5\textwidth}
     \begin{block}{$dX_t=2X_t dt+X_t dX_t, \qquad X_0=1$}
       \includegraphics[width=\textwidth]{./images/ConvergenciaEMM.png}
    \end{block}
   \end{columns}
  %\end{overlayarea}
  \biblio{BibliografiaTesis}
\end{frame}
\end{bibunit}
%%%%%%%%%%%%%%%%%%%%%%%%%%%%%%%%%%%%%%%%%%%%%%%%%
\begin{bibunit}[aalpha]
\begin{frame}{Propiedades Te\'oricas para EDEs Multiplicativas ($A$-estabilidad)}
  %\begin{overlayarea}{\textwidth}{.6\textheight}
  \begin{Definicion}
    Dadas $f,g$ funciones que satisfacen las condiciones del teorema de
    existencia y unicidad. Sea $X_t$ la soluci\'on fuerte,
    de la ecuación
    $$
    dX_t=f(X_t)dt+g(X_t)dB_t, \qquad X_0\neq 0 \text{ con probabilidad 1},
    $$
    entonces se dice que  $X_t$ es una soluci\'on MC-estable si y s\'olo si
     $$
      \lim _{t\to\infty}\mathbb{E}(|X_t|^2)=0.
     $$\cite{saito1996stability}
  \end{Definicion}
\biblio{BibliografiaTesis}
\end{frame}
\end{bibunit}
%%%%%%%%%%%%%%%%%%%%%%%%%%%%%%%%%%%%%%%%%%%%%%%%%%%%%%%%%%%%%%%%%%%%%%%%%%%%%
%%%%%%%%%%%%%%%%%%%%%%%%%%%%%%%%%%%%%%%%%%%%%%%%%%%%%%%%%%%%%%%%%%%%%%%%%%%%%
\begin{frame}{Propiedades Te\'oricas para EDEs Multiplicativas}
  \begin{columns}
  \column{.45\textwidth}
  \begin{block}{$\lim\limits_{t\to\infty}\mathbb{E}(|X_t)=0\Leftrightarrow$}
    \begin{itemize}
      \item Problema \\
      $
            2\mathbf{Re}(\mu)+|\sigma|^2<0.
      $
      \item EMM\\
      $
          |1+\mu\phi(\Delta t)|^2 + |\sigma|^2 (\Delta t)<1.
        $
    \end{itemize}
  \end{block}
  \only<2>{
    \begin{align*}
      dX_t=&-3X_t dt+ \sqrt{3} X_tdB_t,
     X_0=&1.
    \end{align*}
  }
\column{.55\textwidth}
    \begin{exampleblock}{%Regi\'on de estabilidad EMM
     $\phi(\Delta t)=(1-\exp(-\mu\Delta t))\mu^{-1}$}
            \includegraphics[width=\textwidth]<1| handout:0>{./images/RegionEstabilidadBiPhi.png}
            \includegraphics[width=\textwidth]<2>{./images/StabilityEMM.png}
    \end{exampleblock}
     \end{columns}
\end{frame}
%%%%%%%%%%%%%%%%%%%%%%%%%%%%%%%%%%%%%%%%%%%%
\begin{bibunit}[aalpha]
\begin{frame}{Propiedades Te\'oricas para EDEs Aditivas $A-estabilidad$}
  \begin{equation*}
      dX_t=f(X_t)dt+\gamma dB_t \qquad X_0=x_0
    \end{equation*}
\begin{Definicion}
    Un algor\'itmo EMM para la ecuaci\'on de prueba es $A$-estable si su
    correspondiente funci\'on de estabilidad  $\mathcal{R}$  mapea a cada  $z\in\mathbb{C}^-$ en el disco
unitario
    complejo.
\end{Definicion}
\begin{Teorema}
    El m\'etodo EMM es $A$-estable si y s\'olo si, lo es su componente determinista.
  \cite{hernandez1992stability}.
\end{Teorema}
  \biblio{BibliografiaTesis}
\end{frame}
\end{bibunit}
%%%%%%%%%%%%%%%%%%%%%%%%%%%%%%%%%%%%%%%%%%%%%%%%%%%%%%%%%%%%%%%%%%%%%%%%%%%%%%%%%%%%%%%%%%%%%%%%
\begin{frame}{Propiedades Te\'oricas para EDEs Aditivas $A-estabilidad$}
  \begin{equation*}
      dX_t=f(X_t)dt+\gamma dB_t \qquad X_0=x_0
    \end{equation*}
  \begin{block}{Bosquejo}
    \only<+>{
      $\Rightarrow$
      $$\frac{X_{k+1}-X_{k}}{\phi(\Delta t)}=\Phi(X_k,\lambda,\Delta t)$$
      $$X_{j+1}=\underbrace{\phi(\Delta t) \Phi(X_j,\lambda,\Delta t)+X_j}_{= A(X_j)}+\gamma \Delta B_j,$$
      $$\mathbb{E}(X_{j+1})=\mathbb{E}\left(A(X_j )\right)+\gamma\mathbb{E}(\Delta B_t)$$
    }
    \only<+>{
      $\Leftarrow$
      $$\lim_{j\to\infty}\mathbb{E}(|X_{j}|^2)=0.$$
      Convergencia en media cuadr\'atica implica la convergencia en media
      $$
        \lim_{j\to\infty}\mathbb{E}(|X_{j}|)= \lim_{j\to\infty}A(X_j) =0.
      $$
    }
  \end{block}
\end{frame}

