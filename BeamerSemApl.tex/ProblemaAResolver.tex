%======================================
%Pendiente
%======================================
%\begin{frame}
  %\frametitle{Movimiento Browniano}
 % \flashmovie[width=.8\textwidth,height=.7\textheight,engine=jw-player,auto=0]{Movimiento4.mp4}
%\end{frame}
%%%%%%%%%%%%%%%%%%%%%%%%%%%%%%%%%%%%%%%%%%%%%%%
\definecolor{DarkSlateGrey}{HTML}{001B0C}
\definecolor{LightSteelBlue}{HTML}{B3D7F6}
\definecolor{DarkGray}{HTML}{394F50}
\definecolor{LightGoldenrodYellow}{HTML}{F8FDCB}
\setbeamercolor{color titulo caja}{fg=DarkSlateGrey,bg=LightSteelBlue} %%
\setbeamercolor{color cuerpo caja}{fg=DarkGray,bg=LightGoldenrodYellow}%
\begin{frame}%[label=frm:4]{}
  \frametitle{Algoritmo de dinamica Browniana}
  \tikzstyle{decision} = [diamond, draw, fill=yellow!20,
    text width=4.5em, text badly centered, node distance=4cm, inner sep=5pt]
  \tikzstyle{block} = [rectangle, draw, fill=blue!20,
    text width=5em, text centered, rounded corners, minimum height=4em]
  \tikzstyle{blockIm}= [rectangle, draw, fill=red!40,
    text width=6em, text centered, rounded corners, minimum height=4em]
  \tikzstyle{line} = [draw, -latex]
  \tikzstyle{cloud} = [draw, ellipse,fill=red!20, node distance=3cm,
    minimum height=2em]
  \frametitle{Simulación}
  \begin{center}
	\begin{tikzpicture}[node distance = 2cm, auto]
    % Place nodes
	  \node [block] (Init) {Inicializar};
	  \node [block, below of=Init] (Fuerza) {Calculo de Fuerza};
	  \node [decision, left of=Fuerza] (Serie) {Serie de Tiempo};
	  \node [blockIm, below of=Fuerza] (Posiciones) {Posiciones};
	  \node [block, left of=Posiciones,node distance=6cm] (Promedio){Promediar
	      cantidades de interes};
    % Draw edges
	  \path [line] (Init) -- (Fuerza);
	  \path [line] (Fuerza) --(Posiciones);
	  \path [line] (Posiciones)-|(Serie);
	  \path [line] (Serie)--(Fuerza);
	  \path [line] (Serie)-| (Promedio);
	\end{tikzpicture}
  \end{center}
\end{frame}
% \begin{frame}
%  \frametitle{Simulación}
% \begin{center}
%   \includegraphics[scale=1]{./images/DiagramaSDB.png}
%  % DiagramaSDB.png: 338x215 pixel, 100dpi, 8.59x5.46 cm, bb=0 0 243 155
% \end{center}
% \end{frame}
%%%%%%%%%%%%%%%%%%%%%%%%%%%%%%%%%%%%%%%%%%%%%%%
% Define block styles
% \tikzstyle{decision} = [diamond, draw, fill=blue!20, text width=4.5em, text badly centered,
% node distance=3cm, inner sep=0pt]
% \tikzstyle{block} = [rectangle, draw, fill=blue!20, text width=6em, text centered, rounded corners,
% minimum height=3.5em]
% \tikzstyle{line} = [draw, -latex']
% \tikzstyle{cloud} = [draw, ellipse,fill=red!20, node distance=3cm, text width=4em, text centered, minimum
% height=2em]
% \everymath{\displaystyle}
%   \begin{frame}
%      \frametitle{Simulaci\'on}%
%        \begin{tikzpicture}[node distance = 2cm, auto]
% %         % Place nodes
%          \node [block] (INI) {Inicializaci\'on};
%          \node [block, below of= INI] (CF) {Calculo de Fuerzas};
%          \node [block, below of= CF] (AP) {Actualizaci\'on de posiciones};
%          \node [cloud, left of= CF] (ST) {Serie de Tiempo};
%          \node [block, left of=ST, node distance=3cm] (PCI) {Promediar Cantidades de inter\'es};
% %         % Draw edges
%          \path [line] (INI) -- (CF);
%          \path [line] (CF) -- (AP);
%          \path [line] (AP) -| (ST);
%          \path [line] (ST) |- (INI);
%          \path [line] (ST) -- (PCI);
%      \end{tikzpicture}
%   \end{frame}

% %%%%%%%%%%%%%%%%%%%%%%%%%%%%%%%%%%%%%%%%%%%%%%%
\begin{frame}
  \frametitle{Método convencional de  dinámica Browniana}
  \begin{columns}
     \column{.6\textwidth}
     \begin{exampleblock}{Aproximación en primer orden de Ermak y McCammon}
     \begin{align}
        r_{i\alpha}(\Delta t)=&r_{i\alpha}+\frac{D}{T}F_{i\alpha}\Delta t+R_{i\alpha}\\
        \langle R_{i\alpha}\rangle=&0 \label{eqn:mediaEMc}\\
        \langle R_{i\alpha} R_{j\beta}\rangle=&2D\Delta t \delta_{ij}\delta_{\alpha\beta}\label{eqn:CovEMc}
     \end{align}
    \end{exampleblock}
    \column{.5\textwidth}
     \begin{itemize}
      \item $r_{i\alpha}\equiv r_{i\alpha}(0)$: posición inicial.
      \item $\Delta t:$ incremento temporal.
      \item $F_{i\alpha}:$ fuerza neta sobre la partícula $i$ en la dirección $\alpha$.
      \item $\{i,j\},\{\alpha,\beta,\gamma\}$: partículas, ejes coordenados respectivamente
      \item $R_{i\alpha}:$ desplazamiento aleatorio con distribución Gaussiana,  media y covarianza
     como en (\ref{eqn:mediaEMc}),(\ref{eqn:CovEMc}) respectivamente.
      \item $D=\frac{k_B T}{\gamma}$: coeficiente de difusión de Stokes - Einstein
    \end{itemize}
   \end{columns}
\end{frame}
