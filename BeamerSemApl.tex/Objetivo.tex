\begin{bibunit}[aalpha] 
  \begin{frame}
    \frametitle{Objetivo principal}
    \begin{alertblock}{Responder a la siguiente hip\'otesis:}
      Al aplicar la formulaci\'on de los esquemas en diferencias no est\'andar \cite{Mickens:2005} para
      resolver la ecuaci\'on  de Langevin,
        $$
          \frac{dX}{dt}=\beta D F+D^{\frac{1}{2}}\xi,
        $$
      \emph{se obtiene un esquema num\'erico con mejores propiedades de estabilidad y de simplicidad algebraica
      similar al   CBD}.
    \end{alertblock}
    \biblio{BibliografiaPresentacion} 
  \end{frame}
\end{bibunit}
%%%%%%%%%%%%%%%%%%%%%%%%%%%%%%%%%%%%%%
\begin{frame}
    \frametitle{Objetivos}
    \begin{exampleblock}{Objetivos particulares:}
      \begin{itemize}
        \item
          Formular un algoritmo en diferencias finitas no estándar estocástico para la formulación de Langevin.
        \item
          Estudio cualitativo del algorítmo.
        \item
          Validación  en el contexto de simulaci\'on.
      \end{itemize}
    \end{exampleblock}
  \end{frame}
