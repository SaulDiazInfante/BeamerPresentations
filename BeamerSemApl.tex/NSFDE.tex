\tikzstyle{every picture}+=[remember picture]
%%%%%%%%%%%%%%%%%%%%%%%%%%%%%%%%%%%%%%%%%%%%%%%%%%

\begin{bibunit}[aalpha]
\begin{frame}{MDFNS}
  \begin{block}{génesis}
    Los métodos en diferencias finitas no estándar (MDFNE) para integrar ecuaciones diferenciales tienen su
génesis
    en el articulo \cite{Mickens1989}. Las reglas para su construcción  aparecen en \cite{mickens1994nonstandard}
y su aplicación
    especifica a varios problemas no lineales en varias publicaciones \cite{Mickens}.

  \end{block}
  \biblio{BibliografiaTesis}
\end{frame}
\end{bibunit}
\begin{bibunit}[aalpha]
\begin{frame}
  \begin{exampleblock}{Aplicaciones}
  \begin{overlayarea}{\textwidth}{.2\textheight}
  \begin{itemize}
     \item
      \only<+>{
        problemas con frontera singular en coordenadas cilíndricas o esféricas
        \cite{buckmire2003investigations} ,
      }
      \only<+>{
        un modelo  reacci\'on difusi\'on generalizado de Nagumo  \cite{Chen2003},
      }
      \only<+>{
       ecuaciones de modelado para  la estructura estelar \cite{mickens2003non},
      }
      \only<+>{
       din\'amica de la transmisi\'on del VIH \cite{Gumel2004},
      }
      \only<+>{  problemas de transporte lineal modificado de calor / difusión\cite{mickens2004positivity}.
      }
  \end{itemize}
  \end{overlayarea}
  \end{exampleblock}
  \biblio{BibliografiaTesis}
\end{frame}
\end{bibunit}

\begin{frame}
    \frametitle{Esquemas en diferencias finitas no est\'andar}
    % Below we mix an ordinary equation with TikZ nodes. Note that we have to
    % adjust the baseline of the nodes to get proper alignment with the rest of
    % the equation.
    \begin{itemize}
        \item <2-> Se aproxima por
      $\displaystyle \frac{y_{k+1}-y_{k}}{\textcolor{red}{\phi(h)}}$,\\
      $ \phi(h)=h+\mathcal{O}(h^2)$.
      \tikz[na] \node [coordinate] (n1) {};
    \end{itemize}
    \begin{equation*}
      \tikz[baseline]{
          \node[fill=blue!20,anchor=base] (t1)
          {$\frac{dy}{dt}$};
      }=
      \tikz[baseline]{
          \node[fill=red!20, ellipse,anchor=base] (t2)
          {$F(t,y,y_0,\lambda)$};
      }
    \end{equation*}

    \begin{itemize}
        \item<3-> se discretiza  de \emph{forma no  local}.
      \tikz[na] \node [coordinate] (n2) {};
    \end{itemize}

    % Now it's time to draw some edges between the global nodes. Note that we
    % have to apply the 'overlay' style.
    \begin{tikzpicture}[overlay]
      \path[->]<2-> (n1) edge [out=0, in=180] (t1);
      \path[->]<3-> (n2) edge [bend right] (t2);
    \end{tikzpicture}
  \end{frame}
%%%%%%%%%%%%%%%%%%%%%%%%%%%%%%%%%%%%%%%%%%%%%%%%
 \begin{frame}
     \tikzset{
           state/.style={
          rectangle,
          rounded corners,
          draw=black, very thick,
          minimum height=2em,
          inner sep=4pt,
          text centered,
          fill=gray!20
          },
    }
    \frametitle{Esquemas en diferencias finitas no est\'andar}
    \begin{tikzpicture}[->,>=stealth,overlay]
     \node[state,xshift=2cm, text width=.3\textwidth] (EDO)%
     {%
      \begin{align*}
        \frac{dy}{dt}=&f(y,t,\lambda),\\
        y(t_0)=&y_0.\\
        \\
        y(t)=&\phi(\lambda,y_0,t_0,t),\\
        y_0=&\phi(\lambda,y_0,t_0,t_0).
      \end{align*}
     };
     % State: ACK with different content
     \node[state,       % layout (defined above)
      text width=.35\textwidth,   % max text width
      yshift=1cm,       % move 2cm in y
      right of=EDO,     % Position is to the right of QUERY
      node distance=6.5cm,    % distance to QUERY
      anchor=center] (EDF)    % posistion relative to the center of the 'box'
     {%
      \begin{align*}
        y_{k+1}=&g(\lambda,h,y_k,t_k), \\
        t_k=&hk.\\
        \\
        y_k=&\psi(\lambda,h,y_0,t_0,t_k),\\
         y_0=&\psi(\lambda,h,y_0,t_0,t_0).
      \end{align*}
     };
    %draw path
    \path (EDO)  edge[bend left=20]  (EDF);
     \end{tikzpicture}
  \end{frame}

\begin{frame}[dfn:EsquemaExacto]
  \frametitle{Esquemas exactos}
  \begin{Definicion}[Esquema en diferencias exacto] 
    Un esquema en diferencias se dice exacto, si su soluci\'on general y la de la EDO asociada, son
    equivalentes,  es decir,si y s\'olo  si
    $$
      y_k=y(t_k), \quad  k=0,1,\dots
    $$
    y para todo valor de $h$.
   \end{Definicion}
  \uncover<2>{
  \begin{Teorema}
    A cada ecuaci\'on diferencial
    $
      \frac{dy}{dt}=f(y,t,\lambda),  \qquad y(t_0)=y_0,
    $
    le corresponde un esquema de diferencia finita dado por
    $$
    y_{k+1}=\phi(\lambda,y_k,t_k,t_{k+1}),
    $$
    donde $\phi$  es la soluci\'on general de la EDO.
  \end{Teorema}
  }
\end{frame}
