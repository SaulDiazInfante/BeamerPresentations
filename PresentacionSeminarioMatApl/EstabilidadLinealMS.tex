\begin{bibunit}[apalike]
\begin{frame}%[label=frm:20]
    \frametitle{Estabilidad en Media Cuadrática}
	\biblio{BibliografiaTesis}
\end{frame}
\end{bibunit}
%%%%%%%%%%%%%%%%%%%%%%%%%%%%%%%%%%%%%%%%%%%%%%%%%%%%%%%%%%%%%%%%%%%
%\begin{bibunit}[apalike]
\begin{frame}%[label=frm:20]
    \frametitle{Estabilidad en Media Cuadrática}
	\begin{overlayarea}{\textwidth}{.25\textheight}	
		\only<1-12>{
			\begin{empheq}[box=\shadowbox*]{equation*}
				dX_t=\lambda X_t dt +\beta dB_t, \quad X_0=cte, \lambda, \beta \in
				\mathbb{C} \qquad \text{(DP)}
			\end{empheq}	
		}
		\only<13>{
		 	\begin{empheq}[box=\shadowbox*]{equation*}				
				\lim_{n \to \infty}\mathbb{E}[|X_n|^2]=-\frac{|\beta|^2}{2Re(\lambda)}.
			\end{empheq}	
		}	
	\end{overlayarea}
	\begin{columns}	
		\column{.5\textwidth}		
		\begin{overlayarea}{\textwidth}{\textheight}				
			\centering{Euler-Mayurama}		
			\only<1->{				
				\begin{empheq}[box={\ovalbox}]{align*}
						X_{n+1} = &(1+\lambda h) X_n +\beta \Delta B_n 
				\end{empheq}							
			}
			\only<2-5>{
				\begin{empheq}{align*}		
					\mathbb{E}(|X_{n+1}|^{2})=&
						|1+\lambda h|^{2}\mathbb{E}(|X_{n}|^{2})+|\beta|^{2}h
				\end{empheq}
			}
			\only<3-4>{
				\scalebox{0.9}{% Scale by 90%									
					$\displaystyle
						=
						|1+\lambda h|^{2}(|1+\lambda h|^{2}\mathbb{E}(|X_{n-1}|	^{2})+|\beta|^{2}h)
						+|\beta |^{2}h
					$
				}					
			}
 		\only<4>{					
				\scalebox{0.9}{% Scale by 90%									
					$\displaystyle 
						=
						|1+\lambda h|^{4}\mathbb{E}(|X_{n-1}|^{2})
						+
						\left[|1+\lambda h|^{2}+1\right]|\beta|^{2}h\				
					$						
				}
				$$
						\vdots
				$$
		}
		\only<5-6>{
			\scalebox{0.9}{% Scale by 90%
					$ =
						|1+\lambda h|^{2(n+1)}\mathbb{E}(|X_{0}|^{2})+
						\underbrace{						
							\left[|1+\lambda h|^{2n}+\cdots+|1+\lambda h|^{2}+1\right]
							}_{\textit{Serie geométrica}}
							|\beta|^{2}h
					$
			}
		}
		\only<6>{
			\scalebox{0.9}{% Scale by 90%
					 $	=
					 	|1+\lambda h|^{2(n+1)}
					 	\mathbb{E}(|X_{0}|^{2})+
					 	\frac{|1+\lambda h|^{2(n+1)}-1}{|1+\lambda h|^{2}-1}
					 	|\beta|^{2}h
					$	
			}
			\scalebox{0.9}{% Scale by 90%
					$					 
					  =
					 	|1+\lambda h|^{2(n+1)}\mathbb{E}(|X_{0}|^{2})+
					 	\frac{|1+\lambda h|^{2(n+1)}-1}{2Re(\lambda)+|\lambda|^{2}h}|\beta|^{2}.			
					$
			}		
		}
		\only<7-12>{
			\begin{empheq}{align*}
					\mathbb{E}(|X_{n+1}|^{2}) & = |1+\lambda h|^{2(n+1)}\mathbb{E}(|X_{0}|^{2})\\
						&+
							\frac{|1+\lambda h|^{2(n+1)}-1}{2Re(\lambda)+|\lambda|^{2}h}|\beta|^{2}\\
			\end{empheq}
		}
		\only<13>{
			Si $Re(\lambda h)<0$			
			\begin{align*}
  					\mathbb{E}[|X_{n+1}|^2]
  					\xrightarrow{n\to\infty}  			
  					\colorbox{darkyellow}{
  				 		$\frac{-|\beta|^2 }{2Re(\lambda)+|\lambda|^2h}$
  					}
  				\end{align*}
		}
		\end{overlayarea}
%--------------------------------------------------------------------------------		
		\column{.6\textwidth}		
		\begin{overlayarea}{\textwidth}{\textheight}	
%			\only<8->{
%				\centering{\alert<5->{\textbf{Steklov}}}					
%					\begin{empheq}[box=\mybluebox]{align*}
%						X_{n+1} = & X_n \exp(\lambda h)+\beta \Delta B_n
%					\end{empheq}
%			}						
%			\only<9>{	
%				\begin{equation*}
% 					\mathbb{E}[|X_{n+1}|^2]=\exp(2Re(\lambda)h)\mathbb{E}[|X_n|^2]+|\beta|^2 h.
%				\end{equation*}	
%			}
%			\only<10>{	
%					\begin{align*}
%						\mathbb{E}[|X_{n+1}|^2]&=\exp(2Re(\lambda)h)\mathbb{E}[|X_n|^2]+|\beta|^2 h\\
%  						\vdots &=\vdots\\
%								&= \exp(2Re(n+1)\lambda h)\mathbb{E}[|X_{0}|]\\
%								&+|\beta|^2h 
%	    				\underbrace{
%	      					\left(
%	      						1+\cdots+\exp(2nRe(\lambda) h)
%	     						\right)
%	    						}_{\text{Serie geométrica}}\\
%  					\end{align*}	
%			}
%			\only<11>{
%				\begin{align*}
%					\mathbb{E}[|X_{n+1}|^2]&=\exp(2(n+1)Re(\lambda)h)\mathbb{E}[|X_0|^2]\\
%  							+&\beta^2 h \frac{(\exp(2Re(\lambda)h))^{n+1}-1}{\exp(2Re(\lambda)h)-1}  		
%  				\end{align*}
%			}	
%			\only<12->{
%				Si $Re(\lambda)<0$
%				\begin{align*}
%  					\mathbb{E}[|X_{n+1}|^2]
%  					\xrightarrow{n\to\infty}
%  					\colorbox{hellcyan}{
%  				 		$\frac{-|\beta|^2 h}{\exp(2Re(\lambda)h)-1}$
%  					}
%  				\end{align*}		
%			}
		\end{overlayarea}
	\end{columns}
	%\biblio{BibliografiaTesis}
\end{frame}
%\end{bibunit}
%%%%%%%%%%%%%%%%%%%%%%%%%%%%%%%%%%%%%%%%%%%%%%%%%%%%%%%%%%%%%%%%%%%%%%%%%%%%%%%%%%%%%%%%%%%%%%%%%
\begin{frame}
  \frametitle{Estabilidad en Media Cuadrática}
  \begin{empheq}{equation*}
 		dX_t=\lambda X_t dt +\beta dB_t, \qquad X_0=cte., \lambda, \beta \in \mathbb{C} \tag{E}.
  \end{empheq}
   \begin{columns}
	  \column{.5\textwidth}
 	  \begin{definicion}[Consistencia lineal en MC]
 		Un esquema numérico para  la ecuación (E) se dice asintóticamente consistente en media
 		cuadrática si la solución numérica satisface
 		$$
 		  \lim_{\substack{ n\to \infty\\ h\to 0}} X_n= \frac{-\beta}{2Re(\lambda)}
 		$$
 	  \end{definicion}
	  \column{.5\textwidth}
% 	  \begin{Teorema}
% 		 El esquema Steklov para la ecuación (E), es asintóticamente consistente en MC.
% 	  \end{Teorema}
  \end{columns}
\end{frame}
%%%%%%%%%%%%%%%%%%%%%%%%%%%%%%%%%%%%%%%%%%%%%%%%%%%%%%%%%%%%%%%%%%%%%%%%%%%%%%%%%%%%%%%%%%%%%%%%%