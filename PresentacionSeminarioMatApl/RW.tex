%%%%%%%%%%%%%%%%%%%%%%%%%%%%%%%%%%%%%%%%%%%
	\begin{frame}{Historia}
		\begin{center}
			\movie[width=9.1cm,height=5.2cm,showcontrols=true,loop,poster]{}{BrownianMotion.flv}
			\end{center}							
		\end{frame}
%%%%%%%%%%%%%%%%%%%%%%%%%%%%%%%%%%%%%%%%%%%
%%%%%%%%%%%%%%%%%%%%%%%%%%%%%%%%%%%%%%%%%%
\begin{frame}{Caminata Aleatoria}
	\begin{overlayarea}{\textwidth}{.7\textheight}
	\only<1>{	
	\begin{center}
		\includegraphics[width=\textwidth]{./IMAGENES/RW/RandomWalk(1).png}	
	\end{center}
	}
	\only<2>{	
		\begin{center}
			\includegraphics[width=\textwidth]{./IMAGENES/RW/RandomWalk[2].png}	
		\end{center}
	}
	\only<3>{
		\begin{center}
			\includegraphics[width=\textwidth]{./IMAGENES/RW/RandomWalk[3].png}	
		\end{center}	
	}	
	\end{overlayarea}
\end{frame}

%%%%%%%%%%%%%%%%%%%%%%%%%%%%%%%%%%%%%%%%%%%
\begin{frame}{Construcci\'on}
	\begin{overlayarea}{\textwidth}{\textheight}	
		\begin{columns}	
			\column{.5\textwidth}
				\uncover<2->{
					\begin{empheq}{align*}
						&\{ X_{n} \}_{n=1}^{\infty}   \quad v.a.i.i.d\\
						&P(X_{j} = \pm h)= \frac{1}{2}.
					\end{empheq}
				}
				\uncover<3->{ 
					\begin{empheq}[box=\shadowbox*]{align*}
						Y_{\delta, h}(0) = &0\\
						Y_{ \delta,h}(n \delta ) = &X_{1}+X_{2}+\cdots+X_{n}.	
					\end{empheq}	  
				}
				\uncover<4->{
				Interpola linealmente				
				\begin{empheq}{align*}
					Y_{\delta,h}(t) = &\frac{(n + 1)\delta -t}{\delta} Y_{\delta,h}(n \delta )\\
											 + &  \frac{t - n\delta}{\delta}Y_{\delta,h}((n+1) \delta ).	\\
					&n\delta < t < (n + 1)\delta\\
					&t>	0. 
				\end{empheq}
				}			
			\column{.5\textwidth}
				\only<5->{
				\begin{block}{Queremos}
					 $$
						\lim_{
							\substack{
								\delta\to 0\\
								h\to 0								
							}	
						}				 	
					 	Y_{\delta, h}
					 $$
				\end{block}
				}
				\begin{overlayarea}{\textwidth}{.7\textheight}	
				\only<6>	{
		 			$\lambda \in \mathbb{R}$ fijo. 
		 			Calcula \\ 	\hyperlink{dfn:FuncionCaracteristica}{\beamergotobutton{caracteristica}}		
					$\displaystyle\lim_{\delta,h \to 0}
					{\mathbb{E}\left[e^{ i\lambda Y_{\delta,h}(t)}\right]}$. \\
					\hypertarget{cns:Limite}{}
			
				}			
				\only<7->{
					\textcolor<7>{red}{$t=n\delta$}, 		
				}
				\only<7>{
					\begin{eqnarray*}
						\mathbb{E}
							\left[
								e^{
									 i\lambda 					
									Y_{\delta,h}(\textcolor{red}{t})
								}
							\right]&=&
						\prod_{j=1}^{n}\mathbb{E}\left[e^{i\lambda X_{j}}\right] \\
						&=&\left( \mathbb{E}\left[e^{i\lambda X_{j}}\right]\right)^{n}\\
						&=&\left(\frac{1}{2}e^{i\lambda h}+\frac{1}{2}e^{-i\lambda h}\right)^{n}\\
						&=&\left(\cos(\lambda h)\right)^{n}\\
						&=&\left(
									\textcolor{blue}{\cos(\lambda h)}
								\right)^{
										\frac{t}{
											\textcolor{blue}{\delta}
										}
								}.
					\end{eqnarray*}
				}
				\only<8->{
					\textcolor{cyan}{					
						$ u =\left(cos(\lambda h)\right)^{\frac{1}{\delta}}$
					}				
				}
				\only<9>{				
					\begin{empheq}{align*}				
						u \approx & e^{ - \frac{1}{2\delta}\lambda^{2}h^{2}}\\
						\mathbb{E}\left[ e^{i\lambda Y_{\delta,h}(t)}\right]
						\approx & 
						e^{
							- \frac{1}{
									2
									\textcolor{red}{\delta}
							}
							 t\lambda^{2}
							\textcolor{red}{h}^{2}
						}.
				\end{empheq}
				}		
				\only<10->{
	 				\textcolor{red}{$h^{2} = \delta$}
				}
				\only<10->{
					$$
					\displaystyle\lim_{\delta,h \to 0}{\mathbb{E}\left[ e^{i\lambda Y_{\delta,h}(t)}\right]} 
					= e^{ - \frac{1}{2}t\lambda^{2}},
					\quad \quad \lambda \in \mathbb{R}.$$
				}
				\only<11>{
		 			\begin{empheq}[box=\ovalbox]{equation*}
		 			\therefore
		 			B(t)  
		     			\overset{ \mathcal{D} }{ = }
		   				\displaystyle\lim_{\delta \to 0}{Y_{\delta,h}(t)}
		 			\end{empheq}		
				}
			\end{overlayarea}				
		\end{columns}
	\end{overlayarea}
\end{frame}
%%%%%%%%%%%%%%%%%%%%%%%%%%%%%%%%%%%%%%%%%%%%%%%%%%%%%%%%%%%%%%%%%%%%%%%%%%%%%%%%%%%
\begin{frame}{Construcci\'on}
	\begin{Teorema} 
		Sea $Y_{\delta,h}(t)$ una caminata aleatoria que inicia en $0$ 
		de saltos $h$ y $-h$  con igual probabilidad en los tiempos 
		$\delta, 2\delta,3\delta, \ldots $.  
		Asumamos que $h^{2} = \delta$. 
		Entonces para cada $t \geq 0$, el limite
		$$ B(t) = \displaystyle\lim_{\delta \to 0}{Y_{\delta,h}(t)},$$
		existe en distribuci\'on. Adem\'as, tenemos que
		$$
			\mathbb{E}\left[e^{i\lambda B(t)}\right] 
			=e^{- \frac{1}{2}t\lambda^{2}}, \quad \quad \lambda \in 	\mathbb{R}.
		$$
	\end{Teorema}
\end{frame}
%%%%%%%%%%%%%%%%%%%%%%%%%%%%%%%%%%%%%%%%%%%%%%%%%%
\begin{frame}{Codigo}
	\only<+>{
  	\begin{figure}
  	\centering
      \tiny
   	\lstset{language=python}
         \lstinputlisting[firstline=3,lastline=11]{RW01.py}
   \end{figure}
}
\end{frame}
%%%%%%%%%%%%%%%%%%%%%%%%%%%%%%%%%%%%%%%%%%%
\begin{frame}{Caminata Aleatoria de $n$  transiciones}
	\begin{overlayarea}{\textwidth}{.7\textheight}
	\only<1>{	
		\begin{center}
			\includegraphics[width=\textwidth]{./IMAGENES/RW/RW01(1).png}	
		\end{center}
	}
	\only<2>{
		\begin{center}
			\includegraphics[width=\textwidth]{./IMAGENES/RW/RW01(2).png}	
		\end{center}	
	}
	\only<3>{
		\begin{center}
			\includegraphics[width=\textwidth]{./IMAGENES/RW/RW01(3).png}	
		\end{center}	
	}	
	\end{overlayarea}
\end{frame}
%%%%%%%%%%%%%%%%%%%%%%%%%%%%%%%%%%%%%%%%%%%%%%%%%%%%%%%%%%%%%%%%%%%%%%%%%
\begin{frame}{Construcci\'on}
	\begin{empheq}[box={\Garybox[Construcci\'on]}]{align*}
		h^{2}&=\delta\\	
		Y_{\delta,h}(t) 
		& \xrightarrow[\delta ,h \to 0]{\mathcal{D}} B(t) \qquad \forall t \geq 0\\
		\mathbb{E}\left[e^{i\lambda B(t)}\right] 
		&
		\xrightarrow{\delta ,h \to 0}
		e^{-\frac{1}{2}t\lambda^{2}}, \quad \lambda \in 	\mathbb{R}.	
	\end{empheq}	
\end{frame}
%%%%%%%%%%%%%%%%%%%%%%%%%%%%%%%%%%%%%%%%%%%%%%%%%%
\begin{frame}{Caminata Aleatoria en $[0,1]$}
	%\vspace{-1.5cm}		
	\begin{overlayarea}{\textwidth}{.9\textheight}
	\only<1>{	
		\begin{center}
			\includegraphics[width=.8\textwidth]{./IMAGENES/RW/RWs01.png}	
		\end{center}
	}
	\only<2>{
		\begin{center}
			\includegraphics[width=.8\textwidth]{./IMAGENES/RW/RWs01Sigma.png}	
		\end{center}	
	}
	\end{overlayarea}
\end{frame}
%%%%%%%%%%%%%%%%%%%%%%%%%%%%%%%%%%%%%%%%%%%%%%%%%%%%%%%%%%%%%%%%%%%%%%%%%
\begin{frame}{Distribuci\'on Gaussiana}
	
	\begin{center}
		\only<1>{
			\includegraphics[width=0.7\linewidth]{IMAGENES/RW/MachineGalton}
		}
		\only<2>{
			\includegraphics[width=0.7\linewidth]{IMAGENES/RW/NomralDis}
		}
	\end{center}
	
\end{frame}
%%%%%%%%%%%%%%%%%%%%%%%%%%%%%%%%%%%%%%%%%%%%%%%%%%%%%%%%%%%%%%%%%%%%%%%%%%