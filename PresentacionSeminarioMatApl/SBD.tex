\begin{frame}%[label=frm:12]
  \frametitle{Métodos Steklov para din\'amica Browniana(SBD)}
	Queremos aproximar:
	\begin{equation*}
	  dX_t=\frac{1}{k_BT} D F(X_t)+ \sqrt{D}dB_t,\quad X_0=cte\quad t\in[0,T].
	\end{equation*}
	\begin{block}{Considerando su forma integral:}
	  \begin{equation*}
		X_t=X_0+\frac{1}{k_BT} D \int_{0}^{t}F(X_s)ds+\sqrt{D}\int_{0}^t dB_s
	  \end{equation*}
	\end{block}
\end{frame}
% %%%%%%%%%%%%%%%%%%%%%%%%%%%%%%%%%%%%%%%%%%%%%%%%%%%%%%%%%%%%%%%%%%%%%%%%%%
\begin{bibunit}[apalike]
\begin{frame}%[label=frm:13]
  \frametitle{Existencia y unicidad de soluciones}
  \begin{overlayarea}{\textwidth}{.5\textwidth}
	\only<+>{
	\begin{block}{Sea $F:\mathbb{R} \to \mathbb{R}$.}
	  Asumimos que:
	  \begin{itemize}
		\item $D$, $X_0$ constantes.
		\item $\exists \alpha,\beta$ constantes t.q.
		  \begin{align*}
			&|F(x)|\leq \alpha(1+|x|) \qquad x,y\in \mathbb{R}. \\
			&|F(x)-F(y)|\leq \beta |x-y|.
		  \end{align*}
	  \end{itemize}
	\end{block}
	}
	\only<+>{
	  \begin{block}{Bajo estos supuestos  $\exists ! X_t$ t.q.}
		$$
		\mathbb{E}\left(
		\int_{0}^T|X_t|^2dt
		\right)<\infty
		$$
		\cite{KloedenPllaten}
	  \end{block}
	}
  \end{overlayarea}
  \biblio{BibliografiaTesis}
  \end{frame}
\end{bibunit}
%%%%%%%%%%%%%%%%%%%%%%%%%%%%%%%%%%%%%%%%%%%%%%%%%%%%%%%%%%%%%%%%%%%%%%%%
\begin{frame}%[label=frm:14]{}
  \frametitle{Construcción de métdos Steklov}
  \begin{overlayarea}{\textwidth}{.5\textwidth}
	\only<1-2>{
	  Fijemos notación.
	\begin{block}{Discretizamos $[0,T]$ con un  paso uniforme $h$:}
	  \begin{itemize}
		\item $t_n=nh$ $n=0,1,2,\dots, N$.
		\item $X_n \approx X_{t_n}$
	  \end{itemize}
	\end{block}
	}
	\only<2>{
	\begin{block}{Para cada nodo}
		\begin{equation*}
		  X_{t_{n+1}}=X_{t_{n}}+
		  \underbrace{\int_{t_n}^{t_{n+1}}F(X_{s_n})ds}_{\approx \text{Con algún método}}
		  +\underbrace{B_{t_{n+1}}-B_{t_n}}_{:=\Delta B_n}
		\end{equation*}
	  \end{block}
	}
	\only<3>{
	  %Entonces dependiendo de la integral numérica empleada para aproximar
	  %$\int_{t_n}^{t_{n+1}}F(X_{s_n})ds$ se obtiene un método distinto. 
	  
	\begin{exampleblock}{Para el Euler-Mayurama (EM) se considera}
	  \begin{equation*}
	 	\int_{t_n}^{t_{n+1}}F(X_{s_n})ds \approx F(X_{t_n})h
	  \end{equation*}
	 Así, el esquema EM para aproximar :
	 $$X_{t_{n+1}}=X_{t_{n}}+
		  \int_{t_n}^{t_{n+1}}F(X_{s_n})ds
		  +B_{t_{n+1}}-B_{t_n}
	$$
	 	\begin{equation*}
	 		X_{n+1}=X_n+F(X_n)h+\Delta B_n \quad n=0,1\dots N-1
	  \end{equation*}
	\end{exampleblock}
	}
  \end{overlayarea}
\end{frame}
%%%%%%%%%%%%%%%%%%%%%%%%%%%%%%%%%%%%%%%%%%%%%%%%%%%%%%%%%%%%%%%%%%%%%%%%%%%%
\begin{bibunit}[alpha]
\begin{frame}%[label=frm:15]{}
  \frametitle{Construcción del SBD}
  \begin{overlayarea}{\textwidth}{.5\textwidth}
    \only<+>{
      \begin{block}{Proponemos usar el promedio de Steklov para aproximar $F(X_t)$}
	  \begin{align*}
	  			\textcolor{cyan}{F(X_t)\approx \varphi(X_n,X_{n+1})}=&
		        \left(
		        \frac{1}{X_{n+1}-X_{n}}
		        \int_{X_n}^{X_{n+1}} \frac{du}{F(u)}
		        \right)^{-1}\\
		      	t_n\leq & t \leq t_{n+1},\\
		      	X_n=&X_{t_n}, \quad t_n=nh.
	  \end{align*}
	\cite{matus2005exact}    
    \end{block}
	}
	\only<+>{
		\begin{block}{Aproximamos}
			\begin{equation*}
				\int_{t_n}^{t_{n+1}}F(X_{s_n})ds \approx \varphi(X_n,X_{n+1})h
			\end{equation*}
		\end{block}
	}
  
  \end{overlayarea}   
	\biblio{BibliografiaTesis}
\end{frame}
\end{bibunit}
%%%%%%%%%%%%%%%%%%%%%%%%%%%%%%%%%%%%%%%%%%%%%%%%%%%%%%%%%%%%%%%%%%%%%%%%%%%%
\tikzstyle{na} = [baseline=-.5ex]
\tikzstyle{blockYellow}= [rectangle, draw, fill=yellow!40,
    text width=6em, text centered, rounded corners, minimum height=4em]

\tikzstyle{blockGreen}= [rectangle, draw, fill=green!40,
    text width=6em, text centered, rounded corners, minimum height=4em]

\tikzstyle{blockRed}= [rectangle, draw, fill=red!40,
    text width=6em, text centered, rounded corners, minimum height=4em] 
\tikzstyle{every picture}+=[remember picture]
\everymath{\displaystyle}
\begin{frame}
	\frametitle{Método Steklov (SBD)}
% \only<+->{
  \begin{block}{Así obtenemos un método Steklov}
    \begin{equation*}
      X_{n+1}=X_n+
      \tikz[baseline]{
		\node[fill=blue!20,anchor=base] (t1)
		{$\varphi(X_n,X_{n+1}) h$};
      }+
      \tikz[baseline]{
		\node[fill=red!20,anchor=base] (t2)
		{$\Delta B_n$};
      }
    \end{equation*}
  \end{block}
  \begin{overlayarea}{\textwidth}{.5\textheight}
    \begin{columns}
      \column{.5\textwidth}
      \begin{itemize}
	\item <2-> 
	    \tikz[na] \node [blockYellow] (n1) {$\approx\int_{X_n}^{X_{n+1}} \frac{du}{F(u)}$};
	\item<4-> 
	    \tikz[na] \node[blockRed] (n3) {$\varphi=\varphi(X_n)$};
      \end{itemize}
      \column{.5\textwidth}
      \begin{itemize}
	  \item <3-> 
	    \tikz[na] \node[blockGreen] (n2) {$\varphi(X_n,X_{n+1}^*)$};
      \end{itemize}
    \end{columns}	
    \begin{tikzpicture}[overlay]
	\path [->]<2->    (t1) edge[bend right]   (n1);%
	\path [->]<3->    (t1) edge[bend left]    (n2);%
	\path [->]<4->    (t1) edge[bend left]   (n3);%
    \end{tikzpicture}
  \end{overlayarea}
\end{frame}
