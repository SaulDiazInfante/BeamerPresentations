\begin{frame}
  \frametitle{Por que EDEs?}
	\begin{empheq}[box={\Garybox[En ocaciones]}]{align*}
		EDO+ruido=Mejor \text{ } modelo
	\end{empheq}	
	\begin{overlayarea}{\textwidth}{.7\textheight}
		\begin{columns}
    		\column{.5\textwidth}
			\only<2-3>{		
			\begin{exampleblock}{Crecimiento de Poblaciones}
				$$
					\frac{dN}{dt}=a(t)N(t) \qquad N_0=N(0)=cte.
				$$
			\end{exampleblock}	
			}		
			\only<4-8>{	
			\begin{exampleblock}{Circuitos Eléctricos}
			\begin{align*}
				&L\cdot Q''(t)+
				R\cdot Q'(t)+
				\frac{1}{C}\cdot Q(t)
				=F(t)\\
				&Q(0)=Q_0\\
				&Q'(0)=I_0
			\end{align*}				
		\end{exampleblock}	
		}
		\only<7-8>{
		\begin{empheq}[box=\shadowbox*]{equation*}					
			Q(t)=Z(t)+"ruido"		
		\end{empheq}		
		}		
		\column{.5\textwidth}
		\only<3>{
			\begin{empheq}[box=\shadowbox*]{equation*}		
				a(t)=r(t)+"ruido"
			\end{empheq}
		}	
		\only<5-8>{
			\includegraphics[width=\textwidth]{./images/CircuitRLC.png}		
		}		
		\only<6-8>{
		\begin{empheq}[box=\shadowbox*]{equation*}					
			F(t)=G(t)+"ruido"	
		\end{empheq}		
		}
		\only<8>{
			Estima $Q(t)$ observando $Z(t)$		
		}
	\end{columns}    
	\end{overlayarea}
\end{frame}
%%%%%%%%%%%%%%%%%%%%%%%%%%%%%%%%%%%%%%%%%%%%%%%
\begin{frame}
	\frametitle{Por que hacer métodos numéricos para EDEs?}
	\begin{empheq}[box={\Garybox[En ocaciones]}]{align*}
		EDO+ruido=Mejor \text{ } modelo
	\end{empheq}	
	\begin{overlayarea}{\textwidth}{.7\textheight}
		\begin{columns}
    		\column{.5\textwidth}	
				\begin{alertblock}{Solución analítica?}
					muy RARA
				\end{alertblock}			
			\column{.5\textwidth}
				\begin{block}{Usa }
					Teoría de diferencias finitas y haz una extención estocástica.
				\end{block}		
		\end{columns}    
	\end{overlayarea}
\end{frame}
%%%%%%%%%%%%%%%%%%%%%%%%%%%%%%%%%%%%%%%%%%%%%%%%
\begin{frame}
	\frametitle{Objetivo}
	\begin{alertblock}{Objetivo de la charla}
		\emph{Ilustrar} como aproximar soluciones de EDEs  a partir de \emph{conocimientos básicos} 
		de los \emph{métodos deterministas} y nociones muy elementales de variables aleatorias.
	\end{alertblock}
\end{frame}
