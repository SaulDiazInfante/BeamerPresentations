\definecolor{DarkSlateGrey}{HTML}{001B0C}
\definecolor{LightSteelBlue}{HTML}{B3D7F6}
\definecolor{DarkGray}{HTML}{394F50}
\definecolor{LightGoldenrodYellow}{HTML}{F8FDCB}
\setbeamercolor{color titulo caja}{fg=DarkSlateGrey,bg=LightSteelBlue} %%
\setbeamercolor{color cuerpo caja}{fg=DarkGray,bg=LightGoldenrodYellow}%
\begin{frame}%[label=ReglasDeFormulacionNS]{Hyperlinks}
    \frametitle{Reglas de la formulaci\'on NE}
    \begin{columns}
    \column{.5\textwidth}
      \begin{beamerboxesrounded}
      [upper=color titulo caja, lower=color cuerpo caja, shadow=true]
      {Reglas de formulaci\'on}
        \begin{itemize}[<+->]
            \item
              \emph{Ordenes} iguales
            \item
              \emph{Denominador} con m\'as estructura.
            \item
              \emph{T\'erminos no lineales} de forma no local.
            \item
              Reproducir \emph{soluciones especiales}.
            \item
              Evitar \emph{soluciones espurias}.
          \end{itemize}
      \end{beamerboxesrounded}
    \column{.5\textwidth}
     \begin{exampleblock}{Ecuaci\'on log\'istica}
      \begin{align*}
        \frac{dy}{dt}=&\lambda_1 y-\lambda_2 y^2, \\
         y(t_0)=&y_0
      \end{align*}
     \end{exampleblock}
     \begin{exampleblock}{MDFNE}
      $$%\begin{equation*}
          \frac{y_{k+1}-y_k}{
                  \textcolor{blue}{\left(
                  \frac{\exp(\lambda_1 h)-1}{\lambda_1}
                \right)}
              }=
              \lambda_1 y_k-\lambda_2 \textcolor{red}{ y_{k+1}y_{k}} .
        $$%\end{equation*}
      \end{exampleblock}
    \end{columns}
\end{frame}
%%%%%%%%%%%%%%%%
\begin{bibunit}[aalpha]
  \begin{frame}
    \frametitle{MDFNE}
    \begin{Definicion}[Esquema en diferencias finitas no est\'andar]
    Dada $\displaystyle \frac{dy}{dt} = f(y,\lambda)$,
    $$\displaystyle y_{k+1}=F(h,\lambda,y_0),$$
    es  un esquema  en diferencias finitas no  est\'andar si cumple  al menos una de las siguientes:
    \begin{enumerate}[(i)]
    \item
        En las derivadas discretas que aparecen en la formulaci\'on del m\'etodo, el denominador $h$ se
        remplaza  por una funci\'on no negativa $\phi$ tal que
      $
        \phi(h)\to h+\mathcal{O}(h^2) \text{ conforme } h\to 0.
      $
      \item
        Los t\'erminos no lineales que aparecen en $f(y)$  se trabajan de forma no local sobre la malla.
      \end{enumerate}
      \cite{ibijola2008nonstandard}.
    \end{Definicion}
  \biblio{BibliografiaTesis}
  \end{frame}
\end{bibunit}
%%%%%%%%%%%%%%%%%%%%%%%%%%%%%%%%%%%%%%%%%%%%%%%%%%%%%
\begin{bibunit}[aalpha]
  \begin{frame}%[label= dfn:ConsistenciaDinamica]
    \frametitle{Consistencia din\'amica}
    \begin{columns}
      \column{.5\textwidth}
        \begin{Definicion}[Consistencia din\'amica]
            $$%\begin{equation}\label{eqn:ConsistenciaDinamicaODE}
              \frac{dy}{dt}=f(y,t,\lambda),
            $$%\end{equation}
            $$%\begin{equation}\label{eqn:ConsistenciaDinamicaDE}
                y_{k+1}=F(y_k,t_k,h,\lambda).
            $$%\end{equation}
            Si la ecuaci\'on diferencial o inclusive sus soluciones tienen la
            propiedad $P$, el modelo discreto mencionado, es \emph{consistente} con la  din\'amica de
            si este o inclusive sus soluciones reproducen correctamente la
            propiedad  $P$.
          \end{Definicion}
      \column{.5\textwidth}
        \begin{beamerboxesrounded}
        [upper=color titulo caja, lower=color cuerpo caja, shadow=true]
        {Son particularmente  importantes   :}
          \begin{itemize}
            \item positividad
            \item acotamiento
            \item monoton\'ia
            \item puntos fijos y su correspondiente estabilidad
            \item variables dependientes de valor entero
          \end{itemize}
        \cite{Mickens}
      \end{beamerboxesrounded}
    \end{columns}
  \biblio{BibliografiaTesis}
  \end{frame}
\end{bibunit}