\begin{frame}{Conclusiones}
  \begin{alertblock}{}
  \begin{itemize}
		\item<+>	
			los esquemas construidos bajo las reglas de discretización no estándar
			con la alternativa 1 o 2, tienen mejores propiedades de
			estabilidad que el método EM correspondiente.
			
		\item<+>
			En base a los resultados y en comentarios de expertos en el
			tema de dinámica Browniana, podemos concluir que sí tiene sentido aplicar
			la formulación no estándar en la simulación DB.		
	\end{itemize}
\end{alertblock}
\end{frame}
%%%%%%%%%%%%%%%%%%%%%%%%%%%%%%%%%%%%%%%%%%%%%%%%%%%%%%%
\begin{bibunit}[aalpha]
\begin{frame}{Perspectivas}
  \begin{block}{}
  \begin{itemize}
		 \begin{overlayarea}{\textwidth}{.4\textheight}		
		\only<+>{		
		\item	
			Considerar,  $f$ en su serie de Taylor y truncar a cierto orden. Otra tiene
			que ver con una generalización de esta serie, a la cual se conoce como expansión de
			Burman-Lagrange \cite{Silvia011}.
		}
		\only<+>{
		\item
			Calculo de cantidades est\'aticas, tales como factor de estructura $s(q)$ y funci\'on de
			distribución radial $g(r)$.Incorporar más partículas, al menos
			otra dimensión espacial y otros potenciales, es de particular inter\'es
			el potencial de tipo Yukawa, \cite{herrera2010ordering}.
			}		
		\only<+>{		
		\item
			Aplicar la formulaci\'on
			no est\'andar a la soluci\'on num\'erica de EDEs en derivados parciales.
		}
		\only<+>{		
		\item
			Respecto a la simulación de din\'amica Browniana, otrra de las alternativas para describir
			la dinámica del coloide se logra con la ecuacion de Fokker-Planck.
		}
		\only<+>{		
		\item
			Bajo este mismo sentido de equivalencia tenemos inter\'es por modelar el transporte
			en medios porosos y así obtener las ecuaciones de transporte del tipo Langevin, ver
			\cite{balescu2007v}.
		}		
		\end{overlayarea}		
	\end{itemize}
  \end{block}
\biblio{BibliografiaTesis}
\end{frame}
\end{bibunit}
%%%%%%%%%%%%%%%%%%%%%%%%%%%%%%%%%%%%%%%%%%%%

\begin{frame}{Respuesta a Hip\'otesis}
	\begin{exampleblock}{}
		Al aplicar la formulaci\'on de los esquemas en diferencias no est\'andar para resolver la ecuación de 
		Langevin se obtiene un esquema numérico con mejores propiedades de estabilidad y de
		simplicidad algebraica similar al esquema convencional de dinámica Browniana (CBD).	
	\end{exampleblock}
\end{frame}

%%%%%%%%%%%%%%%%%%%%%%%%%%%%%%%%%%%%5
%\begin{bibunit}[aalpha]
\begin{frame}{Preguntas}
	\begin{exampleblock}{}
			Preguntas
	\end{exampleblock}
%\biblio{BibliografiaTesis}
\end{frame}
%\end{bibunit}

