\tikzstyle{na} = [baseline=-.5ex]
\tikzstyle{every picture}+=[remember picture]
\everymath{\displaystyle}
\subsection{Ecuaciones de Langevin}
  \begin{frame}{ Formulación de Langevin:}
\begin{columns}
  \column{.5\textwidth}
  \begin{itemize}
      \item  $x=x(t)$:  posici\'on a tiempo $t$.
      \item <2-> Fuerza de fricci\'on,  dónde $\gamma=6\pi\eta a$,   $\eta$ es la viscosidad laminar del
        solvente  y $a$ radio  de la partícula.
        \tikz[na] \node [coordinate] (n1) {};
    \end{itemize}
    \begin{alertblock}{Ecuaciones de Movimiento}
      \begin{equation*}
        m\frac{d^2x}{dt^2}=
        \tikz[baseline]{
          \node[fill=blue!20,anchor=base] (t1)
          {$ -\gamma \frac{dx}{dt}$};
        }+
        \tikz[baseline]{
          \node[fill=green!20,anchor=base] (t2)
          {$\Gamma(t)$};
        }
      \end{equation*}
    \end{alertblock}
%
     \begin{itemize}
        \item <3->$\Gamma(t)$ : efecto estoc\'astico debido a las colisiones .
        \tikz[na] \node [coordinate] (n2) {};
      \end{itemize}
%
    \begin{tikzpicture}[overlay]
      \path[->]<2-> (n1) edge [bend left] (t1);
      \path[->]<3-> (n2) edge [bend right] (t2);
  \end{tikzpicture}
  \column{.5\textwidth}
  \begin{center}
  %\includegraphics[width=\textwidth]{./images/colloid.jpg}
  %\includegraphics[width=\textwidth]{./images/browngranular-foto2p.jpg}
  %\includegraphics[width=.5\textwidth]{./images/browniano.png}
  \only<2>{
    \includegraphics[width=\textwidth]{./images/browngranular-foto1p.jpg}
  }
  \only<3>{
    \includegraphics[width=\textwidth]{./images/browngranular-foto2p.jpg}
  }
  \end{center}
\end{columns}
\end{frame}
%%%%%%%%%%%%%%%%%%%%%%%%%%%%%%%%%%%%%%%%%%%%%%%%%%%%%%%%%%%%%%%%%
   \begin{frame}{ Formulación de Langevin:}
    \begin{equation*}
        m\frac{d^2x}{dt^2}= -\gamma \frac{dx}{dt} + \Gamma(t).
    \end{equation*}
    \begin{exampleblock}{Simplificaci\'on}
    \begin{overlayarea}{\textwidth}{.4\textheight}
         \only<+>{
           En equilibrio termodin\'amico
           $
            \left\langle
              \frac{1}{2}mv^2
            \right\rangle=
              \frac{1}{2}k_BT
          $,
           $T$ sigue una distribuci\'on de Boltzmann.
           %
        %\only<+>{
           \begin{align*}
              \frac{dx}{dt}=&v\\
             m \frac{dv}{dt}=&-\gamma v+\Gamma(t)
           \end{align*}
           }
           \only<+>{
             $\Gamma(t)$ es un proceso tal que
            \begin{itemize}
              \item
                $\langle
                  \Gamma(t)
                  \rangle=0
                $
              \item De incrementos independientes
                $\langle
                  \Gamma(t),\Gamma(t')
                  \rangle=2\gamma k_BT\delta(t-t')
                $
              \item
                $\Gamma(t)=\sqrt{2k_BT\gamma }\tilde{\xi(t)},$
                $\tilde{\xi}(t)$ es un movimiento Browniano de distribuci\'on $N(0,1)$.
            \end{itemize}
             $$m\frac{dv}{dt}=-\gamma v +\sqrt{2k_BT\gamma } \tilde{\xi}(t)$$
           }
           \only<+>{
          Integrando $m\frac{dv}{dt}=-\gamma v +\sqrt{2k_BT\gamma } \tilde{\xi}(t)$
          $$
            v(t)=v(0)e^{-\frac{\gamma}{m}t}+\frac{\sqrt{2k_BT\gamma }}{m}
            \int_{0}^t e^{-\frac{\gamma}{m}(t-t')} \tilde{\xi}(t')dt'
          $$
           Para $t \gg \frac{m}{\gamma}\equiv \tau$
            }
          \only<+>{
           $$
            v(t)\to\frac{\sqrt{2k_BT\gamma }}{m} %\tilde{\xi}(t)
            \int_{\infty}^t e^{-\frac{\gamma}{m}(t-t')} \tilde{\xi}(t')dt'
           $$
             }
           \only<+>{
          Sea $\eta(t,\tau)=\frac{1}{\tau}\int_{\infty}^t e^{-\frac{1}{\tau}(t-t')}dB(t)$ entonces
           $\frac{dx}{dt}=\sqrt{\frac{2k_BT}{\gamma}}\eta(t,\tau)$.
           \\
            Si $\tau\to 0$
              $\langle
                  \eta(t,\tau),\eta(t',\tau)
                  \rangle=
                \frac{1}{2\tau}e^{\frac{1}{\tau}|t-t'|} \xrightarrow[\tau \to0]{}\delta(t-t')
              $
              $$
                  \frac{dx}{dt}=\sqrt{\frac{2k_BT}{\gamma}}\tilde{\xi}(t)
              $$
            }
            \only<+>{
              Considerando que la partícula interacciona en un potencial  $U=U(x)$
              $$
                m\frac{dv}{dt}=-\gamma v -\frac{dU}{dx}+\sqrt{2k_BT\gamma } \tilde{\xi}(t).
              $$
            Para $\gamma$ lo suficientemente grande $v(t)$ llega a un estado estacionario 
            $$\frac{dv}{dt}\to 0.$$
            }
            \only<+>{
            $$v=-\frac{1}{\gamma} F+ \frac{1}{\gamma}\sqrt{2k_BT\gamma } \tilde{\xi}(t).$$
            Por lo tanto, tomando $D=\frac{k_BT}{6\pi\eta a}$, $\xi(t)=\sqrt{2}\tilde{\xi}(t)$
            }
     \end{overlayarea}
   \end{exampleblock}
\end{frame}
% %%%%%%%%%%%%%%%%%%%%%%%%%%%%%%%%%%%%%%%%%%%%%%%%
\begin{bibunit}[aalpha] 
\begin{frame} 
   \begin{alertblock}{\cite{gardiner1985handbook}}
      \begin{equation*}
          \frac{dx}{dt}=\frac{1}{k_BT} D F+D^{\frac{1}{2}}\xi.
        \end{equation*}
    \end{alertblock}
  \begin{itemize}
      \item $x=x(t)$: posici\'on a tiempo $t$.
      \item $k_B,T$: $k_B$  constantes de  Boltzmann, $T$ temperatura,
      \item $F= -\frac{dU}{dx}$:  fuerza de la part\'icula Browniana inmersa en un potencial $U$,
      \item $D=\frac{k_BT}{6\pi\eta a}$: coeficiente de difusi\'on,
      \item $\xi$ : ruido blanco,\\
        $%\begin{align*}
          \langle\xi(t) \rangle=0, \quad
          \langle \xi(t),\xi(t')\rangle=2\delta(t-t').
        $%\end{align*}
   \end{itemize}
  \biblio{BibliografiaTesis} 
\end{frame}
\end{bibunit}
%
%%%%%%%%%%%%%%%%%%%%%%%%%%%%%%%%
\subsection{Ecuación de Fokker - Planck}
\begin{frame}
  \frametitle{Formulación de Fokker - Planck}
  \begin{exampleblock}{Ecuación para la distribución de velocidades $W(v,t)$}
   \begin{equation*}\label{eqn:EDPDisttribution}
      \frac{\partial W}{\partial t }=\gamma \frac{\partial(vW)}{\partial v}
      +\gamma \frac{k _BT}{m} \frac{\partial ^2 W}{\partial v^2}
    \end{equation*}
  \end{exampleblock}
  \begin{itemize}
    \item
      $W$ distribución de probabilidad para la velocidad.
    \item
      $k_B:$ Constante de Boltzmann.
    \item
      $T:$ Temperatura.
    \item
      $\gamma:$  coeficiente de fricción como en la formulación anterior.
    \item
      $m:$ masa.
  \end{itemize}
\end{frame}
%%%%%%%%%%%%%%%%%%%%%%%%%%%%
\begin{bibunit}[aalpha] 
  \begin{frame}
    \frametitle{Relación entre ambas formulaciones \cite{Risk88:FokkeEquatMethoSolutAppli}.}
    \begin{alertblock}{Propósitos}
      \begin{itemize}
        \item
          Al resolver la ecuación de Langevin se describe el movimiento  de las partículas.
        \item
          Al resolver la ecuación de Fokker - Planck se obtiene una función de
          distribución $W(v,t)$ para describir el plano fase de $N$ partículas .
      \end{itemize}
      \end{alertblock}
    \biblio{BibliografiaPresentacion} 
  \end{frame}
\end{bibunit}
%%%%%%%%%%%%%%%%%%%%%%%%%%%%%%%%%%%%%%
\begin{bibunit}[aalpha]
  \begin{frame}
    \frametitle{Resultado importante}
    \begin{exampleblock}{Ambas formulaciones resultan equivalentes}
      en la descripción estadística del espacio fase para $N$ partículas Brownianas.
      \cite{ErmakMcCammon}
    \end{exampleblock}
    \biblio{BibliografiaPresentacion} 
  \end{frame}
\end{bibunit}
