%%%%%%%%%%%%%%%%%%%%%%%%%%%%%%%%%%%%%%%%%%%%%%%%%%%%%%%%%%%%%%%%%%%%%%%%%%%%%%%
%\begin{bibunit}[apalike]
\begin{frame}%[label=frm:20]
    \frametitle{Aproximación de trayectorias}
	\begin{overlayarea}{1.05\textwidth}{1.05\textheight}
	\only<+>{
	\begin{center}
	  \begin{figure}
\includegraphics[width=.9\textwidth,height=.8\textheight]{./Imagenes/SDETestAdditiveNoise/Trayectorias/SDETestAdditiveNoiseDt_025.png}
	  \caption{Trayectorias generadas con $h=\num{.25}$}
	  \end{figure}
	\end{center}
		}
		\only<+>{
		\begin{figure}

\includegraphics[width=.9\textwidth,height=.8\textheight]{./Imagenes/SDETestAdditiveNoise/Trayectorias/SDETestAdditiveNoiseDt_1.png}
			\caption{Trayectorias generadas con $h=\num{1.0}$}
		\end{figure}
		}
		\only<+>{
		\begin{figure}

\includegraphics[width=.9\textwidth,height=.8\textheight]{./Imagenes/SDETestAdditiveNoise/Trayectorias/SDETestAdditiveNoiseDt_2.png}
			\caption{Trayectorias generadas con $h=\num{2.0}$}
		\end{figure}
		}
	\end{overlayarea}
%\biblio{BibliografiaTesis}
\end{frame}
%\end{bibunit}
%%%%%%%%%%%%%%%%%%%%%%%%%%%%%%%%%%%%%%%%%%%%%%%%%%%%%%%%%%%%%%%%%%%%%%%%%%%%%%%%
%%%%%%%%%
%\begin{bibunit}[apalike]
\begin{frame}%[label=frm:20]
  \frametitle{Aproximación de momentos}
    Calculando \num{1000} trayectorias se genera la siguiente tabla.
    \begin{overlayarea}{\textwidth}{\textheight}
      \only<2>{
    \begin{table}
     \rowcolors{2}{RoyalBlue!5}{RoyalBlue!20}
      \begin{tabular}{l|l|l} \hline
	\multicolumn{3}{c}{$\varepsilon_{debil}=
	\left\|\mathbb{E}\left[|X_{t_n}|\right]-\mathbb{E}[|X_n|]\right\|_2$}
	\\
	\hline
	  $h$		&Euler-Mayurama	& Steklov\\
	  \hline
	  \num{.25}	&\num{1.3878}		&\num{0.2370} \\
	  \num{.5}	&\num{2.1409}		&\num{0.2851} \\
	  \num{1}	&\num{3.9688}		&\num{0.2229} \\
	  \num{2}	&\alert{40.4466}	&\num{0.1439}
	  \\
	  \hline
	\end{tabular}
	\caption{Error en sentido débil para el primer momento.}
      \end{table}
    }
    \only<3>{
      \begin{table}
	\rowcolors{2}{RoyalBlue!5}{RoyalBlue!20}
	 \begin{tabular}{l|l|l} \hline
		\multicolumn{3}{c}{$\varepsilon_{debil}=
	\left\|\mathbb{E}\left[|X_{t_n}|^2\right]-\mathbb{E}[|X_n|^2]\right\|_2$}
		\\
		\hline
	  $h$		&CBD	& SBD\\
	 \hline
	  \num{.25}	&\num{11.4890}			&\num{4.2098} \\
	  \num{.5}	&\num{15.1000}			&\num{1.7700} \\
	  \num{1}	&\num{13.5000}			&\num{0.9760} \\
	  \num{2}	&\alert{\num{468.0000}}	&\num{4.1900}
	  \\
	  \hline
	\end{tabular}
	    \caption{Error en sentido débil para el segundo momento.}
	\end{table}
	}
  \end{overlayarea}
\end{frame}
%\end{bibunit}
% % % % % % % % % % % % % % % % % % % % % % % % % % % % % % % % % % % % % % % % %
\begin{bibunit}[apalike]
\begin{frame}%[label=frm:20]
	\frametitle{Ecuación Logística \cite{Schurz2007}}
%
\vspace{-0.5 cm}
	\begin{empheq}[box=\shadowbox*]{align*}
		dX_t& = \lambda X_t(K-X_t)dt+ \sigma X_t^\alpha|K-X_t|^\beta dB_t\\
		X_0& = 50, 
		K = 1000, 
		\alpha = 1,
		\beta = \num{0.5}, 
		\lambda = 0.25, 
		\rho = 0,
		\sigma = \num{0.05}
	\end{empheq}
%
	\begin{overlayarea}{1.0\textwidth}{.55\textheight}
		\vspace{-.8cm}
		\only<2->{
			\begin{center}
				\includegraphics[scale=.5]{images/LogisticSDE.eps}
			\end{center}
		}
	\end{overlayarea}
	\vspace{-.9cm}
	\biblio{BibliografiaTesis}
\end{frame}
\end{bibunit}
% % % % % % % % % % % % % % % % % % % % % % % % % % % % % % % %
\begin{bibunit}[apalike]
	\begin{frame}%[label=frm:20]
		\frametitle{Brownina Dynamics \cite{Braanka1998}}
		%
		\vspace{-0.3 cm}
		\begin{empheq}[box=\shadowbox*]{equation*}
				dX_t= -X_t^3 +\xi dB_t,
			\qquad Dta=\num{1e-6}
		\end{empheq}
		%
		\begin{overlayarea}{1.0\textwidth}{.55\textheight}
			\only<2->{
				\begin{center}
					\includegraphics[scale=.5]{images/short-longMSD.eps}
				\end{center}
			}
		\end{overlayarea}
		\biblio{BibliografiaTesis}
	\end{frame}
\end{bibunit}
