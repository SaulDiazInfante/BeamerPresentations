\begin{bibunit}[apalike]
\begin{frame}%[label=frm:20]
    \frametitle{Estabilidad en Media Cuadrática \cite{saito1996stability}}
	\biblio{BibliografiaTesis}
\end{frame}
\end{bibunit}
%%%%%%%%%%%%%%%%%%%%%%%%%%%%%%%%%%%%%%%%%%%%%%%%%%%%%%%%%%%%%%%%%%%
%\begin{bibunit}[apalike]
\begin{frame}%[label=frm:20]
	\frametitle{Estabilidad en Media Cuadrática}
	\begin{overlayarea}{\textwidth}{.25\textheight}	
		\only<1-4>{
			\begin{empheq}[box=\shadowbox*]{equation*}
				dX_t=\lambda X_t dt +\beta dB_t, \quad X_0=cte, \lambda, \beta \in
				\mathbb{C} \qquad \text{(DP)}
			\end{empheq}
		}
		\only<5>{
		 	\begin{empheq}[box=\shadowbox*]{equation*}
				\lim_{t \to \infty}\mathbb{E}[|X_t|^2]=-\frac{|\beta|^2}{2Re(\lambda)}.
			\end{empheq}	
		}	
	\end{overlayarea}
	\begin{columns}	
		\column{.5\textwidth}		
		\begin{overlayarea}{\textwidth}{\textheight}				
			\centering{Euler-Mayurama}		
			\only<1->{				
				\begin{empheq}[box={\ovalbox}]{align*}
						X_{n+1} = &(1+\lambda h) X_n +\beta \Delta B_n 
				\end{empheq}							
			}
%			\only<2-5>{
%				\begin{empheq}{align*}		
%					\mathbb{E}(|X_{n+1}|^{2})=&
%						|1+\lambda h|^{2}\mathbb{E}(|X_{n}|^{2})+|\beta|^{2}h
%				\end{empheq}
%			}
%			\only<3-4>{
%				\scalebox{0.9}{% Scale by 90%									
%					$\displaystyle
%						=
%						|1+\lambda h|^{2}(|1+\lambda h|^{2}\mathbb{E}(|X_{n-1}|	^{2})+|\beta|^{2}h)
%						+|\beta |^{2}h
%					$
%				}					
%			}
%		\only<4>{					
%				\scalebox{0.9}{% Scale by 90%									
%					$\displaystyle 
%						=
%						|1+\lambda h|^{4}\mathbb{E}(|X_{n-1}|^{2})
%						+
%						\left[|1+\lambda h|^{2}+1\right]|\beta|^{2}h\				
%					$						
%				}
%				$$
%						\vdots
%				$$
%		}
%		\only<5-6>{
%			\scalebox{0.9}{% Scale by 90%
%					$ =
%						|1+\lambda h|^{2(n+1)}\mathbb{E}(|X_{0}|^{2})+
%						\underbrace{						
%							\left[|1+\lambda h|^{2n}+\cdots+|1+\lambda h|^{2}+1\right]
%							}_{\textit{Serie geométrica}}
%							|\beta|^{2}h
%					$
%			}
%		}
%		\only<6>{
%			\scalebox{0.9}{% Scale by 90%
%					 $	=
%					 	|1+\lambda h|^{2(n+1)}
%					 	\mathbb{E}(|X_{0}|^{2})+
%					 	\frac{|1+\lambda h|^{2(n+1)}-1}{|1+\lambda h|^{2}-1}
%					 	|\beta|^{2}h
%					$	
%			}
%			\scalebox{0.9}{% Scale by 90%
%					$					 
%					  =
%					 	|1+\lambda h|^{2(n+1)}\mathbb{E}(|X_{0}|^{2})+
%					 	\frac{|1+\lambda h|^{2(n+1)}-1}{2Re(\lambda)+|\lambda|^{2}h}|\beta|^{2}.			
%					$
%			}		
%		}
%		\only<7-12>{
%			\begin{empheq}{align*}
%					\mathbb{E}(|X_{n+1}|^{2}) & = |1+\lambda h|^{2(n+1)}\mathbb{E}(|X_{0}|^{2})\\
%						&+
%							\frac{|1+\lambda h|^{2(n+1)}-1}{2Re(\lambda)+|\lambda|^{2}h}|\beta|^{2}\\
%			\end{empheq}
%		}
		\only<2->{
			Si $Re(\lambda h)<0$			
			\begin{align*}
  					\mathbb{E}[|X_{n+1}|^2]
  					\xrightarrow{n\to\infty}  			
  					\colorbox{darkyellow}{
  				 		$\frac{-|\beta|^2 }{2Re(\lambda)+|\lambda|^2h}$
  					}
  				\end{align*}
		}
		\end{overlayarea}
%--------------------------------------------------------------------------------		
		\column{.6\textwidth}		
		\begin{overlayarea}{\textwidth}{\textheight}	
			\only<3->{
				\centering{\alert<5->{\textbf{Steklov}}}					
					\begin{empheq}[box=\mybluebox]{align*}
						X_{n+1} = & X_n \exp(\lambda h)+\beta \Delta B_n
					\end{empheq}
			}						
%			\only<+>{	
%				\begin{equation*}
% 					\mathbb{E}[|X_{n+1}|^2]=\exp(2Re(\lambda)h)\mathbb{E}[|X_n|^2]+|\beta|^2 h.
%				\end{equation*}	
%			}
%			\only<10>{	
%					\begin{align*}
%						\mathbb{E}[|X_{n+1}|^2]&=\exp(2Re(\lambda)h)\mathbb{E}[|X_n|^2]+|\beta|^2 h\\
%  						\vdots &=\vdots\\
%								&= \exp(2Re(n+1)\lambda h)\mathbb{E}[|X_{0}|]\\
%								&+|\beta|^2h 
%	    				\underbrace{
%	      					\left(
%	      						1+\cdots+\exp(2nRe(\lambda) h)
%	     						\right)
%	    						}_{\text{Serie geométrica}}\\
%  					\end{align*}	
%			}
%			\only<11>{
%				\begin{align*}
%					\mathbb{E}[|X_{n+1}|^2]&=\exp(2(n+1)Re(\lambda)h)\mathbb{E}[|X_0|^2]\\
%  							+&\beta^2 h \frac{(\exp(2Re(\lambda)h))^{n+1}-1}{\exp(2Re(\lambda)h)-1}  		
%  				\end{align*}
%			}	
			\only<4->{
				Si $Re(\lambda)<0$
				\begin{align*}
  					\mathbb{E}[|X_{n+1}|^2]
  					\xrightarrow{n\to\infty}
  					\colorbox{hellcyan}{
  				 		$\frac{-|\beta|^2 h}{\exp(2Re(\lambda)h)-1}$
  					}
  				\end{align*}		
			}
		\end{overlayarea}
	\end{columns}
	%\biblio{BibliografiaTesis}
\end{frame}
%\end{bibunit}
%%%%%%%%%%%%%%%%%%%%%%%%%%%%%%%%%%%%%%%%%%%%%%%%%%%%%%%%%%%%%%%%%%%%%%%%%%%%%%%%%%%%%%%%%%%%%%%%%
\begin{frame}
  \frametitle{Estabilidad en Media Cuadrática}
  \begin{empheq}{equation}
 		dX_t=\lambda X_t dt +\beta dB_t, \qquad X_0=cte., \lambda, \beta \in \mathbb{C} \tag{E}.
  \end{empheq}
   \begin{columns}
	  \column{.5\textwidth}
		\begin{definicion}[Consistencia lineal en MC]
	 		Un esquema numérico para  la ecuación (E) se dice asintóticamente consistente en media
	 		cuadrática si la solución numérica satisface
 		$$
 		  \lim_{\substack{ n\to \infty\\ h\to 0}} X_n= \frac{-\beta}{2Re(\lambda)}
 		$$
 	  \end{definicion}
	  \column{.5\textwidth}
 	  \begin{Teorema}
 		 El esquema Steklov para la ecuación (E), es asintóticamente consistente en MC.
 	  \end{Teorema}
  \end{columns}
\end{frame}
%%%%%%%%%%%%%%%%%%%%%%%%%%%%%%%%%%%%%%%%%%%%%%%%%%%%%%%%%%%%%%%%%%%%%%%%%%%%%%%%%%%%%%%%%%%%%%%%%
\begin{bibunit}[alpha]
\begin{frame}
	\frametitle{Stabilidad lineal por trayectorias}
	\begin{empheq}{equation*}
		dX_t=\lambda X_t dt +\beta dB_t, \qquad X_0=cte., \lambda, \beta \in \mathbb{R} \tag{E}.
	\end{empheq}
	\nocite{Buckwar2011a}
	\begin{columns}
		\column{.5\textwidth}
			\begin{overlayarea}{\textwidth}{.5\textheight}
				\centering Pullback attractor
				\only<2->{
					\begin{empheq}[box=\shadowbox*]{equation*}
						\lim_{t_0\to-\infty} X_t =\widehat{O}_t %
						:= e^{\lambda t}\int\limits_{-\infty}^{t}e^{-\lambda s}dB_s, 
					\end{empheq}
				}
		\end{overlayarea}
		\column{.5\textwidth}
			\begin{overlayarea}{\textwidth}{.5\textheight}
			\vspace{-1.0cm}
			\only<3->{
				\begin{Teorema}
					Sea $\lambda<0$,  el método Steklov para (E)
					tiene el siguiente atractor
					\begin{equation*}
						\widehat{O}_n^{(h)}  :=
						\xi \sum_{j=-\infty}^{n-1}\exp(\lambda h(n-1-j)) \Delta B_j,
					\end{equation*}
					$\widehat{O}_n^{(h)} \to \widehat{O}_t$, \quad $h\to 0$, \quad pathwise.
				\end{Teorema}
			}
			\end{overlayarea}
	\end{columns}
\biblio{BibliografiaTesis}
\end{frame}
\end{bibunit}
%%%%%%%%%%%%%%%%%%%%%%%%%%%%%%%%%%%%%%%%%%%%%%%%%%%%%%%%%%%%%%%%%%%%%%%%%%%%%%%%%%%%%%%%%%%%%%%%%
\begin{frame}
	\frametitle{Estabilidad en Media Cuadrática Ruido Multiplicativo}
	\begin{empheq}[box=\shadowbox*]{equation*}
		dX_t=\lambda X_t dt +\xi X_t dB_t, \qquad X_0=cte., \lambda, \xi \in \mathbb{R} \tag{E}.
	\end{empheq}
	\begin{columns}
		\column{.5\textwidth}
		\only<2->{
			\begin{exampleblock}{MS-estabilidad Lineal}
				\begin{itemize}
					\item
						diagonal (EM)
					\item
						vertical (Steklov)
				\end{itemize}
			\end{exampleblock}
		}
		\column{.5\textwidth}
			\centering
			\only<3->{
				\includegraphics[width=1\linewidth]{images/StabilityPlotMultiplicativeNoise}
			}
	\end{columns}
\end{frame}



