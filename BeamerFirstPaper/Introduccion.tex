%%%%%%%%%%%%%%%%%%%%%%%%%%%%%%%%%%%%%%%%%%%%%%%%%%%%%%
\begin{frame}%[label=frm:1]{Hyperlinks}
    \frametitle{Ejemplo de Coloides}
	\begin{center}
		\movie[width=9.1cm,height=6.5cm,showcontrols=true,loop,poster]{}{Animacion16p.mpg}
	\end{center}
 \end{frame}
%---------------------------------------------------------------------------------------------------
%%%%%%%%%%%%%%%%%%%%%%%%%%%%%%%%%%%%%%%%%%%%%%%%%%%%%%%
\begin{frame}[label=frm:2]{}
    \frametitle{Suspensiones Coloidales}
    \centering
    \only<1>{
    \begin{center}
       \includegraphics[scale=.20,keepaspectratio=true]{./Imagenes/Introduccion/colloid.jpg}
	  % colloid.jpg: 826x898 pixel, 72dpi, 29.14x31.68 cm, bb=0 0 826 898
        %\includegraphics[width=\textwidth, height=\textheight ]{./Imagenes/Introduccion/colloid.jpg}
    \end{center}
    }
    \only<2>{
    \begin{center}
		\includegraphics[width=.7\textwidth]{./Imagenes/Introduccion/browngranular-foto2p.jpg}
    \end{center}
    }
\end{frame}
%%%%%%%%%%%%%%%%%%%%%%%%%%%%%%%%%%%%%%%%%%%%%%%%%%%%%%%%%%%%%%%%%%%%%%
 \begin{frame}[label=frm:3]{}
  \frametitle{Suspensiones Coloidales}
    \begin{center}
      \includegraphics[width=.7\textwidth]{./Imagenes/Introduccion/browngranular-foto2p.jpg}
    \end{center}
 \end{frame}
%========================================
%Aqui quiero poner la animación con VMD
% 
% 
%=======================================
%%%%%%%%%%%%%%%%%%%%%%%%%%%%%%%%%%%%%%%%%%%%%%%%%%%%%%%%%%%%%%%%%%%%%%
%\definecolor{DarkSlateGrey}{HTML}{001B0C}
%\definecolor{LightSteelBlue}{HTML}{B3D7F6}
%\definecolor{DarkGray}{HTML}{394F50}
%\definecolor{LightGoldenrodYellow}{HTML}{F8FDCB}
%\setbeamercolor{color titulo caja}{fg=DarkSlateGrey,bg=LightSteelBlue} %%
%\setbeamercolor{color cuerpo caja}{fg=DarkGray,bg=LightGoldenrodYellow}%
%\begin{frame}%[label=frm:4]{}
%  \frametitle{Simulación de Dinámica Browniana}
%  \tikzstyle{decision} = [diamond, draw, fill=yellow!20,
%    text width=4.5em, text badly centered, node distance=4cm, inner sep=5pt]
%  \tikzstyle{block} = [rectangle, draw, fill=blue!20,
%    text width=5em, text centered, rounded corners, minimum height=4em]
%  \tikzstyle{blockIm}= [rectangle, draw, fill=red!40,
%    text width=6em, text centered, rounded corners, minimum height=4em]
%  \tikzstyle{line} = [draw, -latex]
%  \tikzstyle{cloud} = [draw, ellipse,fill=red!20, node distance=3cm,
%    minimum height=2em]
%  \begin{center}
%	\begin{tikzpicture}[node distance = 2cm, auto]
%    % Place nodes
%	  \node [block] (Init) {Inicializar};
%	  \node [block, below of=Init] (Fuerza) {Calculo de Fuerza};
%	  \node [decision, left of=Fuerza] (Serie) {Serie de Tiempo};
%	  \node [blockIm, below of=Fuerza] (Posiciones) {Posiciones};
%	  \node [block, left of=Posiciones,node distance=6cm] (medio){Promediar cantidades de interes};
%    % Draw edges
%	  \path [line] (Init) -- (Fuerza);
%	  \path [line] (Fuerza) --(Posiciones);
%	  \path [line] (Posiciones)-|(Serie);
%	  \path [line] (Serie)--(Fuerza);
%	  \path [line] (Serie)-| (Promedio);
%	\end{tikzpicture}
%  \end{center}
%\end{frame}
%%%%%%%%%%%%%%%%%%%%%%%%%%%%%%%%%%%%%%%%%%%%%%%%%%%%%%%%%%%%%%%%%%%%%%%%%%%%%%%%%%%%%
\tikzstyle{na} = [baseline=-.5ex]
\tikzstyle{every picture}+=[remember picture]
\everymath{\displaystyle}
%%%%%%%%%%%%%%%%%%%%%%%%%%%%%%%%%%%%%%%%%%%%%%%%%%
\begin{frame}
  \frametitle{Formulación de Langevin}
  \begin{alertblock}{Ecuaciones de Movimiento}
    \begin{equation*}
      m\frac{d^2x}{dt^2}=
      \tikz[baseline]{
      \node[fill=blue!20,anchor=base] (t1)
      {$ -\gamma \frac{dx}{dt}$};
      }+
      \tikz[baseline]{
      \node[fill=green!20,anchor=base] (t2)
      {$\Gamma(t)$};
      }
    \end{equation*}
  \end{alertblock}
  \begin{columns}
    \column{.4\textwidth}
    \begin{itemize}
	\item <2-> $x=x(t)$:  posici\'on a tiempo $t$.
	\item <3-> Fuerza de fricci\'on,  dónde $\gamma=6\pi\eta a$,   $\eta$
	  es la viscosidad laminar  \tikz[na] \node [coordinate] (n1) {};
	  del solvente  y $a$ el radio  de la partícula.
	\item <4->$\Gamma(t)$ :
	  efecto estoc\'astico  
	  debido a las colisiones. \tikz[na] \node [coordinate] (n2) {};
    \end{itemize}
    \column{.6\textwidth}
  \end{columns} 
    \begin{tikzpicture}[overlay]
      \path [->]<3->  (n1) edge[bend right]  (t1);%
      \path [->]<4->  (n2) edge[bend right]  (t2);%
  \end{tikzpicture}
\end{frame}
%%%%%%%%%%%%%%%%%%%%%%%%%%%%%%%%%%%%%%%%%%%%%%%%%%%%%%%
\begin{bibunit}[apalike] 
 \begin{frame}[label=frm:6]{}
   \begin{alertblock}{Al aplicar eliminación
	adiabática \cite{gardiner1985handbook}}
      \begin{equation*}
          \frac{dx}{dt}=\frac{1}{k_BT} D F+D^{\frac{1}{2}}\xi.
        \end{equation*}
    \end{alertblock}
  \begin{itemize}
      \item $x=x(t)$: posici\'on a tiempo $t$.
      \item $k_B,T$: $k_B$  constantes de  Boltzmann, $T$ temperatura,
      \item $F= -\frac{dU}{dx}$:  fuerza de la part\'icula inmersa en un potencial $U$,
      \item $D=\frac{k_BT}{6\pi\eta a}$: coeficiente de difusi\'on,
      \item $\xi$ : ruido blanco,\\
        $
         	\mathbb{E}(\xi(t)) =0, \quad
        	 \mathbb{E}(\xi(t)\xi(t'))=2\delta(t-t').
        $
   \end{itemize}
  \biblio{BibliografiaTesis}
\end{frame}
\end{bibunit}
%%%%%%%%%%%%%%%%%%%%%%%%%%%%%%%%%%%%%%%%%%%%%%%%%%%%%%%%%%%%%%%%%%%%%%%%%
\begin{bibunit}[apalike]
\begin{frame}\frametitle{Prop\'osito}
    \begin{block}
	{Resolvemos \quad $\frac{dx}{dt}=\frac{1}{k_BT} D F+D^{\frac{1}{2}}\xi.$}
	Para \textcolor{cyan}{entender} los mecanismos de \textcolor{cyan}{difusión}
	en una suspensión coloidal.
	\emph{Sin embargo, en la práctica \textcolor{red}{no se tiene solución analítica}.}
    \end{block}
  %\biblio{BibliografiaTesis}
\end{frame}
\end{bibunit}
%%%%%%%%%%%%%%%%%%%%%%%%%%%%%%%%%%%%%%%%%%%%%%%%%%%%%%%%%%%%%%%%%%%%%%%%
\begin{bibunit}[apalike] 
\begin{frame}
  \frametitle{Método convencional }
   \begin{exampleblock}{Euler-Mayurama}
     \begin{align}
        Y_{j+1}^{(\alpha)}(h)=&
        	Y_{j}^{(\alpha)}+
        	\frac{D}{T}F_{j}^{(\alpha)}\Delta t+
        	R_{j}^{(\alpha)}\\        
      	\mathbb{E}
      		\left[
      			R_{j}^{(\alpha)}
        	\right]
        	=&0 \label{eqn:mediaEMc}\\
        \mathbb{E}
        \left[
        	R_{j}^{(\alpha)} R_{j}^{(\beta)}
		\right]
        	=&
        		2D h \delta_{ij}\delta_{\alpha \beta}
			&\alpha,\beta=x,y,z \label{eqn:CovEMc}
     \end{align}
    \end{exampleblock}
 	\begin{overlayarea}{\textwidth}{.4\textwidth}
	\only<+>{
  	\begin{columns}
		\column{.5\textwidth}
		\begin{itemize}
      		\item $Y_{j}^{(\alpha)}$: posición. 
      		\item $h:$ incremento temporal.
      		\item $F_{j}^{(\alpha)}:$ fuerza neta sobre la partícula $i$ en la dirección $\alpha$.
  		\end{itemize}
    	\column{.5\textwidth}
		\begin{itemize}
	        \item $R_{j}^{(\alpha)}:$ ruido blanco discreto, con  media y covarianza
		  	como en \eqref{eqn:mediaEMc} y \eqref{eqn:CovEMc}.
      		\item $D=\frac{k_B T}{\gamma}$: coeficiente de difusión de Stokes - Einstein
    	\end{itemize}
   	\end{columns}
	}
	\only<+>{
		\begin{columns}
		\column{.5\textwidth}
			\begin{itemize}
		  		\item Es fácil de implementar.
		  	\end{itemize}
			\column{.5\textwidth}
			\begin{itemize}
			    \item Trabaja con un tamaño de \textcolor{red}{paso restrictivo}.
		  	\end{itemize}
	   	\end{columns}
	}
	\only<+->{
	\begin{exampleblock}{}
		Existen varios esquemas para discretizar la ecuación ya mencionada \cite{branka1999blgorithms}.
  		Sin embargo, no representan una mejora significativa a la precisión respecto al coste computacional.
	\end{exampleblock}
 	 }
	\only<+>{\biblio{BibliografiaTesis}}
	\end{overlayarea}
\end{frame}
\end{bibunit}
%%%%%%%%%%%%%%%%%%%%%%%%%%%%%%%%%%%%%%%%%%%%%%%%%%%%%%%%%%%%%%%%%%%%%%%%%%%%%%%%%%%%%%%%%%%%%%%%%%
\begin{bibunit}[alpha]
  \begin{frame}
    \frametitle{Nuestra idea}
	\hypertarget{Idea}{}    
    En \cite{matus2005exact}, usando el \hyperlink{dfn:Steklov}{\textbf{promedio de Steklkov}}, se 	logra un esquema en 	diferencias   exacto para resolver EO no lineales de la forma
    \begin{equation*}
    	\frac{dx}{dt}=f_1(x)f_2(t)
    \end{equation*}
  \biblio{BibliografiaTesis}
  \end{frame}
\end{bibunit}
%%%%%%%%%%%%%%%%%%%%%%%%%%%%%%%%%%%%%%%%%%%%%%%%%%%%%%%%%%%%%%%%%%%%%%%%%%%%%%%%%%%%%%%%%%%%%%%%%%%%%%%%%%%%%%%%
\begin{bibunit}[apalike] 
\begin{frame}
	\frametitle{Ejemplo EDE multiplicativa}
	\hypertarget<2>{ex:EDEMult}{}
		\vspace{-.75cm}
	\begin{empheq}[box={\Garybox[$dX_t=\frac{1}{X_t} +X_tdB_t$, \quad $X_0=2$ ]}]{align*}
		Y_{j+1}=&\sqrt{
				\textcolor{red}{\varphi(h)}+Y_j^2
			}+Y_j\Delta B_j &
		\hyperlink{dfn:Steklov}{ \textcolor{red}{\varphi(h)=2h}
			\quad h=\num{0.5}
			}
    \end{empheq}	
	\begin{overlayarea}{1.05\textwidth}{.75\textheight}    	
    	\only<2->{   	
			\vspace{-.5cm}
			\begin{center}			
				\includegraphics[width=\textwidth]{./images/InverseMultiplicative.png}
			\end{center}		
		}
	\end{overlayarea}
\end{frame}
\end{bibunit}
%%%%%%%%%%%%%%%%%%%%%%%%%%%%%%%%%%%%%%%%%%%%%%%%%%%%%%%%%%%%%%%%%%%%%%%%%%%%%%%%%%%%%%%%%%%%%%%%%%
\begin{frame}{Ejemplo EDE aditiva}
	\begin{columns}
		\column{.45\textwidth}
			\begin{exampleblock}{$dX_t=-\lambda X_t dt+\mu \exp(-t)dB_t$}
				\begin{align*}
          X_{j+1}=&X_j(1-\lambda \textcolor{blue}{\varphi(h)})\\
          				&+\mu\exp(-h)\Delta B_j\\
         \textcolor{blue}{\varphi(h)}=& \frac{1-\exp(-\lambda h)}{\lambda}
        \end{align*}
     \end{exampleblock}
     \column{.6\textwidth}
      \begin{alertblock}{Comparaci\'on con EM, $h=\num{0.5},X_0=\num{10}$}
        \begin{center}
          \includegraphics[width=\textwidth]{./images/linealExp.png}
      \end{center}
      \end{alertblock}
  \end{columns}
%  % \biblio{BibliografiaTesis}
\end{frame}
%%%%%%%%%%%%%%%%%%%%%%%%%%%%%%%%%%%%%%%%%%%%%%
\begin{bibunit}[apalike]
\begin{frame}%[label=frm:11]
    \frametitle{Objetivo}
	%Resulta que muchos de los potenciales de interés se pueden expresar como
	%$$
	%  U(X)=f_1(X)f_2(t).
	%$$
  \begin{overlayarea}{\textwidth}{.5\textwidth}
  \only<+>{
  \begin{alertblock}{Objetivo de esta charla}
	\begin{itemize}
	    \item Mostrar \textcolor{red}{c\'omo adaptamos} el promedio de 
	    \textcolor{red}{Steklov} para aproximar
	      $$\frac{dx}{dt}=\frac{1}{k_BT} D F+D^{\frac{1}{2}}\xi.$$
	    \item Exhibir la \textcolor{red}{estabilidad lineal} de los esquemas propuestos.
		\item Hablar de las \textcolor{red}{propiedades teóricas} que buscamos estudiar.
	\end{itemize}
  \end{alertblock}
  }
  \only<+>{
  \begin{alertblock}{Con ello buscamos}
	\begin{itemize}
		\item
			Desarrollar \textcolor{blue}{bases teóricas} para los esquemas Steklov.
		\item
			Obtener esquemas que \textcolor{red}{mejoren el tamaño de paso} del método 
			Euler- Mayurama con un  coste computacional similar.
		\item
			Extension a sistemas y condiciones locales.
	\end{itemize}
	\end{alertblock}
	}
  \end{overlayarea}
  %\biblio{BibliografiaTesis}
\end{frame}
\end{bibunit}
%%%%%%%%%%%%%%%%%%%%%%%%%%%%%%%%%%%%%%%%%%%%%%%%%%%%%%%%%%%%%%%%%%%%%%%%%%%%%%%%%%%%%%%%%%%
%\begin{frame}
%	\frametitle{Plan de Charla}
%    \tableofcontents[pausesections]
%\end{frame}