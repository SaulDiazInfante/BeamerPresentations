\documentclass[spanish,10pt,xcolor=dvipsnames,table]{beamer}
\usepackage[spanish,activeacute]{babel}
\usepackage{color}
\usepackage{xcolor}
\usepackage{colortbl}
\usepackage{amsmath}
\usepackage{amssymb}
\usepackage{graphicx}
\usepackage{latexsym}
\usepackage{ucs}
\usepackage[utf8]{inputenc}
\usepackage{bibunits}
\usepackage[amssymb]{SIunits}
\usepackage{sistyle}
\usepackage{times}
\usepackage{tikz}
\usepackage{verbatim}
\usepackage{hyperref}
\usepackage{url}
\usetheme{Oxygen}
\usepackage{thumbpdf}
\usepackage{wasysym}
\usepackage{pgf,pgfarrows,pgfnodes,pgfautomata,pgfheaps,pgfshade}
\usepackage{verbatim}
\usepackage{multimedia}
\usepackage{empheq}
\usepackage{fancybox}
\usepackage{epstopdf}
\usetikzlibrary{arrows,shapes}
\theoremstyle{plain} % default
\newtheorem{Teorema}{Teorema}
\newtheorem{Ejemplo}{Ejemplo}
\theoremstyle{definition}
\newtheorem{definicion}{Definici\'on}
\newtheorem{Corolario}{Corolario}
\newtheorem{Proposicion}{Proposici\'on}
\newtheorem{lema}{Lema}
\newtheorem{Prueba}{Prueba}
\usepackage{esint}
\usepackage{lipsum}
\def\Q#1#2{\frac{\partial #1}{\partial #2}}
\usepackage{listings}
\lstset{%
    language=[AlLaTeX]TEX,%
    float=hbp,%
    basicstyle=\ttfamily\small, %
    identifierstyle=\color{colIdentifier}, %
    keywordstyle=\color{colKeys}, %
    stringstyle=\color{colString}, %
    commentstyle=\color{colComments}, %
    columns=flexible, %
    tabsize=3, %
    frame=single, %
    extendedchars=true, %
    showspaces=false, %
    showstringspaces=false, %
    numbers=left, %
    numberstyle=\tiny, %
    breaklines=true, %
    backgroundcolor=\color{hellgelb}, %
    breakautoindent=true, %
    captionpos=b,%
    xleftmargin=18pt,%
    xrightmargin=\fboxsep%
}
%--------------------------Fancy boxes-------------------------------------------------------------------
\definecolor{myblue}{rgb}{.8, .8, 1}
\definecolor{shadecolor}{cmyk}{0,0,0.41,0}
\newcommand*\mybluebox[1]{%
	\colorbox{myblue}{\hspace{1em}#1\hspace{1em}}
}
\newcommand*\myyellowbox[1]{%
	\colorbox{darkyellow}{\hspace{1em}#1\hspace{1em}}
}
%--------------------------------------------------------------------------
\definecolor{shadecolor}{cmyk}{0,0,0.41,0}
\definecolor{light-blue}{cmyk}{0.25,0,0,0}
\newsavebox{\mysaveboxM} % M for math
\newsavebox{\mysaveboxT} % T for text
\newcommand*\Garybox[2][Example]{%
	\sbox{\mysaveboxM}{#2}%
		\sbox{\mysaveboxT}{\fcolorbox{black}{light-blue}{#1}}%
			\sbox{\mysaveboxM}{%
	\parbox[b][\ht\mysaveboxM+.5\ht\mysaveboxT+.5\dp\mysaveboxT][b]{%
		\wd\mysaveboxM}{#2}%
	}%
	\sbox{\mysaveboxM}{%
		\fcolorbox{black}{shadecolor}{%
		\makebox[\linewidth-10em]{\usebox{\mysaveboxM}}%
		}%
	}%
	\usebox{\mysaveboxM}%
	\makebox[0pt][r]{%
		\makebox[\wd\mysaveboxM][c]{%
			\raisebox{\ht\mysaveboxM-0.5\ht\mysaveboxT
			+0.5\dp\mysaveboxT-0.5\fboxrule}{\usebox{\mysaveboxT}}%
		}%
	}%
}
\newcommand\Fontvi{\fontsize{7}{7.2}\selectfont}
%%%%%%%%%%%%%%%%%%%%%%%%%%%%%%%%%%%%%%%%%%%%
\definecolor{kugreen}{RGB}{50,93,61}
\definecolor{kugreenlys}{RGB}{132,158,139}
\definecolor{kugreenlyslys}{RGB}{173,190,177}
\definecolor{kugreenlyslyslys}{RGB}{214,223,216}
\definecolor{greenArea}{RGB}{124,252,124}
\definecolor{hellmagenta}{rgb}{1,0.75,0.9}
\definecolor{hellcyan}{rgb}{0.75,1,0.9}
\definecolor{hellgelb}{rgb}{1,1,0.8}
\definecolor{colKeys}{rgb}{0,0,1}
\definecolor{colIdentifier}{rgb}{0,0,0}
\definecolor{colComments}{rgb}{1,0,0}
\definecolor{colString}{rgb}{0,0.5,0}
\definecolor{darkyellow}{rgb}{1,0.9,0}
\setbeamercovered{transparent}
\mode<presentation>
{  
	%\usetheme{PaloAlto}
	%\usecolortheme[named=kugreen]{structure}
	\useinnertheme{progressbar}
	%\usefonttheme{default}
	\usefonttheme{serif}
	%\setbeamercovered{transparent}
	\setbeamertemplate{blocks}[rounded][shadow=true]
	%s\setbeamertemplate{navigation symbols}[only frame symbol]
}
\setbeamertemplate{background}{
\parbox[c][\paperheight]{\paperwidth}
    {
    \vfill \hfill
    \begin{tikzpicture}
      \node[opacity=.1]
      {
       \includegraphics[width=.5\textwidth]{./Imagenes/Logo/CimatLogo.png}
      };
    \end{tikzpicture}
      \vspace{.5cm} %\hspace{.5cm}
      }
  }
\logo{\includegraphics[width=1.0cm]{./Imagenes/Logo/Logo.png}}
\title{Métodos Steklov para EDEs }
%\subtitle{()}
\author[]{Sa\'ul D\'iaz Infante \and Asesor: Dra.Silvia Jerez Galiano}
\institute{CIMAT A.C.}
\date\today
\AtBeginSection[]
{
  \begin{frame}<beamer>{Métodos Steklov (SBD)}
    \tableofcontents[currentsection,currentsubsection]
  \end{frame}
}
\begin{document}
  \frame{\titlepage \vspace{-0.5cm}}
 \section*{Introducci\'on}
 	\begin{frame}
  \frametitle{Por que EDEs?}
	\begin{empheq}[box={\Garybox[En ocaciones]}]{align*}
		EDO+ruido=Mejor \text{ } modelo
	\end{empheq}	
	\begin{overlayarea}{\textwidth}{.7\textheight}
		\begin{columns}
    		\column{.5\textwidth}
			\only<2-3>{		
			\begin{exampleblock}{Crecimiento de Poblaciones}
				$$
					\frac{dN}{dt}=a(t)N(t) \qquad N_0=N(0)=cte.
				$$
			\end{exampleblock}	
			}		
			\only<4-8>{	
			\begin{exampleblock}{Circuitos Eléctricos}
			\begin{align*}
				&L\cdot Q''(t)+
				R\cdot Q'(t)+
				\frac{1}{C}\cdot Q(t)
				=F(t)\\
				&Q(0)=Q_0\\
				&Q'(0)=I_0
			\end{align*}				
		\end{exampleblock}	
		}
		\only<7-8>{
		\begin{empheq}[box=\shadowbox*]{equation*}					
			Q(t)=Z(t)+"ruido"		
		\end{empheq}		
		}		
		\column{.5\textwidth}
		\only<3>{
			\begin{empheq}[box=\shadowbox*]{equation*}		
				a(t)=r(t)+"ruido"
			\end{empheq}
		}	
		\only<5-8>{
			\includegraphics[width=\textwidth]{./images/CircuitRLC.png}		
		}		
		\only<6-8>{
		\begin{empheq}[box=\shadowbox*]{equation*}					
			F(t)=G(t)+"ruido"	
		\end{empheq}		
		}
		\only<8>{
			Estima $Q(t)$ observando $Z(t)$		
		}
	\end{columns}    
	\end{overlayarea}
\end{frame}
%%%%%%%%%%%%%%%%%%%%%%%%%%%%%%%%%%%%%%%%%%%%%%%
\begin{frame}
	\frametitle{Por que hacer métodos numéricos para EDEs?}
	\begin{empheq}[box={\Garybox[En ocaciones]}]{align*}
		EDO+ruido=Mejor \text{ } modelo
	\end{empheq}	
	\begin{overlayarea}{\textwidth}{.7\textheight}
		\begin{columns}
    		\column{.5\textwidth}	
				\begin{alertblock}{Solución analítica?}
					muy RARA
				\end{alertblock}			
			\column{.5\textwidth}
				\begin{block}{Usa }
					Teoría de diferencias finitas y haz una extención estocástica.
				\end{block}		
		\end{columns}    
	\end{overlayarea}
\end{frame}
%%%%%%%%%%%%%%%%%%%%%%%%%%%%%%%%%%%%%%%%%%%%%%%%
\begin{frame}
	\frametitle{Objetivo}
	\begin{alertblock}{Objetivo de la charla}
		\emph{Ilustrar} como aproximar soluciones de EDEs  a partir de \emph{conocimientos básicos} 
		de los \emph{métodos deterministas} y nociones muy elementales de variables aleatorias.
	\end{alertblock}
\end{frame}

 	\begin{frame}
	\frametitle{Plan de Charla}
    \tableofcontents[pausesections]
\end{frame}
%%%%%%%%%%%%%%%%%%%%%%%%%%%%%%%%%%%%%%%%%%%%%%
  \section{Esquemas Steklov}
   \begin{frame}%[label=frm:12]
	\frametitle{Métodos Steklov para din\'amica Browniana(SBD)}
	Queremos aproximar:
	\begin{equation*}
		dX_t=\frac{1}{k_BT} D F(X_t)+ \sqrt{D}dB_t,\quad X_0=cte\quad t\in[0,T].
	\end{equation*}
	\begin{block}{Considerando su forma integral:}
		\begin{equation*}
			X_t=X_0+\frac{1}{k_BT} D \int_{0}^{t}F(X_s)ds+\sqrt{D}\int_{0}^t dB_s
		\end{equation*}
	\end{block}
\end{frame}
% %%%%%%%%%%%%%%%%%%%%%%%%%%%%%%%%%%%%%%%%%%%%%%%%%%%%%%%%%%%%%%%%%%%%%%%%%%
\begin{bibunit}[apalike]
\begin{frame}%[label=frm:13]
	\frametitle{Existencia y unicidad de soluciones}
	\begin{overlayarea}{\textwidth}{.5\textwidth}
	\only<+>{
	\begin{block}{Sea $F:\mathbb{R} \to \mathbb{R}$.}
	  Hipótesis:
	  \begin{itemize}
		\item $D$, $X_0$ constantes.
		\item $\exists \alpha,\beta$ constantes t.q.  $\qquad \forall x,y\in \mathbb{R}$:
		  \begin{align*}
			&|F(x)|\leq \alpha(1+|x|), \\
			&|F(x)-F(y)|\leq \beta |x-y|.
		  \end{align*}
	  \end{itemize}
	\end{block}
	}
	\only<+>{
	  \begin{block}{Bajo estos supuestos  $\exists ! X_t$ t.q.}
		$$
		\mathbb{E}\left(
		\int_{0}^T|X_t|^2dt
		\right)<\infty
		$$
		\cite{KloedenPllaten}
	  \end{block}
	}
  \end{overlayarea}
  \biblio{BibliografiaTesis}
  \end{frame}
\end{bibunit}
%%%%%%%%%%%%%%%%%%%%%%%%%%%%%%%%%%%%%%%%%%%%%%%%%%%%%%%%%%%%%%%%%%%%%%%%
\begin{frame}%[label=frm:14]{}
  \frametitle{Construcción de métdos Steklov}
  \begin{overlayarea}{\textwidth}{.5\textwidth}
	\only<1-2>{
	  Fijemos notación.
	\begin{block}{Discretizamos $[0,T]$ con un  paso uniforme $h$:}
	  \begin{itemize}
		\item $t_n=nh$ $n=0,1,2,\dots, N$.
		\item $X_n \approx X_{t_n}$
	  \end{itemize}
	\end{block}
	}
	\only<2>{
	\begin{block}{Para cada nodo}
		\begin{equation*}
		  X_{t_{n+1}}=X_{t_{n}}+
		  \underbrace{\int_{t_n}^{t_{n+1}}F(X_{s_n})ds}_{\approx \text{Con algún método}}
		  +\sqrt{D} \underbrace{
				(
					B_{t_{n+1}}-B_{t_n}
				)
				}_{:=\Delta B_n}
		\end{equation*}
	  \end{block}
	}
	\only<3>{
	  %Entonces dependiendo de la integral numérica empleada para aproximar
	  %$\int_{t_n}^{t_{n+1}}F(X_{s_n})ds$ se obtiene un método distinto. 
	  
	\begin{exampleblock}{Para el Euler-Mayurama (EM) se considera}
	  \begin{equation*}
	 	\int_{t_n}^{t_{n+1}}F(X_{s_n})ds \approx F(X_{t_n})h
	  \end{equation*}
	 Así, el esquema EM para aproximar :
	 $$X_{t_{n+1}}=X_{t_{n}}+
		  \int_{t_n}^{t_{n+1}}F(X_{s_n})ds
		  + \sqrt{D} (B_{t_{n+1}}-B_{t_n})
	$$
	 	\begin{equation*}
	 		X_{n+1}=X_n+F(X_n)h + \sqrt{D} \Delta B_n \quad n=0,1\dots N-1
	  \end{equation*}
	\end{exampleblock}
	}
  \end{overlayarea}
\end{frame}
%%%%%%%%%%%%%%%%%%%%%%%%%%%%%%%%%%%%%%%%%%%%%%%%%%%%%%%%%%%%%%%%%%%%%%%%%%%%
\begin{bibunit}[apalike]
\begin{frame}%[label=frm:15]{}
  \frametitle{Construcción del SBD}
  \begin{overlayarea}{\textwidth}{.5\textwidth}
    \only<+>{
      \begin{block}{Proponemos usar el promedio de Steklov para aproximar $F(X_t)$}
	  \begin{align*}
	  			\textcolor{cyan}{F(X_t)\approx \varphi(X_n,X_{n+1})}=&
		        \left(
		        \frac{1}{X_{n+1}-X_{n}}
		        \int_{X_n}^{X_{n+1}} \frac{du}{F(u)}
		        \right)^{-1}\\
		      	t_n\leq & t \leq t_{n+1},\\
		      	X_n=&X_{t_n}, \quad t_n=nh.
	  \end{align*}
	\nocite{matus2005exact}
    \end{block}
	}
	\only<+>{
		\begin{block}{Aproximamos}
			\begin{equation*}
				\int_{t_n}^{t_{n+1}}F(X_{s_n})ds \approx \varphi(X_n,X_{n+1})h
			\end{equation*}
		\end{block}
	}
  
  \end{overlayarea}   
	\biblio{BibliografiaTesis}
\end{frame}
\end{bibunit}
%%%%%%%%%%%%%%%%%%%%%%%%%%%%%%%%%%%%%%%%%%%%%%%%%%%%%%%%%%%%%%%%%%%%%%%%%%%%
\tikzstyle{na} = [baseline=-.5ex]
\tikzstyle{blockYellow}= [rectangle, draw, fill=yellow!40,
    text width=6em, text centered, rounded corners, minimum height=4em]

\tikzstyle{blockGreen}= [rectangle, draw, fill=green!40,
    text width=6em, text centered, rounded corners, minimum height=4em]

\tikzstyle{blockRed}= [rectangle, draw, fill=red!40,
    text width=6em, text centered, rounded corners, minimum height=4em] 
\tikzstyle{every picture}+=[remember picture]
\everymath{\displaystyle}
% % % % % % % % % % % % % % % % % % % % % % % % % % % % % % % % % % % % % % % % %
\begin{frame}
	\frametitle{Método Steklov (SBD)}
% \only<+->{
	\begin{block}{Así obtenemos un método Steklov}
		\begin{equation*}
			X_{n+1}=
				\tikz[baseline]{
					\only<1-3>{
						\node[fill=blue!20,anchor=base] (t1)
							{$X_n+\varphi(X_n,X_{n+1}) h$};
						}
					\only<4->{
						\node[fill=green!20,anchor=base] (t1)
							{$X_n^{\star}$};
					}
				}
					+
				\tikz[baseline]{
					\only<1-3>{
						\node[fill=red!20,anchor=base] (t2)
						{$G(X_n)\Delta B_n$};
					}
					\only<4->{
						\node[fill=red!20,anchor=base] (t2)
							{$G(X^{\star}_n)\Delta B_n$};
					}
				}
		\end{equation*}
	\end{block}
	\begin{overlayarea}{\textwidth}{.5\textheight}
		\begin{columns}
			\column{.7\textwidth}
				\begin{itemize}
					\item <2-> 
						\tikz[na] \node [blockYellow] (n1) {$\approx\int_{X_n}^{X_{n+1}} \frac{du}{F(u)}$};
					\item<4-> 
						\tikz[na] \node[fill=green!20,anchor=base] (n3) {$X_n^{\star} = X_n + h \varphi(X_n, X_n^{\star})$};
				\end{itemize}
			\column{.5\textwidth}
			\begin{itemize}
				\item <3-> 
					\tikz[na] \node[fill=blue!20,anchor=base] (n2) {$\varphi(X_n,X_{n+1}^*)$};
			\end{itemize}
		\end{columns}	
		\begin{tikzpicture}[overlay]
			\path [->]<2->    (t1) edge[bend right]   (n1);%
			\path [->]<3->    (t1) edge[bend left]    (n2);%
			\path [->]<4->    (t1) edge[bend left]   (n3);%
		\end{tikzpicture}
	\end{overlayarea}
\end{frame}

  \section{Estabilidad lineal}
    %----------------------------------------------------------------------------------------------
%\begin{bibunit}[apalike]
\begin{frame}
   \frametitle{Estabilidad lineal (métodos deterministas)}% \cite{dahlquist2008numerical}}
   \begin{columns}
   		\column{.4\textwidth}
   		\begin{empheq}[box=\shadowbox]{align*}
			\frac{dx}{dt}=&\lambda x,\\
			 x(0)=&x_0 \quad\lambda \in \mathbb{C}.\\
		\end{empheq}
		\visible<2->{
				\colorbox{yellow}{$x(t)=x_0\exp(\lambda t)$} \\
			} 		
		\visible<3->{
		\begin{equation*}
			\lim_{t\to \infty}x(t)=0\Leftrightarrow Re(\lambda)<0		
		\end{equation*}							
		}		
		\column{.5\textwidth}
		\begin{overlayarea}{1.1\textwidth}{.9\textheight}
			\only<4>{
				\begin{center}
   					\includegraphics[width=\textwidth]{Imagenes/Estabilidad/EstabilidaLineal.png}
   				\end{center}
 			}
 		\end{overlayarea} 	    		
   \end{columns}
%   %\biblio{BibliografiaTesis}
\end{frame}
%%\end{bibunit}
%%------------------------------------Metodo de Euler-------------------------------
%%%%%%%%%%%%%%%%%%%%%%%%%%%%%%%%%%%%%%%%%%%%%%%%%%%%%%%%%%%%%%%%%%%%
\begin{frame}%[label=frm:17]
  \frametitle{Estabilidad Lioneal (métodos deterministas)}% \cite{dahlquist2008numerical}}
  \begin{columns}
	\column{.4\textwidth}
	  \colorbox{darkyellow}{
			$
				\displaystyle				
				\lim_{t\to \infty}x(t)=0\Leftrightarrow Re(\lambda)<0
			$
	  }
	\visible<+->{	
		\begin{center}
			Método de Euler	
		\end{center} 
		\begin{empheq}[box=\ovalbox]{equation*}	 
			\frac{x_{n+1}-x_n}{h}=\lambda x_n
		\end{empheq}	 
	  }
	  \visible<+->{
		\begin{overlayarea}{.8\textwidth}{.3\textheight}
		  \only<+>{
			\begin{empheq}[box=\mybluebox]{align*}
		     % \begin{align*}
			  x_{n+1}=&\lambda x_n h+x_n\\
				=&
				  x_n(\lambda h +1)\\
				\vdots\\
				=&
				  x_0(\lambda h +1)^{n+1}
		    % \end{align*}
			\end{empheq}
		  }
		  \only<+->{
			\colorbox{greenArea}{
			$
			  \displaystyle
			  \lim_{n\to \infty}x_n=0\Leftrightarrow |\lambda h +1|<1
			$
			}				
		  }
	  \end{overlayarea}
		}
	\column{.6\textwidth}
	  \begin{overlayarea}{1.1\textwidth}{.9\textheight}
		\only<+>{
		  \begin{center}
			\includegraphics[width=\textwidth]{Imagenes/Estabilidad/EstabilidadEuler1.png}
		  \end{center}
  		}
  		\only<+>{
		  \begin{center}
			\includegraphics[width=\textwidth]{Imagenes/Estabilidad/EstabilidadEuler2.png}
		  \end{center}
  		}
  		\only<+>{
		  \begin{center}
			\includegraphics[width=\textwidth]{Imagenes/Estabilidad/EstabilidadEuler3.png}
		  \end{center}
  		} 
	  \end{overlayarea}
  \end{columns}
  %\biblio{BibliografiaTesis}
\end{frame}
%%----------------------------------------------------------------------------
\begin{frame}%[label=frm:18]
   \frametitle{Estabilidad lineal EDEs}
	\begin{overlayarea}{\textwidth}{.3\textheight}
		\begin{empheq}[box={\Garybox[Consideremos]}]{equation*}\label{eqn:SDETest}
 			dX_t=\lambda X_t dt +\beta dB_t, \qquad X_0=cte., \lambda, \beta \in \mathbb{C} \tag{DP}.
		\end{empheq}
	\end{overlayarea}	
	\begin{overlayarea}{\textwidth}{.4\textheight}	
		\only<2>{
		\begin{block}{Solución exacta}
			\begin{empheq}{equation*}\label{eqn:SolSDETest}
				 X_t=X_0\exp(\lambda t) +\beta \int_{0}^t \exp(\lambda (t-s))dB_s \tag{*}
			\end{empheq}
			\begin{empheq}{align*}
			 	 \label{eqn:Mean}
		  		\mathbb{E}[X_t]=&\exp(\lambda t)\mathbb{E}[X_0]  \tag{**} \\
		  		\label{eqn:Variance}
		  		\mathbb{E}[|X_t|^2]=& \exp(2Re(\lambda)t) \mathbb{E}[|X_0|^2] -
		  		\frac{|\beta|^2}{2Re(\lambda)}(1-\exp(2Re(\lambda)t)) .\tag{***}
			\end{empheq}
	\end{block}
	}
	\only<3>{
		(DP) tiene solución asintótica-estable
 		\begin{equation*}
 			\colorbox{yellow}{$
 			\begin{matrix}
 		  		\lim_{t \to \infty} \mathbb{E} |X(t)|= 0 \\
 		  		\lim_{t \to \infty} \mathbb{E}[|X(t)|^2]= -\frac{|\beta|^2}{2 Re(\lambda)} %
 			\end{matrix}$}
 	      \Leftrightarrow Re(\lambda)<0
 		\end{equation*}	
	}
	\end{overlayarea}	
\end{frame}
%%----------------------------------------------------------------------------------------------------
%%\begin{bibunit}
\begin{frame}%[label=frm:19]
    \frametitle{Estabilidad Lineal}
	\begin{columns}
		\column{.5\textwidth}
		\begin{overlayarea}{\textwidth}{\textheight}			
			\only<+->{
				\begin{definicion}
  					$X(t)$ es asint\'oticamente estable en media (AEM) si y s\'olo si
					\begin{equation*}\label{eqn:ConMeanEstable}
	 					\lim_{t \to \infty}\mathbb{E}[|X(t)|]=0.
					\end{equation*}
				\end{definicion}
			}
			\only<+->{
			\begin{definicion}
				 $X(t)$ es asint\'oticamente estable en media  cuadr\'atica (AEMC) si y sólo si
				\begin{equation*}\label{eqn:ConMeanSquareEstable}
	 				\lim_{t \to \infty}\mathbb{E}[|X(t)|^2]=-\frac{|\beta|^2}{2Re(\lambda)}.
				\end{equation*}
			\end{definicion}
			}
		\end{overlayarea}
		\column{.5\textwidth}
		\only<+->{ 
				\begin{definicion}
  					La solución  numérica es asint\'oticamente estable en media (AEM) si y s\'olo si
					\begin{equation*}\label{eqn:ConMeanEstable}
	 					\lim_{n \to \infty}\mathbb{E}[|X_n|]=0.
					\end{equation*}
				\end{definicion}
				\begin{definicion}
	  				La solución numérica es asint\'oticamente estable en media  cuadr\'atica (AEMC) si y sólo si
					\begin{equation*}\label{eqn:ConMeanSquareEstable}
		 				\lim_{n \to \infty}\mathbb{E}[|X_n|^2]=-\frac{|\beta|^2}{2Re(\lambda)}.
					\end{equation*}
				\end{definicion}
			}
	\end{columns}	
%%\biblio{BibliografiaTesis}
\end{frame}
%%\end{bibunit}
%%%%%%%%%%%%%%%%%%%%%%%%%%%%%%%%%%%%%%%%%%%%%%%%%%%%%%%%%%%%%%%%%%%%
%\begin{bibunit}[apalike]
\begin{frame}%[label=frm:20]
    \frametitle{$A$-estabilidad}
	\begin{overlayarea}{\textwidth}{\textheight}
		\only<1>{
		\begin{block}{Buscamos $\lambda h$}
			para los cuales el m\'etodo Steklov \textcolor{red}{reproduce
			correctamente} la  \textcolor{cyan}{estabilidad en media y media cuadr\'atica.}
		\end{block}
		}
		\only<2>{
		\begin{definicion}
			Diremos que un método será \textcolor{red}{$A$-estable} en \textcolor{red}{media o media cuadrática} si \cite{higham2000stability}
			$$
				\text{Problema estable } \Rightarrow \text{ método estable }\forall h
			$$
		\end{definicion}
		}
	\end{overlayarea}
\end{frame}
%\end{bibunit}
%%---------------------------Resultado sobre Media---------------------------------------------------------
\begin{frame}%[label=frm:20]
	\begin{empheq}[box=\shadowbox*]{equation*}
				dX_t=\lambda X_t dt +\beta dB_t, \quad X_0=cte, \lambda, \beta \in
				\mathbb{C} \qquad \text{(DP)}
	\end{empheq}
%%-------------------------------------------------------------------------------------------------------------------------------	
	\begin{columns}		
		\column{.5\textwidth}
		\begin{overlayarea}{\textwidth}{\textheight}
			\only<2-6>{
				\centering{\alert<3-4>{\textbf{Euler-Mayurama}}}
				\begin{empheq}[box={\ovalbox}]{align*}
					X_{n+1} = &(1+\lambda h) X_n +\beta \Delta B_n \\
					\Delta B_n=& B_{t_{n+1}}-B{t_n}\\
					B_{n}=&\sqrt{h}V_n\\
						V_n\sim& N(0,1).
				\end{empheq}			
			}
			\only<4-6>{
				\begin{empheq}{align*}
					\mathbb{E}(X_{n+1})=\mathbb{E}(X_0)(1+\lambda h)^{n+1}
				\end{empheq}				
			}
			\only<7>{
				\begin{Teorema}
					El método Steklov para la ecuación (DP)	es $A$-estable en media.
				\end{Teorema}
			}	
		\end{overlayarea}
	\end{columns}
\end{frame}
%%%%%%%%%%%%%%%%%%%%%%%%%%%%%%%%%%%%%%%%%%%%%%%%%%%%%%%%%%%%%%%%%%%%%%%%%%%%%%%%%%%%%%%%%%%%%%%%%

	\begin{bibunit}[apalike]
\begin{frame}%[label=frm:20]
    \frametitle{Estabilidad en Media Cuadrática \cite{saito1996stability}}
	\biblio{BibliografiaTesis}
\end{frame}
\end{bibunit}
%%%%%%%%%%%%%%%%%%%%%%%%%%%%%%%%%%%%%%%%%%%%%%%%%%%%%%%%%%%%%%%%%%%
%\begin{bibunit}[apalike]
\begin{frame}%[label=frm:20]
	\frametitle{Estabilidad en Media Cuadrática}
	\begin{overlayarea}{\textwidth}{.25\textheight}	
		\only<1-4>{
			\begin{empheq}[box=\shadowbox*]{equation*}
				dX_t=\lambda X_t dt +\beta dB_t, \quad X_0=cte, \lambda, \beta \in
				\mathbb{C} \qquad \text{(DP)}
			\end{empheq}
		}
		\only<5>{
		 	\begin{empheq}[box=\shadowbox*]{equation*}
				\lim_{t \to \infty}\mathbb{E}[|X_t|^2]=-\frac{|\beta|^2}{2Re(\lambda)}.
			\end{empheq}	
		}	
	\end{overlayarea}
	\begin{columns}	
		\column{.5\textwidth}		
		\begin{overlayarea}{\textwidth}{\textheight}				
			\centering{Euler-Mayurama}		
			\only<1->{				
				\begin{empheq}[box={\ovalbox}]{align*}
						X_{n+1} = &(1+\lambda h) X_n +\beta \Delta B_n 
				\end{empheq}							
			}
%			\only<2-5>{
%				\begin{empheq}{align*}		
%					\mathbb{E}(|X_{n+1}|^{2})=&
%						|1+\lambda h|^{2}\mathbb{E}(|X_{n}|^{2})+|\beta|^{2}h
%				\end{empheq}
%			}
%			\only<3-4>{
%				\scalebox{0.9}{% Scale by 90%									
%					$\displaystyle
%						=
%						|1+\lambda h|^{2}(|1+\lambda h|^{2}\mathbb{E}(|X_{n-1}|	^{2})+|\beta|^{2}h)
%						+|\beta |^{2}h
%					$
%				}					
%			}
%		\only<4>{					
%				\scalebox{0.9}{% Scale by 90%									
%					$\displaystyle 
%						=
%						|1+\lambda h|^{4}\mathbb{E}(|X_{n-1}|^{2})
%						+
%						\left[|1+\lambda h|^{2}+1\right]|\beta|^{2}h\				
%					$						
%				}
%				$$
%						\vdots
%				$$
%		}
%		\only<5-6>{
%			\scalebox{0.9}{% Scale by 90%
%					$ =
%						|1+\lambda h|^{2(n+1)}\mathbb{E}(|X_{0}|^{2})+
%						\underbrace{						
%							\left[|1+\lambda h|^{2n}+\cdots+|1+\lambda h|^{2}+1\right]
%							}_{\textit{Serie geométrica}}
%							|\beta|^{2}h
%					$
%			}
%		}
%		\only<6>{
%			\scalebox{0.9}{% Scale by 90%
%					 $	=
%					 	|1+\lambda h|^{2(n+1)}
%					 	\mathbb{E}(|X_{0}|^{2})+
%					 	\frac{|1+\lambda h|^{2(n+1)}-1}{|1+\lambda h|^{2}-1}
%					 	|\beta|^{2}h
%					$	
%			}
%			\scalebox{0.9}{% Scale by 90%
%					$					 
%					  =
%					 	|1+\lambda h|^{2(n+1)}\mathbb{E}(|X_{0}|^{2})+
%					 	\frac{|1+\lambda h|^{2(n+1)}-1}{2Re(\lambda)+|\lambda|^{2}h}|\beta|^{2}.			
%					$
%			}		
%		}
%		\only<7-12>{
%			\begin{empheq}{align*}
%					\mathbb{E}(|X_{n+1}|^{2}) & = |1+\lambda h|^{2(n+1)}\mathbb{E}(|X_{0}|^{2})\\
%						&+
%							\frac{|1+\lambda h|^{2(n+1)}-1}{2Re(\lambda)+|\lambda|^{2}h}|\beta|^{2}\\
%			\end{empheq}
%		}
		\only<2->{
			Si $Re(\lambda h)<0$			
			\begin{align*}
  					\mathbb{E}[|X_{n+1}|^2]
  					\xrightarrow{n\to\infty}  			
  					\colorbox{darkyellow}{
  				 		$\frac{-|\beta|^2 }{2Re(\lambda)+|\lambda|^2h}$
  					}
  				\end{align*}
		}
		\end{overlayarea}
%--------------------------------------------------------------------------------		
		\column{.6\textwidth}		
		\begin{overlayarea}{\textwidth}{\textheight}	
			\only<3->{
				\centering{\alert<5->{\textbf{Steklov}}}					
					\begin{empheq}[box=\mybluebox]{align*}
						X_{n+1} = & X_n \exp(\lambda h)+\beta \Delta B_n
					\end{empheq}
			}						
%			\only<+>{	
%				\begin{equation*}
% 					\mathbb{E}[|X_{n+1}|^2]=\exp(2Re(\lambda)h)\mathbb{E}[|X_n|^2]+|\beta|^2 h.
%				\end{equation*}	
%			}
%			\only<10>{	
%					\begin{align*}
%						\mathbb{E}[|X_{n+1}|^2]&=\exp(2Re(\lambda)h)\mathbb{E}[|X_n|^2]+|\beta|^2 h\\
%  						\vdots &=\vdots\\
%								&= \exp(2Re(n+1)\lambda h)\mathbb{E}[|X_{0}|]\\
%								&+|\beta|^2h 
%	    				\underbrace{
%	      					\left(
%	      						1+\cdots+\exp(2nRe(\lambda) h)
%	     						\right)
%	    						}_{\text{Serie geométrica}}\\
%  					\end{align*}	
%			}
%			\only<11>{
%				\begin{align*}
%					\mathbb{E}[|X_{n+1}|^2]&=\exp(2(n+1)Re(\lambda)h)\mathbb{E}[|X_0|^2]\\
%  							+&\beta^2 h \frac{(\exp(2Re(\lambda)h))^{n+1}-1}{\exp(2Re(\lambda)h)-1}  		
%  				\end{align*}
%			}	
			\only<4->{
				Si $Re(\lambda)<0$
				\begin{align*}
  					\mathbb{E}[|X_{n+1}|^2]
  					\xrightarrow{n\to\infty}
  					\colorbox{hellcyan}{
  				 		$\frac{-|\beta|^2 h}{\exp(2Re(\lambda)h)-1}$
  					}
  				\end{align*}		
			}
		\end{overlayarea}
	\end{columns}
	%\biblio{BibliografiaTesis}
\end{frame}
%\end{bibunit}
%%%%%%%%%%%%%%%%%%%%%%%%%%%%%%%%%%%%%%%%%%%%%%%%%%%%%%%%%%%%%%%%%%%%%%%%%%%%%%%%%%%%%%%%%%%%%%%%%
\begin{frame}
  \frametitle{Estabilidad en Media Cuadrática}
  \begin{empheq}{equation}
 		dX_t=\lambda X_t dt +\beta dB_t, \qquad X_0=cte., \lambda, \beta \in \mathbb{C} \tag{E}.
  \end{empheq}
   \begin{columns}
	  \column{.5\textwidth}
		\begin{definicion}[Consistencia lineal en MC]
	 		Un esquema numérico para  la ecuación (E) se dice asintóticamente consistente en media
	 		cuadrática si la solución numérica satisface
 		$$
 		  \lim_{\substack{ n\to \infty\\ h\to 0}} X_n= \frac{-\beta}{2Re(\lambda)}
 		$$
 	  \end{definicion}
	  \column{.5\textwidth}
 	  \begin{Teorema}
 		 El esquema Steklov para la ecuación (E), es asintóticamente consistente en MC.
 	  \end{Teorema}
  \end{columns}
\end{frame}
%%%%%%%%%%%%%%%%%%%%%%%%%%%%%%%%%%%%%%%%%%%%%%%%%%%%%%%%%%%%%%%%%%%%%%%%%%%%%%%%%%%%%%%%%%%%%%%%%
\begin{bibunit}[alpha]
\begin{frame}
	\frametitle{Stabilidad lineal por trayectorias}
	\begin{empheq}{equation*}
		dX_t=\lambda X_t dt +\beta dB_t, \qquad X_0=cte., \lambda, \beta \in \mathbb{R} \tag{E}.
	\end{empheq}
	\nocite{Buckwar2011a}
	\begin{columns}
		\column{.5\textwidth}
			\begin{overlayarea}{\textwidth}{.5\textheight}
				\centering Pullback attractor
				\only<2->{
					\begin{empheq}[box=\shadowbox*]{equation*}
						\lim_{t_0\to-\infty} X_t =\widehat{O}_t %
						:= e^{\lambda t}\int\limits_{-\infty}^{t}e^{-\lambda s}dB_s, 
					\end{empheq}
				}
		\end{overlayarea}
		\column{.5\textwidth}
			\begin{overlayarea}{\textwidth}{.5\textheight}
			\vspace{-1.0cm}
			\only<3->{
				\begin{Teorema}
					Sea $\lambda<0$,  el método Steklov para (E)
					tiene el siguiente atractor
					\begin{equation*}
						\widehat{O}_n^{(h)}  :=
						\xi \sum_{j=-\infty}^{n-1}\exp(\lambda h(n-1-j)) \Delta B_j,
					\end{equation*}
					$\widehat{O}_n^{(h)} \to \widehat{O}_t$, \quad $h\to 0$, \quad pathwise.
				\end{Teorema}
			}
			\end{overlayarea}
	\end{columns}
\biblio{BibliografiaTesis}
\end{frame}
\end{bibunit}
%%%%%%%%%%%%%%%%%%%%%%%%%%%%%%%%%%%%%%%%%%%%%%%%%%%%%%%%%%%%%%%%%%%%%%%%%%%%%%%%%%%%%%%%%%%%%%%%%
\begin{frame}
	\frametitle{Estabilidad en Media Cuadrática Ruido Multiplicativo}
	\begin{empheq}[box=\shadowbox*]{equation*}
		dX_t=\lambda X_t dt +\xi X_t dB_t, \qquad X_0=cte., \lambda, \xi \in \mathbb{R} \tag{E}.
	\end{empheq}
	\begin{columns}
		\column{.5\textwidth}
		\only<2->{
			\begin{exampleblock}{MS-estabilidad Lineal}
				\begin{itemize}
					\item
						diagonal (EM)
					\item
						vertical (Steklov)
				\end{itemize}
			\end{exampleblock}
		}
		\column{.5\textwidth}
			\centering
			\only<3->{
				\includegraphics[width=1\linewidth]{images/StabilityPlotMultiplicativeNoise}
			}
	\end{columns}
\end{frame}




  \section{Consistencia y estabilidad}
	\begin{bibunit}[apalike]
  \begin{frame}%[label=frm:10]
    \frametitle{Definiciones y resultados previos}
	\cite{KloedenPllaten}
	\biblio{BibliografiaTesis}
  \end{frame}
\end{bibunit}
%%%%%%%%%%%%%%%%%%%%%%%%%%%%%%%%%%%%%%%%%%%%%%%%%%%%%%%%%%%%%%%%%%%%%%%%%%%%%%%%%%%%%%%%%%%%%%
\begin{frame}
    \frametitle{Consistencia convergencia y estabilidad en sentido fuerte}
	\vspace{-1.0cm}	
	\begin{empheq}[box={\Garybox[EDE]}]{equation*}
		dX_t=a(X_t)dt+b(X_t)dB_t, \qquad X_0=x_0 \quad (EDE)
	\end{empheq}		
	\vspace{-.45cm}	
	\begin{columns}
		\column{.3\textwidth}
			$Y^{h}$ 	esquema con paso 	 $\max h$.		
		\column{.5\textwidth}
		\begin{empheq}[box=\shadowbox]{equation*}
			\varepsilon(h)= \mathbb{E}
		  \left(
		  	X_T-Y^{h}(T)
		  \right)
		\end{empheq}		
	\end{columns}	
	\vspace{-.5cm}	
	\begin{overlayarea}{\textwidth}{.7\textheight}
	\only<+->{
	  \begin{definicion}[Consistencia]
		$Y^{h}$  a los	  tiempos
		$
		  \left(
			\tau
		  \right)_{h}=
		  \left\{
			\tau_{n}:n=0,1,\cdots
		  \right\}
		$
		es \textcolor{red}{consistente en sentido fuerte},
		si $\exists C=C(h)\geq 0,\quad h_0$ t.q. 
		$\textcolor{red}{\forall Y_n^{h}},n=1,2,\cdots N \quad h\in(0,h_0)$
		\begin{itemize}[<+->]
		  \item $\displaystyle \lim_{h\downarrow 0} C(h)=0$
		  \item
		  $
			\displaystyle
			 \mathbb{E}
			  \left(
				\left|
				  \mathbb{E}
				  \left(
					\frac{Y_{n+1}^{h}-Y_n^{h}}{h_n}
					  \left|
						\mathcal{F}_{\tau_n}
					  \right.
				  \right)
				-a\left(
					Y_{n}^{h}
				  \right)
				\right|^2
			 \right)\leq C(h).
		  $
		  \item
			\renewcommand{\arraystretch}{1.5}%
			\scalebox{0.8}{% Scale by 50%
			$
			  \mathbb{E}
			  \left(
				\frac{1}{h_n}
				\left|
				Y_{n+1}^{h}-Y_{n}^{h}-
				\mathbb{E}
				\left(
				  \frac{Y_{n+1}^{h}-Y_n^{h}}{h_n}
					\left|
					\mathcal{F}_{\tau_n}
				\right.
			  \right)
			-b\left(Y_{n}^{h}
			\right)\Delta B_n
			\right|^2
		  \right)\leq C(h).
		 $ }
		\normalsize
		\end{itemize}
	  \end{definicion}
		}
	\end{overlayarea}
\end{frame}
%%%%%%%%%%%%%%%%%Convergencia%%%%%%%%%%%%%%%%%%%%%%%%
\begin{frame}
    \frametitle{Consistencia convergencia y estabilidad en sentido fuerte}
	\vspace{-1.0cm}	
	\begin{empheq}[box={\Garybox[EDE]}]{equation*}
		dX_t=a(X_t)dt+b(X_t)dB_t, \qquad X_0=x_0 \quad (EDE)
	\end{empheq}		
	\vspace{-.45cm}	
	\begin{columns}
		\column{.3\textwidth}
			$Y^{h}$ 	esquema con paso 	 $\max h$.		
		\column{.5\textwidth}
		\begin{empheq}[box=\shadowbox]{equation*}
			\varepsilon(h)= \mathbb{E}
		  \left(
		  	X_T-Y^{h}(T)
		  \right)
		\end{empheq}		
	\end{columns}	
	\vspace{-.5cm}	
	\begin{overlayarea}{\textwidth}{.7\textheight}
	 \only<+>{
	  \begin{definicion}[Convergencia fuerte]
		$Y^{h}$ \textcolor{red}{converge} en \textcolor{red}{sentido fuerte} a $X$ a tiempo $T$ si
		\begin{equation*}
		  \lim_{h \downarrow 0}
		  \mathbb{E}
		  \left(
		  X_T-Y^{h}(T)
		  \right)=0
		\end{equation*}
	  \end{definicion}
	  }
	\only<+>{
	  \begin{definicion}[orden de convergencia]
		$Y^{h}$
		\textcolor{red}{converge} en sentido fuerte \textcolor{red}{con orden $\gamma$},
		 si $\exists C$
		independiente de $h $ y $h_{0}$ t.q.
		\begin{equation*}
		  \epsilon(h)=
		  \mathbb{E}
		  \left(
			|X_T-Y(T)|
			\right)\leq C h^{\textcolor{red}{\gamma}} \qquad\forall h\in (0,h_0).
		\end{equation*}
	  \end{definicion}
	  }
	\end{overlayarea}
\end{frame}
%%%%%%%%%%%%%%%%%%%%%%%%%%%%%%%%%%%%%%%%%%%%%%%
\begin{frame}
	\frametitle{Consistencia convergencia y estabilidad en sentido fuerte}	
	\hypertarget{thm:ConsistenciaConvergencia}{}	
	\vspace{-3.8cm}	
	\begin{empheq}[box={\Garybox[EDE]}]{equation*}
		dX_t=a(X_t)dt+b(X_t)dB_t, \qquad X_0=x_0 \quad (EDE)
	\end{empheq}
	%\vspace{-.68cm}		  
 	 \begin{Teorema}
		Bajo las condiciones del teorema de
		\hyperlink{thm:ExistenciaUnicidadEDE}{ \textbf{existencia y unicidad}} 
		para soluciones fuertes	de  (EDE). Si $Y^{h}$ es \textcolor{red}{consistente} entonces
		$Y^{h}$ \textcolor{red}{converge} en sentido fuerte a la solución $X_t$.
  \end{Teorema}
\end{frame}
%%%%%%%%%%%%%%%%%%%%%%%%%%%%%%%%%%%%%%%%%%%%%%%%
\begin{frame}
	\frametitle{Consistencia convergencia y estabilidad en sentido fuerte}	
	\hypertarget{thm:ConsistenciaConvergencia}{}	
	\vspace{-.4cm}	
	%	\begin{empheq}[box={\Garybox[EDE]}]{equation*}
%		dX_t=a(X_t)dt+b(X_t)dB_t, \qquad X_0=x_0 \quad (EDE)
%	\end{empheq}
\vspace{-0.2cm}		  
\Fontvi 		
	\[
		Z(t):=\sup_{0\leq s\leq t}\mathbb{E}\left(\left|Y_{n_{s}}-X_{s}\right|^{2}\right),\quad
		n_{t}\max\left\{ n:\tau_{n}\leq t\right\} 
	\] 	
	
	\vspace{-.78cm}	
	\begin{overlayarea}{\textwidth}{\textheight} 
 	\begin{proof}
		\only<1-6>{
		\begin{multline*}
		Z(t):=  
		\sup_{0\leq s\leq t} 
		\mathbb{E}
		\left(
			\left|
				\textcolor<2>{red}{\sum_{n=0}^{n_{s}-1}
					\left(	Y_{n+1}^{h}-Y_{n}^{h}\right)}
					-
				\textcolor<3>{red}{%
					\int_{0}^{s}%
						a\left(X_{r}\right)dr-%
					\int_{0}^{s}b%
					\left(X_{r}\right)%
					dB_{r}	
				}					
			\right|^{2}%	
		\right)\\
		\leq% 
		C_{1} \sup_{0\leq s \leq t} \left\{ 
		\textcolor<4>{cyan}{\mathbb{E}
		\left(
			\left|
				\sum_{n=0}^{n_{s}-1}\mathbb{E}
				\left(Y_{n+1}^{h}-Y_{n}^{h}|	\mathcal{A}_{\tau_{n}}\right)
				-a\left(Y_{n}^{h}
			\right)h
			\right|^{2}
		\right)
		}
		 \right.\\		
		 +
		\textcolor<4>{cyan}{\mathbb{E}\left(
			\left|\sum_{n=0}^{n_{s}-1}
				\left(Y_{n+1}^{h}-Y_{n}^{h}\right)
				-\mathbb{E}\left(Y_{n+1}^{h}-Y_{n}^{h}|\mathcal{A}_{\tau_{n}}\right)
				-b\left(Y_{n}^{h}\right)\Delta B_{n}\right|^{2}
		\right)
		}\\
		+
		\textcolor<5>{colKeys}{		
		\mathbb{E}
		\left(\left|\int_{0}^{\tau_{n_{s}}}a\left(Y_{n_{r}}^{h}\right)
		-a\left(X_{r}\right)dr\right|^{2}\right)
		 +
		 \mathbb{E}\left(\left|\int_{0}^{\tau_{n_{s}}}b\left(Y_{n_{r}}^{h}\right)
		 -b\left(X_{r}\right)dB_{r}\right|^{2}\right)
		}		 
		 \\
		 +
		\left.		 
		\textcolor<5>{colKeys}{		 
		 \mathbb{E}
		 	\left(\left|\int_{\tau_{n_{s}}}^{s}a\left(X_{r}\right)\right|^{2}dr\right)+\mathbb{E}
		 	\left(\left|\int_{\tau_{n_{s}}}^{s}b\left(X_{r}\right)\right|^{2}dB_{r}\right)
		}		 
		 \right\}			
	\end{multline*}
	\hyperlink{thm:Isometria}{{\beamergotobutton{Isometr\'ia}}}	
		
	}
	\only<7>{
	\hypertarget{prb:ConsistenciaISO}{}		
		\begin{multline*}
			Z(t)\leq C_{1}\sup_{0\leq s\leq t}
			\left\{	
				T h\sum_{n=0}^{n_{s}-1}
				\mathbb{E}\left(
				\left|\mathbb{E}\left(\frac{Y_{n+1}^{h}-Y_{n}^{h}|\mathcal{A}_{\tau_{n}}}{h}\right)
				-a\left(Y_{n}^{h}\right)\right|^{2}\right)
			\right.\\
			+
			\mathbb{E}
			\left(
				\left|\sum_{n=0}^{n_{s}-1}\left(Y_{n+1}^{h}
				-Y_{n}^{h}\right)-\mathbb{E}\left(Y_{n+1}^{h}-Y_{n}^{h}|
				\mathcal{A}_{\tau_{n}}\right)-b\left(Y_{n}^{h}\right)\Delta B_{n}\right|^{2}
			\right)\\
			+
			K^{2}T\mathbb{E}
			\left(
				\int_{\tau_{n_{s}}}^{s}\left|Y_{n_{r}}^{h}-X_{r}\right|
				^{2}dr\right)+K^{2}\mathbb{E}\left(\int_{0}^{\tau_{n_{s}}}\left|Y_{n_{r}}^{h}-
				X_{r}\right|^{2}dr
			\right)\\
			+
			\left.			
			K^{2}T
			\mathbb{E}\left(
				\int_{\tau_{n_{s}}}^{s}1+\left|X_{r}\right|				^{2}dr\right)+K^{2}\mathbb{E}\left(\int_{0}^{\tau_{n_{s}}}1+\left|X_{r}\right|^{2}dr
			\right)
		\right\}
		\end{multline*}			
	}	
	\only<8>{
		\begin{multline*}			
			Z(t)\leq C_{1}\sup_{0\leq s\leq t}
			\left\{
				Th\sum_{n=0}^{n_{s}-1}
				\mathbb{E}
				\left(
					\left|\mathbb{E}
						\left(\frac{Y_{n+1}^{h}-Y_{n}^{h}|\mathcal{A}_{\tau_{n}}}{h}\right)
						-a\left(Y_{n}^{h}\right)\right|^{2}\right)
				\right.\\
			+
			\mathbb{E}
			\left(
				\left|\sum_{n=0}^{n_{s}-1}\left(Y_{n+1}^{h}-Y_{n}^{h}\right)
				-\mathbb{E}\left(Y_{n+1}^{h}-Y_{n}^{h}|\mathcal{A}_{\tau_{n}}\right)
				-b\left(Y_{n}^{h}\right)\Delta B_{n}\right|^{2}
			\right)\\
				+
			\left. 
					K^{2}(1+T)\int_{0}^{\tau_{n_{s}}}Z(r)dr+K^{2}(1+T)(1+C_{2})h
			\right\}		
		\end{multline*}				
	}
	\only<9->{
		\begin{empheq}[box=\mybluebox]{equation*}			
			Z(t)\leq C_{3}\int_{0}^{\tau_{n_{s}}}Z(r)dr+C_{4}(h+C(h))
 		\end{empheq}
		Por el lema de \hyperlink{lem:Gronwall}{\beamergotobutton{Gronwall}	}	
		\hypertarget{prb:Consistencia2}{}	
	}
	\only<10-11>{	
		\begin{align*}
			Z(t) & \leq C_{4}(h+C(h))e^{\left(\int_{0}^{\tau_{n_{s}}}C_{3}dr\right)}\\
			\textcolor{red}{
				Z(t)
			}
			 &
			\textcolor{red}{ 
				\leq C_{5}(h+C(h))}
		\end{align*}
	}
	\only<11>{	
		\hypertarget{prb:Consistencia3}{}	
		Por desigualdad de \hyperlink{thm:DesLyapunov}{\beamergotobutton{Lyapunov}}
	}
	\only<12->{	
		\colorbox{darkyellow}{		
		$
			\displaystyle			
			\mathbb{E}\left(\left|Y_{T}^{h}-X_{T}\right|\right)\leq\sqrt{{Z(T)}}\leq\sqrt{{C_{5}(h+C(h))}}		
		$ 
		}	
	}	
 	\end{proof}
	\only<13>{
	\begin{empheq}[box={\Garybox[Euler-Mayurama]}]{equation*}
		Y_{n+1}=Y_na(Y_n)h+b(Y_n)\Delta B_n	
	\end{empheq}
	Es consistente y de orden de convergencia $\frac{1}{2}$.	
	}
 \end{overlayarea}
\end{frame}
	\section{Resultados numéricos}
  	%%%%%%%%%%%%%%%%%%%%%%%%%%%%%%%%%%%%%%%%%%%%%%%%%%%%%%%%%%%%%%%%%%%%%%%%%%%%%%%
%\begin{bibunit}[apalike]
\begin{frame}%[label=frm:20]
    \frametitle{Aproximación de trayectorias}
	\begin{overlayarea}{1.05\textwidth}{1.05\textheight}
	\only<+>{
	\begin{center}
	  \begin{figure}
\includegraphics[width=.9\textwidth,height=.8\textheight]{./Imagenes/SDETestAdditiveNoise/Trayectorias/SDETestAdditiveNoiseDt_025.png}
	  \caption{Trayectorias generadas con $h=\num{.25}$}
	  \end{figure}
	\end{center}
		}
		\only<+>{
		\begin{figure}

\includegraphics[width=.9\textwidth,height=.8\textheight]{./Imagenes/SDETestAdditiveNoise/Trayectorias/SDETestAdditiveNoiseDt_1.png}
			\caption{Trayectorias generadas con $h=\num{1.0}$}
		\end{figure}
		}
		\only<+>{
		\begin{figure}

\includegraphics[width=.9\textwidth,height=.8\textheight]{./Imagenes/SDETestAdditiveNoise/Trayectorias/SDETestAdditiveNoiseDt_2.png}
			\caption{Trayectorias generadas con $h=\num{2.0}$}
		\end{figure}
		}
	\end{overlayarea}
	%\biblio{BibliografiaTesis}
\end{frame}
%\end{bibunit}
%%%%%%%%%%%%%%%%%%%%%%%%%%%%%%%%%%%%%%%%%%%%%%%%%%%%%%%%%%%%%%%%%%%%%%%%%%%%%%%%
%%%%%%%%%
%\begin{bibunit}[apalike]
\begin{frame}%[label=frm:20]
  \frametitle{Aproximación de momentos}
    Calculando \num{1000} trayectorias se genera la siguiente tabla.
    \begin{overlayarea}{\textwidth}{\textheight}
      \only<2>{
    \begin{table}
     \rowcolors{2}{RoyalBlue!5}{RoyalBlue!20}
      \begin{tabular}{l|l|l} \hline
	\multicolumn{3}{c}{$\varepsilon_{debil}=
	\left\|\mathbb{E}\left[|X_{t_n}|\right]-\mathbb{E}[|X_n|]\right\|_2$}
	\\
	\hline
	  $h$		&Euler-Mayurama	& Steklov\\
	  \hline
	  \num{.25}	&\num{1.3878}		&\num{0.2370} \\
	  \num{.5}	&\num{2.1409}		&\num{0.2851} \\
	  \num{1}	&\num{3.9688}		&\num{0.2229} \\
	  \num{2}	&\alert{40.4466}	&\num{0.1439}
	  \\
	  \hline
	\end{tabular}
	\caption{Error en sentido débil para el primer momento.}
      \end{table}
    }
    \only<3>{
      \begin{table}
	\rowcolors{2}{RoyalBlue!5}{RoyalBlue!20}
	 \begin{tabular}{l|l|l} \hline
		\multicolumn{3}{c}{$\varepsilon_{debil}=

\left\|\mathbb{E}\left[|X_{t_n}|^2\right]-\mathbb{E}[|X_n|^2]\right\|_2$}
		\\
		\hline
	  $h$		&CBD	& SBD\\
	 \hline
	  \num{.25}	&\num{11.4890}			&\num{4.2098} \\
	  \num{.5}	&\num{15.1000}			&\num{1.7700} \\
	  \num{1}	&\num{13.5000}			&\num{0.9760} \\
	  \num{2}	&\alert{\num{468.0000}}	&\num{4.1900}
	  \\
	  \hline
	\end{tabular}
	    \caption{Error en sentido débil para el segundo momento.}
	\end{table}
	}
  \end{overlayarea}
\end{frame}
%\end{bibunit}
  \section{Trabajo Actual}
  	%%%%%%%%%%%%%%%%%%%%%%%%%%%%%%%%%%%%%%%%%%%%%%%%%%%%%%%%%%%%%%%%%%%%
%\begin{bibunit}[apalike]
\begin{frame}%[label=frm:20]
    \frametitle{Conclusiones}
		\begin{itemize}
			\item<+-> Al usar el promedio de Steklov obtenemos esquemas \alert{$A$-estables en media}.
			\item<+> Los esquemas satisfacen consistencia lineal en media cuadrática.	
		\end{itemize}
\end{frame}
%\end{bibunit}
%%%%%%%%%%%%%%%%%%%%%%%%%%%%%%%%%%%%%%%%%%%%%%%%%%%%%%%%%%%%%%%%%%%
%%%%%%%%%%%%%%%%%%%%%%%%%%%%%%%%%%%%%%%%%%%%%%%%%%%%%%%%%%%%%%%%%%%
%\begin{bibunit}[apalike]
\begin{frame}%[label=frm:20]
    \frametitle{Conclusiones}
		\begin{itemize}
			\item<+-> Analizar propiedades teóricas de las alternativas mencionadas.
			\item<+> Estudiar la estabilidad para EDEs no lineales desde la teoria de Liapunov.
		\end{itemize}
\end{frame}
%\end{bibunit}
%%%%%%%%%%%%%%%%%%%%%%%%%%%%%%%%%%%%%%%%%%%%%%%%%%%%%%%%%%%%%%%%%%%
  	%%%%%%%%%%%%%%%%%%%%%%%%%%%%%%%%%%%%%%%%%%%%%%%%%%%%%%%%%%%%%%%%%%%
%\begin{bibunit}[apalike]
\begin{frame}%[label=frm:20]
	\frametitle{Trabajo Actual \cite{hutzenthaler2015numerical}}
		\begin{overlayarea}{\linewidth}{.3\textheight}
			\only<2>{
				\begin{empheq}[box=\shadowbox*]{align*}
					dy_1(t) &=
						y_1(t) \left[1 - y_1(t) + 2 y_2(t) \right]dt + \varepsilon y_1^2(t) dW_1(t), \\
					dy_2(t) &=
						y_2(t) \left[1 - 2 y_2(t) + 2 y_1(t)\right]dt + \varepsilon y_2^2(t) dW_2(t).
				\end{empheq}
		}
		\only<3>{
			\vspace{-.5cm}
			\begin{empheq}[]{align*}
				dy_1(t) &=
				\left(
				\lambda -\delta y_1(t) - (1 - \gamma) \beta y_1(t) y_3(t)
				\right)dt,
				-\sigma_1 y_1(t) dW^{(1)}_t, 
				\notag \\
				dy_2(t) &= 
				\left(
				(1- \gamma) \beta y_1(t) y_3(t) - a y_2(t) 
				\right)dt,
				-\sigma_1 y_2(t) dW^{(1)}_t, 
				\\
				dy_3(t) & = 
				\left(
				(1 - \eta) N a y_2(t) 
				-u y_3(t)
				-(1 - \gamma ) \beta y_1(t) y_3(t) 
				\right)dt
				- \sigma_2 y_3(t) dW^{(2)}_t.
				\notag
				\end{empheq}
		}
		\only<4>{
			\begin{empheq}[box=\shadowbox]{equation*}
				dX_t=- X_t^5 dt + X_tdW_t, \qquad X_0=1
			\end{empheq}
			}
		\only<5>{
			\begin{empheq}[box=\shadowbox]{align*}
				dy(t)&= \left[by(t)-a y(t)^2\right]dt + \sigma y(t) dW_t,\\
				y(t)&= \frac{
						y_0
						\exp\left[
						b-\frac{1}{2}\sigma^2) t
						+\sigma W_t
						\right]
					}{
					1+a 
					y_0
					\int_0^t
					\exp\left[
					(b -\frac{1}{2})s
					+\sigma W_s
					\right]ds
						}
			\end{empheq}
		}
		\only<6>{
			\begin{empheq}[box=\shadowbox]{align*}
				dX_t^{(1)}&= X_t^{(2)} dt\\
				dX_t^2&=\left\{
				\mu X_t^{(2)}(1 -(X_t^{(1)})^2)-X_t^{1}
				\right\}dt
				+\sigma dW_t
			\end{empheq}
			}
			\only<7>{
				\vspace{-.3cm}
				\begin{empheq}[box=\shadowbox]{equation*}
					dy(t) = -y^3 dt 
					+ \frac{1}{\left[\log(t+1)\right]^{1.1}} dW_t, \qquad t>0.
				\end{empheq}
		}
		\end{overlayarea}
%	
%	
%	
	\begin{overlayarea}{\linewidth}{.8\textheight}
		\only<2>{
			\includegraphics[width=0.5\linewidth]{images/StoDetX1Mao.png}
			\includegraphics[width=0.5\linewidth]{images/StoDetX2Mao.png}
		}
		\only<3>{
			\begin{center}
				\includegraphics[width=0.8\linewidth]{images/InternalHIVDynamics5e-1}
			\end{center}
		}
		\only<4>{
			\begin{center}
				\includegraphics[width=0.7\linewidth]{images/MSETime2}
			\end{center}
		}
		\only<5>{
			\begin{center}
				\includegraphics[width=0.7\linewidth]{images/PathsLVEv4}
			\end{center}
		}
		\only<6>{
			\begin{center}
				\includegraphics[width=0.3\linewidth]{images/PhasePotraitSimDuffingVanDerPol.png}
				\includegraphics[width=0.3\linewidth]{images/X1SimDuffingVanDerPol}
				\includegraphics[width=0.3\linewidth]{images/X2SimDuffingVanDerPol}
			\end{center}
		}
		\only<7>{
			\vspace{-1.1cm}
			\begin{center}
				\includegraphics[width=0.45\linewidth]{images/pathsAppleby2}
			\end{center}
		}
	\end{overlayarea}
\end{frame}
%\end{bibunit}
%%%%%%%%%%%%%%%%%%%%%%%%%%%%%%%%%%%%%%%%%%%%%%%%%%%%%%%%%%%%%%%%%%%
%%%%%%%%%%%%%%%%%%%%%%%%%%%%%%%%%%%%%%%%%%%%%
\begin{frame}[allowframebreaks]{Referencias}
	\bibliographystyle{apalike}
	\bibliography{BibliografiaTesis}
\end{frame}
%\appendix
%	\part{Definiciones y resultados}
%	\begin{frame}
    \frametitle{Movimiento Browniano}
  \begin{alertblock}{Propiedades básicas}
        Un movimiento Browniano unidimensional de parámetro $\sigma^2$ es
        un proceso estocástico ${B_t : t ≥ 0}$ con valores en $R$ que cumple las siguientes
        propiedades.
      \begin{itemize}
        \item $B_0=0$  c.s.
          Las trayectorias $ t \mapsto B_t$ son continuas.
        \item
          El proceso tiene incrementos independientes.
        \item
          La variable aleatoria $B_t-B_s$ tiene distribución normal $N(0,\sigma^2(t-s))$ para cualquier tiempo
          $0\leq s\leq t$
     \end{itemize}
    \end{alertblock}
\end{frame}
%%%%%%%%%%%%%%%%%%%%%%%%%%%%%%%%%
\begin{frame}
   \frametitle{Constante de Boltzmann}
    \begin{alertblock}{Relaciona temperatura absoluta y energía de un sistema.}
      $$k_b\approx \SI{1.3806504e-23}{\frac{\joule}{\kelvin}}$$
    \end{alertblock}
\end{frame}
%%%%%%%%%%%%%%%%%%%%%%%%%%%%%%%%%
\begin{frame}
   \frametitle{Ley de Stockes}
    \begin{exampleblock}{$a:$ radio, $\eta:$ viscosidad}
      La ley de Stokes se refiere a la fuerza de fricción experimentada por objetos esféricos moviéndose en el
      seno    de un fluido viscoso en un régimen laminar de bajos números de Reynolds.
      $F_c=6\pi a \eta v$
   \end{exampleblock}
\end{frame}
%%%%%%%%%%%%%%%%%%%%%%%%%%%%%%%%
\begin{frame}
   \frametitle{Respecto al potencial}
    \begin{block}{Interacción aditiva}
      $$
       \mathbf{F}_i= -\nabla_i U=-\nabla_i \sum_{i\neq j}^N V(\mathbf{r}_i-\mathbf{r}_j)
      $$
    \end{block}
    En la literatura revisada usan:
    \begin{exampleblock}{(Potencial de interacción a pares tipo Yukawa)}
        $$
          V(r)=\frac{V_0}{r} exp[-\lambda(r-1)]
        $$
    \end{exampleblock}
\end{frame}
\end{document}