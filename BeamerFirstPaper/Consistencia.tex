\begin{bibunit}[apalike]
  \begin{frame}%[label=frm:10]
    \frametitle{Definiciones y resultados previos}
	\cite{KloedenPllaten}
	\biblio{BibliografiaTesis}
  \end{frame}
\end{bibunit}
%%%%%%%%%%%%%%%%%%%%%%%%%%%%%%%%%%%%%%%%%%%%%%%%%%%%%%%%%%%%%%%%%%%%%%%%%%%%%%%%%%%%%%%%%%%%%%
\begin{frame}
    \frametitle{Consistencia convergencia y estabilidad en sentido fuerte}
	\vspace{-1.0cm}	
	\begin{empheq}[box={\Garybox[EDE]}]{equation*}
		dX_t=a(X_t)dt+b(X_t)dB_t, \qquad X_0=x_0 \quad (EDE)
	\end{empheq}		
	\vspace{-.45cm}	
	\begin{columns}
		\column{.3\textwidth}
			$Y^{h}$ 	esquema con paso 	 $\max h$.		
		\column{.5\textwidth}
		\begin{empheq}[box=\shadowbox]{equation*}
			\varepsilon(h)= \mathbb{E}
		  \left(
		  	X_T-Y^{h}(T)
		  \right)
		\end{empheq}		
	\end{columns}	
	\vspace{-.5cm}	
	\begin{overlayarea}{\textwidth}{.7\textheight}
	\only<+->{
	  \begin{definicion}[Consistencia]
		$Y^{h}$  a los	  tiempos
		$
		  \left(
			\tau
		  \right)_{h}=
		  \left\{
			\tau_{n}:n=0,1,\cdots
		  \right\}
		$
		es \textcolor{red}{consistente en sentido fuerte},
		si $\exists C=C(h)\geq 0,\quad h_0$ t.q. 
		$\textcolor{red}{\forall Y_n^{h}},n=1,2,\cdots N \quad h\in(0,h_0)$
		\begin{itemize}[<+->]
		  \item $\displaystyle \lim_{h\downarrow 0} C(h)=0$
		  \item
		  $
			\displaystyle
			 \mathbb{E}
			  \left(
				\left|
				  \mathbb{E}
				  \left(
					\frac{Y_{n+1}^{h}-Y_n^{h}}{h_n}
					  \left|
						\mathcal{F}_{\tau_n}
					  \right.
				  \right)
				-a\left(
					Y_{n}^{h}
				  \right)
				\right|^2
			 \right)\leq C(h).
		  $
		  \item
			\renewcommand{\arraystretch}{1.5}%
			\scalebox{0.8}{% Scale by 50%
			$
			  \mathbb{E}
			  \left(
				\frac{1}{h_n}
				\left|
				Y_{n+1}^{h}-Y_{n}^{h}-
				\mathbb{E}
				\left(
				  \frac{Y_{n+1}^{h}-Y_n^{h}}{h_n}
					\left|
					\mathcal{F}_{\tau_n}
				\right.
			  \right)
			-b\left(Y_{n}^{h}
			\right)\Delta B_n
			\right|^2
		  \right)\leq C(h).
		 $ }
		\normalsize
		\end{itemize}
	  \end{definicion}
		}
	\end{overlayarea}
\end{frame}
%%%%%%%%%%%%%%%%%Convergencia%%%%%%%%%%%%%%%%%%%%%%%%
\begin{frame}
    \frametitle{Consistencia convergencia y estabilidad en sentido fuerte}
	\vspace{-1.0cm}	
	\begin{empheq}[box={\Garybox[EDE]}]{equation*}
		dX_t=a(X_t)dt+b(X_t)dB_t, \qquad X_0=x_0 \quad (EDE)
	\end{empheq}		
	\vspace{-.45cm}	
	\begin{columns}
		\column{.3\textwidth}
			$Y^{h}$ 	esquema con paso 	 $\max h$.		
		\column{.5\textwidth}
		\begin{empheq}[box=\shadowbox]{equation*}
			\varepsilon(h)= \mathbb{E}
		  \left(
		  	X_T-Y^{h}(T)
		  \right)
		\end{empheq}		
	\end{columns}	
	\vspace{-.5cm}	
	\begin{overlayarea}{\textwidth}{.7\textheight}
	 \only<+>{
	  \begin{definicion}[Convergencia fuerte]
		$Y^{h}$ \textcolor{red}{converge} en \textcolor{red}{sentido fuerte} a $X$ a tiempo $T$ si
		\begin{equation*}
		  \lim_{h \downarrow 0}
		  \mathbb{E}
		  \left(
		  X_T-Y^{h}(T)
		  \right)=0
		\end{equation*}
	  \end{definicion}
	  }
	\only<+>{
	  \begin{definicion}[orden de convergencia]
		$Y^{h}$
		\textcolor{red}{converge} en sentido fuerte \textcolor{red}{con orden $\gamma$},
		 si $\exists C$
		independiente de $h $ y $h_{0}$ t.q.
		\begin{equation*}
		  \epsilon(h)=
		  \mathbb{E}
		  \left(
			|X_T-Y(T)|
			\right)\leq C h^{\textcolor{red}{\gamma}} \qquad\forall h\in (0,h_0).
		\end{equation*}
	  \end{definicion}
	  }
	\end{overlayarea}
\end{frame}
%%%%%%%%%%%%%%%%%%%%%%%%%%%%%%%%%%%%%%%%%%%%%%%
\begin{frame}
	\frametitle{Consistencia convergencia y estabilidad en sentido fuerte}	
	\hypertarget{thm:ConsistenciaConvergencia}{}	
	%\vspace{-3.8cm}	
	\begin{empheq}[box={\Garybox[EDE]}]{equation*}
		dX_t=a(X_t)dt+b(X_t)dB_t, \qquad X_0=x_0 \quad (EDE)
	\end{empheq}
	\begin{overlayarea}{\textwidth}{\textheight}
 	 \begin{Teorema}
		Bajo las condiciones del teorema de
		\hyperlink{thm:ExistenciaUnicidadEDE}{ \textbf{existencia y unicidad}} (Lipschitz globales)
		para soluciones fuertes de (EDE). Si $Y^{h}$ es \textcolor{red}{consistente} entonces
		$Y^{h}$ \textcolor{red}{converge} en sentido fuerte a la solución $X_t$.
  \end{Teorema}
  \only<2->{
  \begin{Teorema}
	  Bajo las mismas hipótesis, el esquema Steklov converge \only<3->{con \textcolor{red}{ orden \num{0.5}-$\epsilon$}}.
  \end{Teorema}
	}
	\end{overlayarea}
\end{frame}
%%%%%%%%%%%%%%%%%%%%%%%%%%%%%%%%%%%%%%%%%%%%%%%%
