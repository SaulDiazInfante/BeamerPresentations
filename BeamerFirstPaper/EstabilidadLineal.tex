%----------------------------------------------------------------------------------------------
%\begin{bibunit}[apalike]
%\begin{frame}
%   \frametitle{Estabilidad lineal (métodos deterministas)}% \cite{dahlquist2008numerical}}
%   \begin{columns}
%   		\column{.4\textwidth}
%   		\begin{empheq}[box=\shadowbox]{align*}
%			\frac{dx}{dt}=&\lambda x,\\
%			 x(0)=&x_0 \quad\lambda \in \mathbb{C}.\\
%		\end{empheq}
%		\visible<2->{
%				\colorbox{yellow}{$x(t)=x_0\exp(\lambda t)$} \\
%			} 		
%		\visible<3->{
%		\begin{equation*}
%			\lim_{t\to \infty}x(t)=0\Leftrightarrow Re(\lambda)<0		
%		\end{equation*}							
%		}		
%		\column{.5\textwidth}
%		\begin{overlayarea}{1.1\textwidth}{.9\textheight}
%			\only<4>{
%				\begin{center}
%   					\includegraphics[width=\textwidth]{Imagenes/Estabilidad/EstabilidaLineal.png}
%   				\end{center}
% 			}
% 		\end{overlayarea} 	    		
%   \end{columns}
%   %\biblio{BibliografiaTesis}
%\end{frame}
%%\end{bibunit}
%%------------------------------------Metodo de Euler-------------------------------
%%%%%%%%%%%%%%%%%%%%%%%%%%%%%%%%%%%%%%%%%%%%%%%%%%%%%%%%%%%%%%%%%%%%
%\begin{frame}%[label=frm:17]
%  \frametitle{Estabilidad Lioneal (métodos deterministas)}% \cite{dahlquist2008numerical}}
%  \begin{columns}
%	\column{.4\textwidth}
%	  \colorbox{darkyellow}{
%			$
%				\displaystyle				
%				\lim_{t\to \infty}x(t)=0\Leftrightarrow Re(\lambda)<0
%			$
%	  }
%	\visible<+->{	
%		\begin{center}
%			Método de Euler	
%		\end{center} 
%		\begin{empheq}[box=\ovalbox]{equation*}	 
%			\frac{x_{n+1}-x_n}{h}=\lambda x_n
%		\end{empheq}	 
%	  }
%	  \visible<+->{
%		\begin{overlayarea}{.8\textwidth}{.3\textheight}
%%		  \only<+>{
%%			\begin{empheq}[box=\mybluebox]{align*}
%%		     % \begin{align*}
%%			  x_{n+1}=&\lambda x_n h+x_n\\
%%				=&
%%				  x_n(\lambda h +1)\\
%%				\vdots\\
%%				=&
%%				  x_0(\lambda h +1)^{n+1}
%%		    % \end{align*}
%%			\end{empheq}
%%		  }
%		  \only<+->{
%			\colorbox{greenArea}{
%			$
%			  \displaystyle
%			  \lim_{n\to \infty}x_n=0\Leftrightarrow |\lambda h +1|<1
%			$
%			}				
%		  }
%	  \end{overlayarea}
%		}
%	\column{.6\textwidth}
%	  \begin{overlayarea}{1.1\textwidth}{.9\textheight}
%		\only<+>{
%		  \begin{center}
%			\includegraphics[width=\textwidth]{Imagenes/Estabilidad/EstabilidadEuler1.png}
%		  \end{center}
%  		}
%  		\only<+>{
%		  \begin{center}
%			\includegraphics[width=\textwidth]{Imagenes/Estabilidad/EstabilidadEuler2.png}
%		  \end{center}
%  		}
%  		\only<+>{
%		  \begin{center}
%			\includegraphics[width=\textwidth]{Imagenes/Estabilidad/EstabilidadEuler3.png}
%		  \end{center}
%  		} 
%	  \end{overlayarea}
%  \end{columns}
%  %\biblio{BibliografiaTesis}
%\end{frame}
%%----------------------------------------------------------------------------
\begin{frame}%[label=frm:18]
   \frametitle{Estabilidad lineal EDEs}
	\begin{overlayarea}{\textwidth}{.3\textheight}
		\begin{empheq}[box={\Garybox[Consideremos]}]{equation*}\label{eqn:SDETest}
 			dX_t=\lambda X_t dt +\beta dB_t, \qquad X_0=cte., \lambda, \beta \in \mathbb{C} \tag{DP}.
		\end{empheq}
	\end{overlayarea}	
	\begin{overlayarea}{\textwidth}{.4\textheight}	
		\only<2>{
		\begin{block}{Solución exacta}
			\begin{empheq}{equation*}\label{eqn:SolSDETest}
				 X_t=X_0\exp(\lambda t) +\beta \int_{0}^t \exp(\lambda (t-s))dB_s \tag{*}
			\end{empheq}
			\begin{empheq}{align*}
			 	 \label{eqn:Mean}
		  		\mathbb{E}[X_t]=&\exp(\lambda t)\mathbb{E}[X_0]  \tag{**} \\
		  		\label{eqn:Variance}
		  		\mathbb{E}[|X_t|^2]=& \exp(2Re(\lambda)t) \mathbb{E}[|X_0|^2] -
		  		\frac{|\beta|^2}{2Re(\lambda)}(1-\exp(2Re(\lambda)t)) .\tag{***}
			\end{empheq}
	\end{block}
	}
	\only<3>{
		(DP) tiene solución asintótica-estable
 		\begin{equation*}
 			\colorbox{yellow}{$
 			\begin{matrix}
 		  		\lim_{t \to \infty} \mathbb{E} |X(t)|= 0 \\
 		  		\lim_{t \to \infty} \mathbb{E}[|X(t)|^2]= -\frac{|\beta|^2}{2 Re(\lambda)} %
 			\end{matrix}$}
 	      \Leftrightarrow Re(\lambda)<0
 		\end{equation*}	
	}
	\end{overlayarea}	
\end{frame}
%%----------------------------------------------------------------------------------------------------
%%\begin{bibunit}
\begin{frame}%[label=frm:19]
    \frametitle{Estabilidad Lineal}
	\begin{columns}
		\column{.5\textwidth}
		\begin{overlayarea}{\textwidth}{\textheight}			
			\only<+->{
				\begin{definicion}
  					$X(t)$ es asint\'oticamente estable en media (AEM) si y s\'olo si
					\begin{equation*}\label{eqn:ConMeanEstable}
	 					\lim_{t \to \infty}\mathbb{E}[|X(t)|]=0.
					\end{equation*}
				\end{definicion}
			}
			\only<+->{
			\begin{definicion}
				 $X(t)$ es asint\'oticamente estable en media  cuadr\'atica (AEMC) si y sólo si
				\begin{equation*}\label{eqn:ConMeanSquareEstable}
	 				\lim_{t \to \infty}\mathbb{E}[|X(t)|^2]=-\frac{|\beta|^2}{2Re(\lambda)}.
				\end{equation*}
			\end{definicion}
			}
		\end{overlayarea}
		\column{.5\textwidth}
		\only<+->{ 
				\begin{definicion}
  					La solución  numérica es asint\'oticamente estable en media (AEM) si y s\'olo si
					\begin{equation*}\label{eqn:ConMeanEstable}
	 					\lim_{n \to \infty}\mathbb{E}[|X_n|]=0.
					\end{equation*}
				\end{definicion}
				\begin{definicion}
	  				La solución numérica es asint\'oticamente estable en media  cuadr\'atica (AEMC) si y sólo si
					\begin{equation*}\label{eqn:ConMeanSquareEstable}
		 				\lim_{n \to \infty}\mathbb{E}[|X_n|^2]=-\frac{|\beta|^2}{2Re(\lambda)}.
					\end{equation*}
				\end{definicion}
			}
	\end{columns}	
%%\biblio{BibliografiaTesis}
\end{frame}
%%\end{bibunit}
%%%%%%%%%%%%%%%%%%%%%%%%%%%%%%%%%%%%%%%%%%%%%%%%%%%%%%%%%%%%%%%%%%%%
%\begin{bibunit}[apalike]
\begin{frame}%[label=frm:20]
    \frametitle{$A$-estabilidad}
	\begin{overlayarea}{\textwidth}{\textheight}
		\only<1>{
		\begin{block}{Buscamos $\lambda h$}
			para los cuales el m\'etodo Steklov \textcolor{red}{reproduce
			correctamente} la  \textcolor{cyan}{estabilidad en media y media cuadr\'atica.}
		\end{block}
		}
		\only<2>{
		\begin{definicion}
			Diremos que un método será \textcolor{red}{$A$-estable} en \textcolor{red}{media o media cuadrática} si \cite{higham2000stability}
			$$
				\text{Problema estable } \Rightarrow \text{ método estable }\forall h
			$$
		\end{definicion}
		}
	\end{overlayarea}
\end{frame}
%\end{bibunit}
%%---------------------------Resultado sobre Media---------------------------------------------------------
\begin{frame}%[label=frm:20]
	\begin{empheq}[box=\shadowbox*]{equation*}
				dX_t=\lambda X_t dt +\beta dB_t, \quad X_0=cte, \lambda, \beta \in
				\mathbb{C} \qquad \text{(DP)}
	\end{empheq}
%%-------------------------------------------------------------------------------------------------------------------------------	
	\begin{columns}		
		\column{.5\textwidth}
		\begin{overlayarea}{\textwidth}{\textheight}
			\only<2-6>{
				\centering{\alert<3-4>{\textbf{Euler-Mayurama}}}
				\begin{empheq}[box={\ovalbox}]{align*}
					X_{n+1} = &(1+\lambda h) X_n +\beta \Delta B_n \\
					\Delta B_n=& B_{t_{n+1}}-B{t_n}\\
					B_{n}=&\sqrt{h}V_n\\
						V_n\sim& N(0,1).
				\end{empheq}			
			}
			\only<4-6>{
				\begin{empheq}{align*}
					\mathbb{E}(X_{n+1})=\mathbb{E}(X_0)(1+\lambda h)^{n+1}
				\end{empheq}				
			}
			\only<7>{
				\begin{Teorema}
					El método Steklov para la ecuación (DP)	es $A$-estable en media.
				\end{Teorema}
			}	
		\end{overlayarea}
		\column{.5\textwidth}
			\begin{overlayarea}{\textwidth}{\textheight}
			\only<5-7>{
				\centering{\alert<5->{\textbf{Steklov}}}					
				\begin{align*}
					X_{n+1} = & X_n \exp(\lambda h)+\beta \Delta B_n \\
					\Delta B_n=& B_{t_{n+1}}-B{t_n}\\
					B_{n}=&\sqrt{h}V_n\\
						V_n\sim& N(0,1).
				\end{align*}
				}
				\only<6>{
					\begin{empheq}[box=\myyellowbox]{align*}
						E(X_{n+1})=&\mathbb{E}(X_0)\exp((n+1)\lambda h)
					\end{empheq}
				}
		\end{overlayarea}
	\end{columns}
\end{frame}
%%%%%%%%%%%%%%%%%%%%%%%%%%%%%%%%%%%%%%%%%%%%%%%%%%%%%%%%%%%%%%%%%%%%%%%%%%%%%%%%%%%%%%%%%%%%%%%%%
